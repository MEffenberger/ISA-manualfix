\documentclass[a4paper,11pt]{article}

\usepackage[utf8]{inputenc}
\usepackage[czech]{babel}
\usepackage[left=2cm,top=3cm,text={17cm,24cm}]{geometry}
\usepackage{graphicx}
\usepackage{listings}
\usepackage{url}
\usepackage{multirow}

\title{Zabezpečený přenos dat\\
{\bf\large ISA - Laboratorní cvičení č.2}\\
{\bf\large Prokotol ke cvičení}}

\author{Vysoké učení technické v Brně}

\date{\url{https://github.com/nesfit/ISA/tree/master/lab2-zabezpeceni}}

\setlength\parindent{0pt}

\begin{document}

{\let\newpage\relax\maketitle}

Jméno a příjmení:\\
Login:\\
Skupina (číslo nebo čas):\\
Datum:

\section{Synchronizace času pomocí NTP}
{\bf 1.2} Nakonfigurované servery NTP:\\
        
\begin{tabular}{|c|c|c|c|}
\hline
M & S & \hspace{0.5cm}Server (doménové jméno či IP adresa)\hspace{0.5cm} & Stratum \\ \hline
\hspace{0.5cm} & \hspace{0.5cm} & & \\ \hline
& & & \\ \hline
& & & \\ \hline
& & & \\ \hline
\end{tabular}

\vspace{1em}
Význam příznaku M: ~\\ Význam příznaku S: \\
Význam hodnoty Stratum:

\section{Zabezpečený vzdálený přístup pomocí SSH}
\textbf{2.4 e)} Algoritmus pro výměnu klíčů:\\
\textbf{2.4 f)} Uveďte, jaké zajímavé informace lze vyčíst z odchycené komunikace SSH:

\vskip 1.5em
%\underline{\hspace{16cm}}
%~\\

\textbf{2.5 d)} Vyplňte informace o vygenerovaných klíčích SSH:\\

\begin{tabular}{|r|c|c|c|}
\hline
~ & \hspace{2.7cm}Význam\hspace{2.7cm}  & Název souboru & Přístupová práva \\
\hline
\multirow{2}{*}{Veřejný klíč:} & & & \\[1.3em]
\hline
\multirow{2}{*}{Privátní klíč:} & & & \\[1.3em]
\hline
\end{tabular}
\vspace{1em}

\textbf{2.6 c)-d)} Vyplňte hesla požadovaná při přihlášení na vzdálený počítač:\\
~\\
\begin{tabular}{|l|c|c|}
\hline
~ & user & root \\
\hline
Heslo vyžadované při distribuci klíčů: & \hspace{4cm} & \hspace{4cm} \\
\hline
Heslo vyžadované při druhém přihlášení: & & \\
\hline
\end{tabular}
\newpage
%\bigskip

K čemu slouží soubor \texttt{.ssh/authorized\_keys}?\\

\vskip 2em
Jak je možné přihlásit se na vzdálený počítač po neúspěšné autentizaci klíčem SSH?\\

\vskip 2em

\textbf{2.7 d)} Co se zobrazilo na výstup terminálu při přihlášení z účtu {\tt root}?\\

\vskip 2.5em

\textbf{2.8 c)} Jak proběhlo opakované přihlášení na vzdálený počítač pomocí SSH? \\

\vskip 2.8em

\section{Zabezpečení transportní vrstvy pomocí TLS}

\textbf{3.1 e)} Uveďte, zda je možné vidět obsah odchycené komunikace HTTP a proč ano/ne?\\

\vskip 2em

{\bf 3.2 c)} Jak lze zjistit z odchycené komunikace TLS jméno cílového serveru?\\

\vskip 2em

{\bf 3.2 e)} Je možné přečíst obsah komunikace a proč ano/ne?\\

\vskip 2em

{\bf 3.2 g)} Vyplňte informace o použité šifrovací sadě (cipher suite):\\

\renewcommand\arraystretch{1.3}
\begin{tabular}{|l|r|}
%\hline
\hline
Šifrovací sada (ID, název): & \hspace{25em} \\ \hline
Algoritmus pro výměnu klíčů: &  \\ \hline
Algoritmus pro zajištění autentizace: & \\ \hline
Algoritmus pro šifrování přenosu: & \\ \hline
Délka šifrovacího klíče: & \\ \hline
Hešovací algoritmus: & \\ \hline
\end{tabular}
\renewcommand\arraystretch{1}
\vspace{0.5cm}

\textbf{3.3 b)} Použitá šifrovací sada v prohlížeči (name): \underline{\hspace{7cm}} \\ 

%\vskip 1em
Informace o certifikátu:\\

\renewcommand\arraystretch{1.3}
\begin{tabular}{|l|r|}
\hline
Vydavatel certifikátu: & \hspace{30em} \\ \hline
Doba platnosti: & \\ \hline
\end{tabular}
\renewcommand\arraystretch{1}
\vspace{0.5cm}\\

{\bf 3.3 c)} Příklady důvěryhodných CA:\\

\end{document}
%% END OF FILE


\documentclass[a4paper,11pt]{article}

\usepackage[utf8]{inputenc}
\usepackage[czech]{babel}
\usepackage[left=2cm,top=3cm,text={17cm,24cm}]{geometry}
\usepackage{graphicx}
\usepackage{listings}
\usepackage{url}
\usepackage{xr}


\lstset{
  basicstyle=\ttfamily,
  columns=fullflexible,
}

\externaldocument{../manual/manual}

\title{Zabezpečený přenos dat\\
{\bf\large ISA - Laboratorní cvičení č.2}}

\author{Vysoké učení technické v Brně}

\date{\url{https://github.com/nesfit/ISA/tree/master/lab2-zabezpeceni}}

\begin{document}

{\let\newpage\relax\maketitle}

\section*{Cíl cvičení}
\begin{itemize}
  \item Seznámit se se synchronizací času a protokolem NTP (Network Time Protocol).
  \item Nakonfigurovat zabezpečený vzdálený přístup pomocí nástroje SSH (Secure Shell).
  \item Analyzovat komunikaci zabezpečenou protokolem TLS (Transport Layer Security).
\end{itemize}

\section*{Obecné pokyny}
\begin{itemize}
  \item Vypněte si mobily a uschovejte do tašek. Není dovoleno používat mobily během výuky. 
  \item Přihlaste se v OS Linux (F3) jako uživatel {\tt user} s heslem {\tt user4lab}.
  \item V případě potřeby se přepněte na uživatele {\tt root} příkazem {\tt su}
  (switch user), heslo {\tt root4lab}.
  \item Pro editaci konfiguračních souborů můžete použít libovolný editor (nano, mcedit, vim, gedit).
  \item Při práci budete používat jméno počítače ve formátu hXX, kde XX je číslo vašeho počítače.
\end{itemize}

%\newpage
%\section{Laboratorní úlohy}
\section{Synchronizace času pomocí protokolu NTP}
Pro automatizovanou synchronizaci hodin na síťových zařízeních se využívá protokol NTP (Network Time Protocol), viz kapitolu \ref{ntp} v laboratorním manuálu. Správná synchronizace  času je nezbytná pro zajištění bezpečnosti zejména kvůli časovým razítkům omezujícím platnost klíčů, certifikátů či elektronických podpisů.

\begin{enumerate}
  \item Příkazem {\tt systemctl status chronyd.service} zjistěte, zda na počítači běží služba \texttt{chronyd} pro synchronizaci času pomocí protokolu NTP. Příkazem {\tt chronyc -n sources} zkontrolujte seznam serverů, které váš počítač používá
    pro synchronizaci času.

  \item Do protokolu vyplňte IP adresy nakonfigurovaných serverů NTP. 
  U každého serveru vyplňte nastavení příznaků {\em M}, {\em S} a hodnotu {\em Stratum}.
  V protokolu vysvětlete význam těchto hodnot, viz \texttt{man chronyc}.

  \item Jako uživatel \texttt{root} otevřete konfiguračního soubor {\tt /etc/chrony.conf}. V souboru
    nahraďte implicitní server NTP CentOS ({\tt *.centos.pool.ntp.org}) za server na FIT ({\tt ntp.fit.vutbr.cz}), viz {\tt man chrony.conf}.  
  \item Restartujte démona NTP příkazem {\tt systemctl restart chronyd.service}.
  \item Ověřte pomocí příkazu {\tt systemctl status chronyd.service}, zda služba běží. Pokud neběží, zjistěte chybu příkazem {\tt journalctl -u chronyd}. Případnou chybu opravte v konfiguraci a~službu {\tt chronyd} znovu restartujte.
  \item Pomocí příkazu {\tt chronyc -n sources} ověřte, zda je čas správně synchronizován se serverem NTP na FIT.

\end{enumerate}

\section{Zabezpečený vzdálený přístup pomocí SSH (práce ve dvojicích)}
% Poznámka: namísto řetězce {\tt <login>} používejte v této úloze své studentské přihlašovací jméno, například {\tt xstudent00}.
Nástroj SSH (Secure Shell) se používá pro bezpečné připojení na vzdálený počítač. Více podrobností ke službě SSH najdete v kapitole \ref{ssh} laboratorního manuálu.
\begin{enumerate}
  \item Otevřete si dvě terminálová okna: příkazovou řádku pro uživatele {\tt user} a další
    příkazovou řádku pro uživatele {\tt root} příkazem {\tt su} (switch user).
  \item Pomocí příkazu {\tt whoami} ověřte jméno aktivního uživatele. 
  \item V~obou terminálových oknech dočasně vypněte podporu pro SSH agenta příkazem {\tt unset}:
  \begin{lstlisting}
  [user@hXX]$ unset SSH_AUTH_SOCK
  [root@hXX]# unset SSH_AUTH_SOCK
  \end{lstlisting}

  Pokud budete otevírat v průběhu řešení nové terminály, nezapomeňte v nich vypnout podporu agenta SSH.

  \item Bezpečné připojení na vzdálený počítač bez autentizačních klíčů (jako uživatel {\tt user}):
    \begin{enumerate}
      \item Spusťte program Wireshark a zachytávejte komunikaci na portu 22 (rozhraní enp2s0).
      \item Přihlaste se pomocí {\tt ssh} na počítač svého souseda hYY (např. h01): příkaz
        {\tt ssh user@hYY. netlab.fit.vutbr.cz}. Zadejte heslo uživatele {\tt user} v OS {\tt user4lab}.
      \item Na vzdáleném počítači zadejte libovolný příkaz, např. výpis manuálové stránky ssh ({\tt man ssh}), obsahu adresáře ({\tt ls}) či login aktuálně uživatele ({\tt whoami}).
      \item Příkazem {\tt exit} nebo stiskem {\tt Ctrl-D} ukončete spojení SSH.
      \item V programu Wireshark zobrazte obsah komunikace SSH pomocí {\tt Follow TCP stream}. Podí\-vej\-te se, co lze ze zobrazené komunikace vyčíst. Výsledek zapište do protokolu.
    \end{enumerate}

  \item Vytvoření veřejného a privátního klíče pro službu SSH:

  {\em Pozn.:} Klíče budete vytvářet na {\em svém počítači}. Podle promptu \verb|uzivatel@hXX| si vždy ověřte, zda jste přihlášen jako uživatel {\tt user} nebo {\tt root} a zda \verb|XX| odpovídá číslu vašeho počítače.
    \begin{enumerate}
      \item Jako uživatel {\tt user} vygenerujte příkazem \verb|$ ssh-keygen -C <login>@user|\footnote{Řetězec {\tt <login>} nahraďte svým loginem, např. xstudent00.} implicitní klíč pro uživatele {\tt user}. Ponechte implicitní názvy souborů pro uložení klíčů. Zvolte heslo o délce alespoň osmi znaků, například \texttt{fitvutisa}. Heslo se používá pro přístup k soukromému klíči.
      \item Jako uživatel {\tt root} vygenerujte příkazem \verb|# ssh-keygen -N "" -C <login>@root|
        implicitní klíč pro uživatele {\tt root} bez hesla.
      \item Ověřte obsah a přístupová práva nově vzniklých souborů (\verb|ls -l ~/.ssh|).
      \item Do protokolu uveďte, jaký je význam obou klíčů, v jakých souborech jsou uloženy a jak je nastaven přístup k těmto souborům (unixová práva), viz {\tt man ssh}.
    \end{enumerate}

  \item Distribuce klíčů:
    \begin{enumerate}
      \item Oba veřejné klíče zkopírujte na vzdálený počítač hYY do souboru \verb|.ssh/authorized_keys|. Můžete využít následující postup:
        \begin{itemize}
          \item Klíč uživatele {\tt user} zkopírujte příkazem {\tt ssh-copy-id}, viz {\tt man ssh-copy-id}: \\
            {\verb&ssh-copy-id -i ~user/.ssh/id_rsa.pub user@hYY&}
          \item Na vzdáleném počítači nastavte správně přístupová práva k souboru:\\
            {\verb&ssh user@hYY "chmod g-w ~/.ssh/authorized_keys"&} 
          \item Zkopírujte na vzdálený počítač také veřejný klíč uživatele {\tt root} (pod uživatelem {\tt root}): \\
            {\verb&ssh-copy-id -i /root/.ssh/id_rsa.pub root@hYY&}
          \item Zkontrolujte obsah obou souborů a ověřte, zda jsou klíče přidány: \\
            {\verb&ssh user@hYY "cat ~user/.ssh/authorized_keys"&} \\
            {\verb&ssh root@hYY "cat /root/.ssh/authorized_keys"&}
        \end{itemize}
      \item Zkuste se znovu přihlásit jako \texttt{user} i jako \texttt{root} na vzdálený počítač hYY.
      Vzdálená sezení opět ukončete příkazem {\tt exit} nebo stiskem {\tt Ctrl-D}.

      \item Do protokolu uveďte, jaká hesla bylo nutné zadat při přihlášení. Vysvětlete význam souboru \texttt{authorized\_keys}.
      \item Zkuste opakovaně zadat špatné heslo pro přístup ke klíči. Do protokolu napište, jaké heslo bude po vás požadovat SSH k připojení na vzdálený počítač. Při experimentu použijte režim verbose (\texttt{ssh -v}).
    \end{enumerate}

  \item Použití služby SSH pro vzdálené spuštění příkazu:
    
    Nyní omezíme použití klíče uživatele {\tt root} pouze na spuštění konkrétního příkazu na vzdáleném počítači.
    \begin{enumerate}
    \item Přihlaste se jako uživatel {\tt root} na sousední počítač hYY, kam jste nakopírovali své veřejné klíče.
    \item Na vzdáleném počítači upravte soubor s autorizovanými veřejnými klíči tak, že na začátek řádku s klíčem uživatele {\tt root}\footnote{Řádek poznáte tak, že končí řetězcem {\tt <login>@root}.} přidáte příkaz 
        \verb|command="chronyc -n sources"| následova\-ný mezerou a původním obsahem řádku. Tento příkaz se spustí místo vzdáleného příhlášení. 
      \item Odhlaste se ze vzdáleného počítače.
      \item Znovu se přihlaste na sousední počítač hYY pomocí SSH z účtu {\tt root}. Výstup z terminálu po přihlášení zapište do protokolu.
    \end{enumerate}
  \item Opakované použití klíče zabezpečeného heslem pomocí agenta SSH.
    \begin{enumerate}
      \item Otevřete si \textbf{nový unixový terminál} pod uživatelem {\tt user}. Pomocí příkazu \verb|env| ověřte, jak je nastavená proměnná \verb|SSH_AUTH_SOCK|.
      \item Přihlaste se znovu k sousednímu počítači hYY.
      \item Odhlaste se a znovu přihlašte na sousední počítač. Jak proběhlo přihlášení? Výsledek zapište do protokolu.
    \end{enumerate}
\end{enumerate}

\section{Zabezpečení transportní vrstvy pomocí TLS}
V této budeme analyzovat zabezpečený přenos pomocí TLS (Transport Layer Security). Více podrobností k TLS najdete v laboratorním manuálu, kapitole \ref{tls}. 
\begin{enumerate}
  \item Nezabezpečený přenos dat.
    \begin{enumerate}
      \item Spusťte program Wireshark a zachytávejte komunikaci na portu 80.
      \item Pomocí programu {\tt telnet} se připojte k fakultnímu webovému
        serveru: \\ \verb|telnet nes.fit.vutbr.cz 80|.
      \item V programu Wireshark pozorujte navázání spojení pomocí trojcestné synchronizace TCP.
      \item Zašlete serveru požadavek protokolem HTTP, např. \verb|GET / HTTP/1.0|. Dotaz ukončete prázdným řádkem -- tj. je nutné 2x stisknout ENTER. V terminálu pozorujte odpověď.
      \item Zobrazte komunikace HTTP v programu Wireshark.  Do~protokolu uveďte, zda je možné vidět obsah odchycené komunikace.
    \end{enumerate}

  \item Přenos dat zabezpečený protokolem TLS.
    \begin{enumerate}
      \item Spusťte program Wireshark a odchytávejte komunikaci na portu 443.
      \item Pomocí programu {\tt openssl} se připojte k fakultnímu webovému  serveru: \\ \verb|openssl s_client -connect www.fit.vutbr.cz:443 -tls1_2|. Všimněte si řádku \emph{Verify return code}. Co říká o certifikátu?

      \item V programu Wireshark pozorujte navázání spojení TCP a TLS. Zaměřte se na pakety {\tt Client Hello} a {\tt Server Hello}. Jak je možné z odchycené komunikace zjistit jméno serveru? Zapište odpověď do protokolu.

      \item Pomocí programu \texttt{openssl} zašlete webovému serveru FIT požadavek protokolem HTTP, např. \verb|GET / HTTP/1.0|. Dotaz ukončete prázdným řádkem (opět 2x ENTER). V terminálu pozorujte odpověď.

      \item Zobrazte tuto odchycenou komunikaci ve Wiresharku.  Do~protokolu uveďte, zda je možné přečíst obsah komunikace. 

      \item Z výstupu aplikace \texttt{openssl} určete, jaká šifrovací sada (cipher suite) se používá. Identifikátor šifrovací sady zapište do protokolu.

      \item Na stránce \url{https://ciphersuite.info/} vyhledejte detaily k dané šifrovací sadě. Do protokolu vyplňte názvy použitých algoritmů pro zabezpečení komunikace TLS.
    \end{enumerate}

  \item {\bf Zabezpečený přenos dat TLS ve webovém prohlížeči}

    \begin{enumerate}
      \item V prohlížeči Firefox otevřete stránku  \verb|wis.fit.vutbr.cz|.
      \item Klikněte na ikonku zámku nalevo od URL a~zjistěte, jaká šifrovací sada je použita pro dané zabezpečené připojení k serveru. Zobrazte informace o použitém certifikátu.
      Požadované údaje vyplňte do protokolu.
      \item V menu \emph{Edit » Settings » Privacy \& Security} v sekci
        \emph{Certificates} zobrazte certifikáty autorit, kterým důvěřujete (záložka {\em Authorities}). Do protokolu vypište názvy alespoň tří z nich.
    \end{enumerate}

%%  \item {\bf Certificate Transparency}
%%
%%    \begin{enumerate}
%%
%%      \item Příkazem \verb|host -t CAA fit.vutbr.cz| si zobrazte certifikační
%%        autority oprávněné vydávat certifikáty pro doménu \url{fit.vutbr.cz}.
%%
%%      \item V prohlížeči se připojte na stránku
%%        \url{https://certstream.calidog.io/}, klikněte na tlačítko \emph{Open
%%        Fire Hose}. Pozorujte, \textbf{jaké informace} se zde zobrazují.
%%        Zamyslete se, k čemu mohou tyto informace sloužit, případně jak se dají zneužít. {\bf Zapište výsledek do protokolu.}
%%      \item V prohlížeči se připojte na stránku \url{https://crt.sh}.
%%
%%      \item Zobrazte certifikáty vydané pro servery v  doméně \emph{kralovopole.brno.cz}. Podívejte se na jejich obsah, vydavatele a platnost. 
%%
%      \item Najděte na stránce ikonu směřují k souboru \textbf{Atom}.
%      Podívejte se, co soubor obsahuje a zodpovězte otázku v protokolu.
%      Zamyslete se, k čemu by se takový obsah dal použít.
%%   \end{enumerate}
\end{enumerate}


\section{Ukončení práce v laboratoři}
\begin{itemize}
  \item Odevzdejte vyplněný protokol vyučujícímu. 
  \item Jako uživatel \texttt{root} spusťte dávku {\tt /root/isa2/clean}, která odstraní vaši konfiguraci a vypne počítač.
\end{itemize}

\end{document}
%% END OF FILE

Jméno:\\
Login:\\

\section{Zjišťování konfigurace}
\textbf{1.}\\
Rozhraní enp2s0\\
\\
\begin{tabular}{|l|r|}
\hline
MAC adresa: & \hspace{12em} \\
\hline
IPv4 adresa: & \hspace{12em} \\
\hline
Délka prefixu: & \\
\hline
Adresa sítě: & \\
\hline
Broadcastová adresa: & \\
\hline
\end{tabular}
\vspace{1em}
\\
\textbf{2.}\\
IPv4 adresa výchozí brány:\\
MAC adresa výchozí brány:\\
\\
\textbf{3.}\\
Použité příkazy: \\
\\
\\
\textbf{4.}\\
Implicitní DNS servery:\\
\vskip 2em
\textbf{5.}\\
Úprava:\\
\\
\textbf{6.}\\
Záznam + popis:\\
\vskip 8em

\textbf{8.}\\
Použitý příkaz:\\
\\
\textbf{9.}\\
Použitý příkaz:\\
Chybová zpráva:

\section{Wireshark}
\textbf{1.}\\
Capture filter:\\
\\
\textbf{2.}\\
Komu patří nalezené IPv4 adresy a MAC adresy? Vypisovali jste již některé z nich? Proč tomu tak je?\\
\\
\begin{tabular}{|l|c|c|c|c|}
\hline
& \multicolumn{2}{|c|}{\textbf{Požadavek HTTP}} & \multicolumn{2}{|c|}{\textbf{Odpověď HTTP}}\\
\hline
\textbf{Hodnota} & \textbf{Adresa} & \textbf{Role zařízení} & \textbf{Adresa} & \textbf{Role zařízení}\\
\hline
Cílová MAC adresa & \hspace{8em} & \hspace{6em} & \hspace{8em} & \hspace{6em} \\
\hline
Cílová IPv4 adresa & & & & \\
\hline
Zdrojová MAC adresa & & & & \\
\hline
Zdrojová IPv4 adresa & & & & \\
\hline
\end{tabular}
\vspace{2em}

\textbf{3.}\\
Display Filtr:\\
\\
Zamyslete se nad tím jaké výhody a nevýhody má použití Capture a Display
Filteru:
\\
\textbf{6.}\\
Jaký je formát zobrazených dat funkcí \emph{Follow TCP stream}, slovně popište
význam funkce \emph{Follow TCP stream}\\

\section{Konfigurace IPv4 a IPv6}

Zvolená maska sítě pro IPv4:\\
Použité adresy IPv4:\\
Maximální velikost skupiny:\\
\\
\\
Zvolená adresa sítě pro IPv6:\\
Je vám známo, že prefix /48 používá někdo jiný? Jak moc jste si jistí?:\\
Použité adresy IPv6:\\
Maximální velikost skupiny:\\

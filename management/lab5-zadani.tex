%\include{macro_lecture}
\section*{Cíl laboratorního cvičení}
\begin{itemize}
  \item Seznámit se s nástroji pro správu sítě a protokolem ICMP.
  \item Seznámit se s formátem Cisco Netflow pro ukládání statistických dat a nástrojem pro čtení NetFlow {\bf nfdump}.
  \item Naučit se pracovat s protokolem Syslog a nástrojem {\bf rsyslog}.
\end{itemize}

\section*{Pokyny}
\begin{itemize}
  \item Do zadání nepište, slouží pro další skupiny. PDF verzi zadání
  i šablony konfiguračních souborů lze najít v Moodle VUT.
  
  \item Na konci laboratorního cvičení nezapomeňte na poslední bod,
  tj. na {\bf Ukončení práce v laboratoři}!
  
  \item Zapněte si počítače s operačním systémem GNU/Linux a přihlaste se jako uživatel {\bf user} s heslem {\bf user4lab}.
\end{itemize}

\newpage
%\section{Laboratorní úlohy}

%SYSLOG
\section{Syslog}
  \begin{itemize}
    \item Úkol: 
    \begin{itemize}
      \item Seznámit se s protokolem Syslog, který slouží pro přenos
      logovacích zpráv ze spravovaných zařízení. Pojmem Syslog je často označováno také
      programové vybavení implementující samotný přenos, třídění a ukládání zpráv na disk.
      \item Pracujte ve dvojicích, kde jeden bude v~roli serveru a~druhý v~roli klienta.
      \item Pro práci využijte nástroj {\tt rsyslogd}, který bude sloužit jako server i klient.
      K otestování využijte nástroj {\tt logger}.
      \item Na klientovi následně omezte přeposílání pouze na zprávy konkrétního typu.
      \item Na obou stanicích pracujte jako uživatel root.
    \end{itemize}
    \item Příkazy:
       \begin{itemize}
            \item {\tt rsyslogd(8)} -- démon pro Syslog.
            \item {\tt rsyslog.conf(5)} -- Popis konfigurace rsyslog démona.
            \item {\tt logger(1)} -- Nástroj pro generování Syslog zpráv.
        \end{itemize}
    \item Postup:
       \begin{enumerate}
            \item {\bf Na serveru} povolte naslouchání na síťovém soketu. Do souboru {\tt /etc/rsyslog.conf} přidejte nebo odkomentujte:
\begin{verbatim}
  module(load="imudp")
  input(type="imudp" port="514)
\end{verbatim}
            \item Na serveru povolte port 514 na firewallu.
            \begin{verbatim}
                firewall-cmd --add-port=514/udp
            \end{verbatim}
            \item {\bf Na klientovi} nakonfigurujte rsyslog démona tak, aby odesílal veškeré zprávy
         z klienta na serveru pomocí UDP.
         Do souboru {\tt /etc/rsyslog.conf} přidejte na konec konfiguračního souboru následující pravidlo:
\begin{verbatim} 
 *.*  @<doménové_jméno_serveru>:<číslo_portu>
\end{verbatim}
            \item Doménové jméno serveru je ve formátu: \texttt{hXX.netlab.fit.vutbr.cz}, kde \texttt{XX} je číslo PC. Tedy např. \texttt{h10.netlab.fit.vutbr.cz}
            \item {\bf Na serveru i klientovi} restartujte Syslog démona: 
\begin{verbatim}
  systemctl restart rsyslog
\end{verbatim} 
            \item Na serveru ověřte, zda-li proces naslouchá na daném portu (\verb|netstat -ulnp|) 
            \item {\bf Z klienta} ověřte správnou konfiguraci vygenerováním testovací Syslog zprávy pomocí nástroje {\tt logger}:
            
\begin{verbatim} 
  logger -d <obsah_zprávy>
\end{verbatim} 

            \item Zpráva byla přeposlána na server, kde ji lze najít na konci souboru
         {\tt /var/log/messages}.

\begin{verbatim} 
  tail -f /var/log/messages | grep <doménové_jméno_klienta>
\end{verbatim} 

            \item Na serveru/klientovi sledujte Syslog zprávy pomocí nástroje {\tt wireshark}. Zjistěte, na jakém portu a jakým protokolem jsou Syslog zprávy zasílány. Na klientovi se v novém terminálu odhlaste a přihlaste, čímž vygenerujete Syslog zprávy, které jsou zaslány na server. Výsledky zapište do protokolu.
       \end{enumerate}
   \end{itemize}

%NETFLOW
\section{NetFlow}

Seznamte se možnostmi měření provozu pomocí NetFlow. NetFlow slouží pro
přenos statistik o jednotlivých tocích dat vznikajících při komunikaci po síti. Záznamy NetFlow, s~nimiž budete během cvičení pracovat, jsou
pořízeny ze sítě VUT a~anonymizovány.

\subsection*{Postup}
	\begin{enumerate}
		\item Na Vašem počítači se v adresáři {\tt /root/isa5/} nachází
		anonymizovaný soubor s NetFlow daty (\texttt{nfcapd.202211242155.anon}). Tento soubor bude    vstupem pro program  {\tt nfdump}, který využijte ke kladení dotazů nad NetFlow daty.
		\item Prostudujte manuálovou stránku nástroje {\tt nfdump}.
  		\begin{itemize}
			\item V manuálové stránce si najděte, co dělají přepínače {\tt -r, -s, -n, -O, -A}.
                \item Prostudujte si v manuálové stránce příklady v sekci EXAMPLES 
		\end{itemize}
		\item Dotažte se na TOP 20 IP adres podle počtu přenesených bajtů. 
		\begin{verbatim}nfdump -r /root/isa5/nfcapd.202211242155.anon -s srcip/bytes -n 20
            \end{verbatim}
		
		\item Zjistěte, kolik dat je přeneseno pomocí na portech 80 (HTTP) a 443 (HTTPS). Jaký je jejich  poměr?
            \begin{verbatim}nfdump -r /root/isa5/nfcapd.202211242155.anon -s port/bytes 
            \end{verbatim}
		\begin{itemize}
			\item Všimněte si rozdílů v podílech podle toků a podle přenesených bajtů. Vypište a porovnejte statistiku portů řazenou podle počtu toků.
		\end{itemize}
	\end{enumerate}
\end{itemize}


%SNMP
\section{SNMP}
Pomocí protokolu SNMP získejte informace o zvoleném serveru. Informace lze získat pomocí klienta \texttt{snmpwalk}, viz \texttt{man snmpwalk}. Příklad syntaxe:

\begin{verbatim}
snmpwalk -v<protokol> -c <community string> <host> <object>
\end{verbatim}

\subsection*{Postup}
\begin{itemize}
    \item Vyhledejte základní informace o serveru isa2.netlab.fit.vutbr.cz.  Použijte příkaz snmpwalk a OID objekt \texttt{system}. Pokud příkaz snmpwalk není dostupný, doinstalujte balíček \texttt{net-snmp-utils}. (\texttt{yum install net-snmp-utils}). Pro připojení použijte privátní IP serveru (\texttt{10.10.10.1})
      \begin{verbatim}snmpwalk -v2c -c public isa2.netlab.fit.vutbr.cz system\end{verbatim}
    \item Vypište pomocí protokolu SNMP ARP tabulku (mapování mezi IP a MAC) na serveru isa2. Z dané tabulky nalezněte mapování mezi IP a MAC adresou u PC vašeho souseda. Pro zjištění mapování použijte OID objekt \texttt{ipNetToMedia}. Pro přehlednější výpis můžete použít příkaz \texttt{snmptable} se stejnou syntaxí jako \texttt{snmpwalk}.
    \begin{verbatim}snmptable -v2c -c public isa2.netlab.fit.vutbr.cz ipNetToMedia
    \end{verbatim}
    \item Zjistěte \textbf{název} síťového rozhraní serveru isa, na kterém je naučené mapování mezi IP a MAC adresami vašeho souseda. Nápověda: použijte OID objekt \texttt{interface} a příkaz \texttt{snmpwalk}.
\end{itemize}

\section{ICMP}
Protokol ICMP slouží pro základní dohledávání problémů v počítačových sítích. Standardně se využívají nástroje \texttt{ping} a \texttt{traceroute}, nebo pokročilejší nástroje, např. \texttt{mtr}. Pomocí těchto nástrojů zjistěte základní informace o cestě mezi vaším počítačem a vybraným serverem.

\begin{itemize}
    \item Zjistěte pomocí nástroje \texttt{ping} dobu odezvy serveru \texttt{google.com}.
    \item Jaká je maximální velikost obsahu (payload) paketu na cestě k serveru \texttt{www.fit.vutbr.cz}? Jak se tato velikost liší u IPv4 a IPv6?
    \begin{itemize}
        \item Použijte nástroj ping a argumenty \texttt{-s} pro velikost obsahu paketu a \texttt{-M do} pro zakázání fragmentace (nastavení bitu \texttt{Don't Fragment} v hlavičce IP.
        \begin{verbatim}ping -s 1500 -M do www.fit.vutbr.cz
        \end{verbatim}
        \item Pro IPv6 test se přihlašte pomocí protokolu SSH a svého studentského loginu/hesla na server merlin.fit.vutbr.cz a otestujte cestu k serveru \texttt{www.fit.vutbr.cz}.
        \item Zachyťte ICMP provoz nástrojem wireshark a zjistěte zapouzdření a velikost jednotlivých vrstev.
    \end{itemize}
    \item Pomocí nástroje \texttt{traceroute} zjistěte skrz jaké \textbf{autonomní systémy} je směrován provoz na servery \texttt{google.com} a \texttt{idnes.cz}.
    \begin{verbatim}
        traceroute -A google.com
        traceroute -A idnes.cz
    \end{verbatim}
    \item Jací operátoři provozují dané autonomní systémy? Nápověda: využijte nástroj whois, nebo stránky \texttt{stat.ripe.net} případně \texttt{bgp.he.net}.
\end{itemize}

\section*{Ukončení práce v laboratoři}
\begin{itemize}
  \item Počítač vypněte skriptem {\tt /root/isa5/clean}
\end{itemize}

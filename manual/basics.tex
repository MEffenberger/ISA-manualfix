\section{Základy konfigurace linuxového serveru} \label{basics}
Při práci s unixovými systémy doporučujeme využívat manuálové stránky pro zjištění významu daného příkazu (aplikace) a detailního popis parametrů spuštění, viz příkaz {\tt man <příkaz>}, např. {\tt man ifconfig}.
\subsection{Základní orientace v~Linuxu}
Většina práce v~OS Linux bude probíhat v terminálovém okně. Terminál na školních PC spustíte pomocí \texttt{Alt+F2}, zde zadáte "\texttt{gnome-terminal}", případně můžete vybrat aplikaci terminálu v nabídce aplikací.

\subsubsection{Unixový souborový systém}
Všechny soubory v počítači jsou uloženy v~souborovém systému. Ten je organizován jako invertovaný strom souborů a adresářů, kde kořenový adresář \texttt{root} je označen jako lomítko  \texttt{"/"}. Kořenový adresář obvykle obsahuje standardizovaný seznam podadresářů, do kterých se ukládají systémové, aplikační i uživatelské soubory. 

Některé podadresáře a jejich význam:
\begin{itemize}
  \item {\tt /etc} - konfigurační soubory operačního systému,
  \item {\tt /bin} - systémové aplikace a příkazy
  \item {\tt /sbin} - systémové aplikace a příkazy pro správce systému (uživatel {\tt root})
  \item {\tt /usr} - uživatelské aplikace a příkazy ({\tt /usr/bin, /usr/lib, /usr/local})
  \item {\tt /home} - domovské adresáře uživatelů
  \item {\tt /var} - soubory s proměnlivým obsahem, cache paměti, logy (\texttt{/var/log}),
  \item {\tt /dev} - ovladače (drivers) k periferním zařízením počítače.
\end{itemize}

\subsubsection{Základní příkazy pro práci se soubory, adresáři a procesy}
Některé základní příkazy pro práci v~terminálu OS Linux:
\begin{itemize}
  \item \texttt{cd} - změna adresáře,
  \item \texttt{ls} - zobrazení obsahu adresáře,
  \item \texttt{cp} - kopírování souborů,
  \item \texttt{mv} - přesun souborů,
  \item \texttt{mkdir} - vytvoření adresáře,
  \item \texttt{rmdir} - zrušení prázdného adresáře,
  \item \texttt{touch} - vytvoření prázdného souboru,
  \item \texttt{rm} - smazání souboru,
  \item \texttt{head} - zobrazení začátku souboru,
  \item \texttt{tail} - zobrazení konce souboru,
  \item \texttt{cat} - zobrazení obsahu souboru,
  \item \texttt{more} - výpis obsahu souboru po stránkách,
  \item \texttt{grep} - hledání textu v souboru,
  \item \texttt{chmod} - změna přístupových práv souboru,
%  \item \texttt{sed} - editace textu v~příkazové řádce,
%  \item \texttt{awk} - skenování a zpracování textu,
  \item \texttt{ps} - výpis běžících procesů.
\end{itemize}

Mezi textové editory patří například \texttt{nano}, {\tt vi} nebo \texttt{vim}. Tyto editory je vhodné používat k editaci konfiguračních souborů v operačním systému. Kromě řádkových textových editorů lze použít použít i grafické editory, například \texttt{gedit}.

\subsubsection{Role uživatele a správce v OS}
Každý uživatel operačním systému Linux má svůj jedineční identifikátor \texttt{uid} (user ID). Kromě uživatelů jsou v OS definovány i skupiny uživatelů definované identifikátor {\tt gid} (group ID). Každý soubor je vlastněn konkrétním uživatelem patřící do určité skupiny. Přístup k souboru je omezen přístupovými právy pro vlastníka souboru, skupinu či ostatní uživatele. 

Každý proces (běžící aplikace) v~systému je spuštěn pod určitým uživatelem a je jednoznačně identifikován číslem procesu {\tt pid} (process ID).

Základní role uživatelů v unixovém OS:
\begin{itemize}
  \item \texttt{root} - správce systému: má veškerá práva a kontrolu nad systémem.
  \item \texttt{user} - běžný uživatel: má pouze základní správa bez možnosti zasahovat do systémového nastavení.
%  \item \texttt{sudo} - delegace některých práv super uživatele na normálního uživatele (nastavení v~\texttt{/etc/sudoers}).
\end{itemize}

Při práci v operačním systému se snažíme vyhnout používání úrovně  \texttt{root}. Pokud potřebujeme použít pro určité činnosti příkazy vyžadující úroveň práv {\tt root}, můžeme využít příkaz \texttt{sudo}, který deleguje práva správce systému danému uživateli bez nutnosti přihlásit se jako uživatel {\tt root}. Toto oprávnění je definováno v souboru {\tt /etc/sudoers}. Příkaz s privilegovaným přístupem pak můžeme spustit příkazem {\tt sudo command}. 

V případě potřeby je možné změnit roli uživatele příkazem \texttt{su [user]}
\begin{itemize}
  \item \texttt{su} - přepnutí na uživatele \texttt{root}.
  \item \texttt{su user} - přepnutí na uživatele \texttt{user}.
\end{itemize}

\subsection{Správa systémových služeb}
\label{sluzby}
Operační systém Linux poskytuje řadu nástrojů pro správu systémových služeb. 
%různých aplikací plnících roli systémových služeb, tyto mohou poskytovat užitečné funkce jiným aplikacím, uživatelům nebo spravovat konfiguraci subsystémů.
Systémové služby obvykle běží autonomně, na pozadí systému bez přímého rozhraní pro uživatele. Operační systém proto poskytuje nástroje pro jejich manipulaci. Správu operačního systému a systémových služeb v linuxových distribucích provádí aplikace \texttt{systemd} (system demon). Pro spouštění a manipulaci systémových služeb se využívá nástroj {\tt systemctl} (system control).

Příklady použití: 
\begin{itemize}
  \item \texttt{systemctl start [služba]} -- spustí službu
  \item \texttt{systemctl stop [služba]} -- zastaví službu
  \item \texttt{systemctl restart [služba]} -- restartuje službu
  \item \texttt{systemctl enable [služba]} -- povolí automatické spuštění služby po startu systému
  \item \texttt{systemctl disable [služba]} -- zakáže automatické spuštění služby po startu systému
  \item \texttt{systemctl status [služba]} -- vypíše informace aktuálního stavu služby a několik posledních řádků systémových logů této služby. Pokud spustíme příkaz bez parametrů, vypiš  informace
        o~všech aktuálně spuštěných službách.
\end{itemize}

Jméno služby má typicky tvar \texttt{<aplikace>.service}, například
\texttt{sshd.service}. Pro zobrazení logů konkrétní systémové služby lze použít příkaz \texttt{journalctl -u [služba]}. Detailní popis příkazů a jejich parametrů naleznete v~manuálových stránkách.

\subsection{Konfigurace síťových zařízení} \label{basic_ipconfig}
V této kapitole si ukážeme několik nástrojů operačního systému Linux, které slouží pro zjišťování síťové konfigurace.

\subsubsection{Zobrazení konfigurace}
Základním příkazem je příkaz \texttt{ip}. Pomocí tohoto příkazu můžeme zjistit konfiguraci všech síťových rozhraní počítače (fyzických i virtuálních), obsah směrovací tabulky a jiné. Manuálové stránky pro dané možnosti zobrazíte jako \texttt{man ip <volba>}, např. {\tt man ip link}.

\begin{itemize}
  \item \texttt{ip address} -- zobrazí adresu na všech síťových rozhraní (též příkaz {\tt ifconfig}). 
  \item \texttt{ip route} -- zobrazí pravidla ve směrovací tabulce
  \item \texttt{ip link} -- zobrazí konfiguraci síťových rozhraní
  \item \texttt{ip neighbour} -- zobrazí obsah ARP záznamů (též příkaz \texttt{arp}).
\end{itemize}

Příkaz {\tt ip} neslouží jen k výpisu konfiguraci, ale je možné jím i nastavit parametry síťového rozhraní.

Další příkazy pro zobrazení síťové konfigurace:
\begin{itemize}
  \item \texttt{netstat -rn} -- zobrazí obsah směrovací tabulky
  \item \texttt{ss} (\texttt{sockstat}) -- zobrazí čekajících spojení či navázaných síťových spojení včetně zobrazení typu transportního protokolu (parametry {\tt -t} pro TCP, {\tt -u} pro UDP), zdrojových a cílových adres a portů.
  \item {\tt ss state ESTABLISHED -t} -- zobrazí navázaná spojení nad TCP (ve stavu ESTABLISHED)
\end{itemize}

\subsubsection{Testování konektivity a síťových služeb}
Následující příkazy slouží k testování dostupnosti zařízení a konektivity. 
\begin{itemize}
  \item \texttt{ping <ipv4> | <hostname>} -- zašle příkaz ICMP ECHO\_REQUEST na dané zařízení.
  \item \texttt{ping6 <ipv6> | <hostname>} -- zašle příkaz ICMPv6 ECHO\_REQUEST na dané zařízení.
  \item \texttt{traceroute <ipv4> | <hostname>} -- otestuje a zobrazí cestu IP datagramu  k cíli.
  \item \texttt{traceroute6 <ipv6> | <hostname>} -- otestuje a zobrazí cestu IPv6 datagramu k cíli.
  \item \texttt{tcptraceroute <ipv4> | <ipv6> | <hostname>} - otestuje a zobrazí cestu paketu TCP k cíli.
\end{itemize}

\noindent
Pro přihlášení na vzdálený host můžeme použít dva nástroje:
\begin{itemize}
  \item \texttt{telnet <host>} -- vzdálený přístup pomocí nešifrovaného spojení
  \item \texttt{ssh <host>} -- vzdálený přístup pomocí šifrovaného přípojení
\end{itemize}

\noindent
Testování dostupných zařízení a služeb
\begin{itemize}
  \item \texttt{nmap} - skenování aktivních stanic v síti, otevřených portů apod. 
  \item \texttt{telnet <host> <port>} -- testování dostupnosti služeb nad TCP
\end{itemize}

\subsection{Konfigurační soubory}
Konfigurační soubory operačního systému a služeb se většinou ukládají do adresáře \texttt{/etc}. Jednotlivé služby zde mají svůj konfigurační soubor s názvem dané služby. Konfigurace některých služeb je uložena v adresářích  \texttt{"/etc/<sluzba>.d/"} (například \texttt{/etc/rsyslog.d/}). 

Důležité  konfigurační soubory:
\begin{itemize}
  \item \texttt{/etc/hostname} -- obsahuje doménové jméno linuxového systému.
  \item \texttt{/etc/hosts} -- obsahuje statické mapování doménového jména na IP adresu (nevyžaduje DNS)
  \item \texttt{/etc/host.conf} -- obsahuje konfiguraci specifickou pro DNS resolver.
  \item \texttt{/etc/resolv.conf} -- konfigurační soubor pro DNS resolver, který obsahuje zejména odkazy na lokální DNS servery (získané z DHCP)
  \item \texttt{/etc/rsyslog.conf} - konfigurační soubor aplikace {\tt rsyslog} 
\end{itemize}
Podrobnosti o konfiguraci výše uvedených služeb lze získat z manuálových stránek, např. {\tt man resolv.conf}. 

\subsection{Logování událostí}
Systémové i aplikační události jsou zaznamenávány do logovacích souborů. Logovací soubory jsou obvykle uloženy v~adresáři \texttt{/var/log}. Tyto soubory jsou cyklicky rotovány po určitém čase či po dosažení určité velikosti. Dobu, po kterou jsou logy udržovány, lze nastavit v~konfiguračním souboru\\\texttt{/etc/logrotate.conf}.

Důležité logovací soubory:
\begin{itemize}
  \item \texttt{/var/log/messages} -- obsahuje globální systémové zprávy a události.
  \item \texttt{/var/log/auth.log} -- obsahuje zprávy týkající se autentizace uživatelů.
  \item {\tt ./var/log/maillog} -- obsahuje logovací zprávy od mailového serveru. 
%  \item \texttt{/var/log/kern.log} -- ukládá zprávy týkající se jádra OS.
\end{itemize}

Tyto logovací soubory můžeme prohlížet přímo pomocí standardních zobrazovacích příkazů ({\tt cat, more, tail -f, grep}). Zobrazení logů většinou vyžaduje přístupová práva správce systému. 

Systémové události můžeme také prohlížet pomocí nástroje \texttt{journalctl}, který umožňuje podrobněji procházet seznam zpráv, filtrovat a vyhledávat.

Příklad použití \texttt{journalctl}:
\begin{itemize}
  \item \texttt{journalctl -n <x>} - zobrazí posledních x záznamů.
  \item \texttt{journalctl -p <priority>} - zobrazí pouze záznamy s~danou prioritou Syslog.
  \item \texttt{journalctl -u <unit> | <pattern>} - zobrazí pouze záznamy s danou službou nebo obsahující zadaný řetězec
  \item \texttt{journalctl -t <syslog\_id> | <pattern>} -- vyhledává záznamy obsahující zadané ID syslogu nebo záznamy obsahující v ID syslogu zadaný řetězec.
  \item \texttt{journalctl -f} - průběžně tiskne na terminál poslední přidávané události
  \item \texttt{journalctl --since <whenstart> --until <whenend>} - zobrazí záznamy v~daném časovém rozmezí.
\end{itemize}







%%%%%%%%%%%%%%%%%%%%%%%%%%%%%%%%%%%%%%
\section{Základy konfigurace linuxového serveru}
\label{basics}

Detailní popis včetně možných voleb a příkladu použití k~jednotlivým níže zmíněným příkazům můžete nalézt v~manuálových stránkách (\texttt{man <příkaz>}).
\subsection{Základní orientace v~Linuxu}
Většina práce v~OS Linux bude probíhat v~terminálu. Terminál na školních PC spustíte pomocí \texttt{Alt+F2}, zde zadáte "\texttt{gnome-terminal}". Případně můžete vybrat aplikaci terminálu z~postraní nabídky.
\subsubsection{Hierarchie souborového systému}
Všechny soubory jsou uloženy v~souborovém systému. Ten je organizován jako invertovaný strom adresářů,
kde kořenovým adresářem je \texttt{root} adresář (označení \texttt{"/"}). Ten obsahuje podadresáře,
kde každý má svůj standardizovaný účel pro zařazení různých souborů.

Některé podadresáře a jejich význam:
\begin{itemize}
				\item \textbf{/etc} - obsahuje konfigurační soubory systému,
				\item \textbf{/var} - proměnlivá boot perzistentní data, dynamicky se mění, obsahuje databáze, cache adresáře, obsah www stránek (\texttt{/var/html/www}), logy (\texttt{/var/log}),
				\item \textbf{/dev} - obsahuje speciální soubory pro přístup k~hardware zařízením.
\end{itemize}

\subsubsection{Základní příkazy}
Některé základní příkazy pro práci v~terminálu OS Linux:
\begin{itemize}
				\item \texttt{cd} - posun v~adresářové hierarchii,
				\item \texttt{ls} - zobrazení obsahu adresáře,
				\item \texttt{cp} - kopírování souborů,
				\item \texttt{mv} - přesun souborů,
				\item \texttt{mkdir} - vytvoření adresáře,
				\item \texttt{touch} - vytvoření prázdného souboru,
				\item \texttt{head} - zobrazení x řádků od začátku souboru,
				\item \texttt{tail} - zobrazení x řádků z~konce souboru,
				\item \texttt{cat} - zobrazení obsahu souboru,
				\item \texttt{grep} - filtrování/hledání v~textu,
				\item \texttt{sed} - editace textu v~příkazové řádce,
				\item \texttt{awk} - skenování a zpracování textu,
				\item \texttt{ps} - zobrazení informací o~běžících procesech.
\end{itemize}

Mezi command line editory patří například \texttt{nano} nebo \texttt{vim}. Kromě nich můžete taktéž použít grafické editory, například \texttt{gedit}.

\subsubsection{Uživatelé}
Každý uživatel v~OS Linux má své jedinečné \texttt{uid}. Každý proces (program)
v~systému je spuštěn pod nějakým uživatelem. Každý soubor je vlastněn uživatelem
a omezen přístupovými právy. Tato přístupová práva se vztahují taktéž na procesy
spuštěné daným uživatelem.

\begin{itemize}
				\item \texttt{root} - super uživatel, má veškerá práva a kontrolu nad systémem.
				\item \texttt{user} - pouze základní správa, bez možnosti zasahovat do systémového nastavení.
				\item \texttt{sudo} - delegace některých práv super uživatele na normálního uživatele (nastavení v~\texttt{/etc/sudoers}).
\end{itemize}

Při správě OS se pokud možno snažíme vyhnout používání systému jako \texttt{root}. K~tomuto slouží \texttt{sudo},
kde má daný uživatel přesně specifikováno, které operace smí se systémem provádět, a je pak lépe dohledatelné,
kdo co nastavil.

Přepínání uživatele:
\texttt{su [user]}
\begin{itemize}
				\item \texttt{su} - přepnutí na uživatele \texttt{root}.
				\item \texttt{su user} - přepnutí na uživatele \texttt{user}.
\end{itemize}

Spouštění příkazů jako \texttt{sudo}:
\texttt{sudo command...}

\subsection{Správa systémových služeb}
\label{sluzby}

OS Linux poskytuje řadu různých aplikací plnících roli systémových služeb, tyto
mohou poskytovat užitečné funkce jiným aplikacím, uživatelům nebo spravovat
konfiguraci subsystémů. Systémové služby typicky běží autonomně, na pozadí
systému bez přímého rozhraní pro uživatele. Naopak systém poskytuje speciální
rozhraní pro jejich manipulaci. Pro Linuxové distribuce se \texttt{systemd} máme
dostupnou aplikaci \texttt{systemctl}, která poskytuje několik základních
příkazů pro manipulaci systémových služeb:
\begin{itemize}
    \item \texttt{systemctl start [služba]} spustí službu
    \item \texttt{systemctl stop [služba]} zastaví službu
    \item \texttt{systemctl restart [služba]} restartuje službu
    \item \texttt{systemctl enable [služba]} povolí automatické spuštění služby
        po startu systému
    \item \texttt{systemctl disable [služba]} zakáže automatické spuštění
        služby po startu systému
    \item \texttt{systemctl status [služba]} vypíše informace aktuálního stavu
        služby a několik posledních řádků systémových logů této služby. Tento
        příkaz můžete použít i bez specifikace služby, dostanete tak informace
        o~všech aktuálně spuštěných službách.
\end{itemize}

Jméno služby má typicky tvar \texttt{<aplikace>.service}, například
\texttt{sshd.service}.

Pro zobrazení kompletních logů konkrétní systémové služby můžete použít příkaz
\texttt{journalctl -u [služba]}.

Detailní popis včetně možných voleb a příkladů použití můžete nalézt
v~manuálových stránkách.



\subsection{Konfigurace síťových zařízení}
\label{basic_ipconfig}
Ke zjišťování síťové konfigurace na daném zařízení můžeme použít několik nástrojů,
které jsou na OS Linux k~dispozici.

\subsubsection{Zobrazení konfigurace}
Jedním ze základních příkazů je \texttt{ip}. Pomocí tohoto příkazu můžeme zjistit
konfiguraci všech síťových zařízení (fyzických i virtuálních), obsah směrovací tabulky aj. Manuálové stránky pro dané možnosti zobrazíte jako \texttt{ip-<volba>}.

\begin{itemize}
  \item \texttt{ip address} - zobrazí adresu na všech síťových zařízení (příkaz
    \texttt{ifconfig} je dnes již zavržený\footnote{\url{https://www.root.cz/clanky/prikaz-ip-ovladnete-linuxova-sitova-rozhrani/}}).
				\item \texttt{ip route} - zobrazí všechna pravidla v~směrovací tabulce.
				\item \texttt{ip link} - vylistuje všechna síťová zařízení.
        \item \texttt{ip neighbour} - zobrazí obsah ARP záznamů (případně můžete
          využít starší příkaz \texttt{arp}).
\end{itemize}

Tento nástroj mimo jiné slouží také ke konfiguraci, ne jen k~jejímu zobrazení. \footnote{Obdobou tohoto nástroje je \texttt{ifconfig}.}

%- nmcli

Alternativou k~zobrazení obsahu směrovací a ARP tabulky jsou tyto příkazy:
\begin{itemize}
\item \texttt{netstat} - zobrazí mimo jiné obsah směrovací tabulky (volba \texttt{-rn})
\end{itemize}

Příkaz \texttt{ss} (\texttt{sockstat}) slouží
k~zobrazení aktuálně běžících spojení včetně protokolu (parametry {\tt -t}, {\tt
-u}), portu, zdrojové a cílové
adresy. Aktuálně existující spojení můžete filtrovat pomocí stavového filtru.
Např. {\tt ss state ESTABLISHED -t}.

\subsubsection{Konfigurační soubory a logy}
Na OS Linux je veškeré nastavení systému a služeb uloženo v~adresáři \texttt{/etc}. Jednotlivé služby zde mají svůj konfigurační soubor s~příslušným názvem. Pro lepší přehlednost je možné vytvářet více konfiguračních souborů, pro tyto účely zde pak existují adresáře \texttt{"nazevsluzby.d/"} (například \texttt{/etc/rsyslog.d/}). Veškeré nastavení v~souborech v~těchto adresářích je pak aplikováno na danou službu.

Některé ze základních podstatných konfiguračních souborů:
\begin{itemize}
				\item \texttt{/etc/hostname} - obsahuje hostname systému.
				\item \texttt{/etc/hosts} - obsahuje mapování hostname na IP adresu (zde můžete libovolné adrese přiřadit hostname, následné použití tohoto hostname předchází samotnému vyhledání pomocí DNS resolveru).
				\item \texttt{/etc/host.conf} - obsahuje konfiguraci specifickou pro resolver.
				\item \texttt{/etc/resolv.conf} - obsahuje direktivy specifikující výchozí servery, na které je poslán dotaz pro překlad doménového jména na IP adresu.
				\item \texttt{/etc/rsyslog.conf} - obsahuje nastavení syslogu, rozřazování jednotlivých zpráv do příslušných logovacích souborů.
\end{itemize}


Veškeré podstatné události jsou zaznamenávány do logů. Všechny logovací soubory jsou uloženy v~adresáři \texttt{/var/log}. Tyto soubory (logy) jsou cyklicky promazávány po určitém čase či po dosažení určité velikosti. Dobu, po kterou jsou logy udržovány, lze nastavit v~konfiguračním souboru\\\texttt{/etc/logrotate.conf}.

Některé významné logovací soubory:
\begin{itemize}
				\item \texttt{/var/log/messages} - obsahuje obecné zprávy systému (bez kritických a debugovacích).
				\item \texttt{/var/log/auth.log} - zde jsou uloženy zprávy týkající se autentizace uživatelů.
				\item \texttt{/var/log/kern.log} - ukládá zprávy týkající se jádra OS.
\end{itemize}

Tyto logovací soubory můžeme prohlížet přímo pomocí standardních zobrazovacích příkazů.

Systémové události můžeme také prohlížet pomocí nástroje \texttt{journalctl}, ten nám umožní procházet podrobnější informace daných zpráv, filtrovat a vyhledávat.

Příklad použití \texttt{journalctl}:
\begin{itemize}
				\item \texttt{journalctl -n <x>} - zobrazí posledních x záznamů.
				\item \texttt{journalctl -p <priority>} - zobrazí pouze záznamy s~danou syslog prioritou.
				\item \texttt{journalctl -u <unit|pattern>} - umožní vyfiltrovat pouze
          záznamy s~předanou službou (spravovanou systemd), nebo záznamy
          odpovídající předanému vzoru.
        \item \texttt{journalctl -t <syslog\_id|pattern>} vyhledává v syslogovém záznamu
          podle názvu služby (tzn. i služby, které nejsou spravovány systemd,
          např. \emph{sudo}).
				\item \texttt{journalctl -f} - zobrazí kontinuálně posledních 10 událostí.
				\item \texttt{journalctl --since <whenstart> --until <whenend>} - zobrazí záznamy v~daném rozmezí.
\end{itemize}


\subsubsection{Aktivní zjišťování}
Zjišťování dostupnosti zařízení a konektivity. K~tomuto účelu poslouží příkazy:

\begin{itemize}
				\item \texttt{ping ipv4|domain} - zašle ICMP ECHO\_REQUEST k~danému hostu (pro ipv4).
				\item \texttt{ping6 ipv6|domain} - zašle ICMP ECHO\_REQUEST k~danému hostu (pro ipv6).
				\item \texttt{traceroute ipv4|domain} - zobrazí cestu, kudy packet prochází skrz síť k~danému hostu (pro ipv4).
				\item \texttt{traceroute6 ipv6|domain} - zobrazí cestu, kudy packet prochází skrz síť k~danému hostu (pro ipv6).
				\item \texttt{tcptraceroute ipv4/ipv6|domain} - \texttt{traceroute} používající TCP.
\end{itemize}

Mimo základní výše uvedené příkazy \texttt{ping} a \texttt{traceroute}
existuje užitečný nástroj \texttt{nmap}.

\begin{itemize}
				\item \texttt{nmap} - umožňuje pomocí různých protokolů aktivně zjišťovat otevřené porty, aktivní hosty na dané síti apod.
\end{itemize}


Pro přihlášení na vzdálený host můžeme použít dva nástroje:
\begin{itemize}
				\item \texttt{telnet ip} - nešifrované přípojení
				\item \texttt{ssh ip} - šifrované přípojení
\end{itemize}




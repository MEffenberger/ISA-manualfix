%%%%%%%%%%%%%%%%%%%%%%%%%%%%%%%%%%%%%%
\section{Signalizační protokol SIP}\label{sip}
Session Initiation Protocol (SIP, RFC 3261 \cite{rfc3261}) je signalizační protokol pro IP telefonie, který provádí následující operace:
\begin{itemize}
    \item registrace uživatele VoIP na serveru SIP
    \item navázání hovoru, výměna parametrů spojení
    \item směrování hovorů
    \item ukončení či zrušení hovoru
\end{itemize}

Protokol SIP neprovádí správu relací po jejich navázání, nezajišťuje kvalitu přenosu, neslouží k přenosu hlasových dat.

Základními prvky IP telefonie jsou {\em server UAS} (User Agent Server), který provádí registraci klientů, navazování a směrování hovorů, lokalizaci klienta účtování apod., a {\em uživatelský agent UAC} (User Agent Client), který zprostředkovává uživateli volání (vytáčení spojení, příjem telefonátu). 

Služba SIP používá pro adresování speciální URI (Uniform Resource Identifier) ve tvaru {\tt sip:user @domain}, kde {\tt user} je uživatelské jméno, které přidělí poskytovatel uživateli, a {\tt domain} je VoIP doména poskytovatele. Adresa uživatele je součástí metody {\tt INVITE}, která slouží k vytvoření hovoru, nebo se využívá při registraci telefonu (metoda {\tt REGISTER}). 

Směrovací informace se během navazování spojení ukládají v hlavičce SIP paketu {\tt Via:}, {\tt Route} či {\tt Record-Route}. 

\subsection{Příklad signalizace SIP}
{\footnotesize
\begin{verbatim}
INVITE sip:541141118@cesnet.cz SIP/2.0    # navázání spojení s uživatelem 541141118 v doméně cesnet.cz
Call-ID: D40CA785-2EEE-4801-9B04-349632F56CDC@147.229.14.146   # jednoznačný identifikátor hovoru
CSeq: 2 INVITE                                                 # sekvenční číslo zprávy, název metody
From: "Petr Matousek"<sip:matousp@cesnet.cz>;tag=2007034328740 # identifikace volajícího (SIP URI)
To: <sip:541151118@cesnet.cz>                                  # identifikace volaného (SIP URI)
Contact: <sip:matousp@147.229.14.146:59183;ob>                 # identifikace volaného (Device URI)

SIP/2.0 100 Trying -- your call is important to us             # odpověď na metodu INVITE
Call-ID: D40CA785-2EEE-4801-9B04-349632F56CDC@147.229.14.146   # jednoznačný identifikátor hovoru
CSeq: 2 INVITE                                                 # odpověď na metodu INVITE   
From: "Petr Matousek"<sip:matousp@cesnet.cz>;tag=2007034328740 # zopakování hlaviček z metody INVITE
To: <sip:541151118@cesnet.cz>
\end{verbatim}
}

\subsection{Základní metody protokolu SIP}
\begin{itemize}
    \item REGISTER -- žádost o registraci
    \item INVITE, ACK -- vytváření spojení
    \item CANCEL -- zrušení vytvářeného spojení
    \item BYE -- ukončení spojení
    \item OPTIONS -- zjišťování možností přenosu
    \item PRACK -- provizorní ACK
\end{itemize}

\subsection{Další metody protokolu SIP}
\begin{itemize}
    \item INFO (RFC 6086) – přenos aplikačních informací
    \item MESSAGE (RFC 3428) – textové zprávy IM
    \item SUBSCRIBE, NOTIFY (RFC 3265) – zasílání upozornění na události
    \item PUBLISH (RFC 3903) – zveřejnění stavu událostí (prezence)
\end{itemize}

\subsection{Položky hlavičky SIP}
\begin{itemize}
    \item {\bf Via:} obsahuje směrovací informace. Každý server, přes který zpráva {\tt INVITE} prochází,
      přidá do hlavičky jeden řádek {\tt Via:} s IP adresou a portem, na kterém zpracoval paket. Zaznamenaná posloupnost uzlů, přes které zpráva {\tt INVITE} procházela, se pak využije pro směrování odpovědi. 
    \begin{verbatim}
    via: SIP/2.0/UDP 10.11.12.51:5062;rport;branch=z9hG4bk2079031997
    \end{verbatim}
    \item {\bf From:} obsahuje odesílatele SIP zprávy ve formátu SIP URI. Nepovinně může obsahovat i zobrazitelné jméno (display-name), které uživatel vložil do konfigurace klienta. 
    \begin{verbatim}
    From: "user08" <sip:user08@iss.cz>;tag=283687066
    \end{verbatim}
  \item {\bf To:} obsahuje adresu příjemce zprávy. 
    \begin{verbatim}
    To: <sip:1006@iss.cz>
    \end{verbatim}
    \item {\bf Allow} obsahuje seznam metod SIP, které podporuje daný klient.
    \begin{verbatim}
    Allow: INVITE, INFO, PRACK, ACK, BYE, CANCEL, OPTIONS, NOTIFY, REGISTER
    \end{verbatim}
\end{itemize}

\subsection{Protokol SDP (Session Description Protocol)}
Protokol SDP \cite{rfc4566} se používá pro zaslání informací potřebných k vytvoření kanálu pro přenos hlasu (případně obrazu). Jde opět o signalizační protokol, který domlouvá parametry spojení, jako je typ média (video, audio, atd.), transportní protokol (RTP/UDP/IP, H.320, atd.), typ kodeku, IP adresa a port pro příjem hlasu (obrazu).
Zprávy SDP se přenáší ve zprávách typu {\tt INVITE} a {\tt OK}.
{\footnotesize
\begin{verbatim}
v=0                                             # verze protokolu SDP
o=- 3434794301 3434794301 IN IP4 147.229.14.146 # původce spojení (username, IP adresa)
s=Sjphone                                       # jméno spojení: jméno softwaru
c=IN IP4 147.229.14.146                         # spojení: IP adresa pro příjem dat
t=0 0 # time (start + end)
a=direction:passive                               
m=audio 49152 RTP/AVP 3 97 98 8 0_101           # audio stream: číslo portu, nabízené kodeky
a=rtpmap:3 GSM/8000
a=rtpmap:97 iLBC/8000
a=rtpmap:98 iLBC/8000
a=fmtp:98 mode=20
a=rtpmap:8 PCMA/8000
a=rtpmap:0 PCMU/8000
a=rtpmap:101 telephone-event/8000
a=fmtp:101 0-11,16
\end{verbatim} 
}

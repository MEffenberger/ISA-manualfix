%%%%%%%%%%%%%%%%%%%%%%%%%%%%%%%%%%%%%%
\section{Syslog}
\label{syslog}

Protokol Syslog \cite{rfc5424} popisuje způsoby a formát pro záznam systémových událostí. Je využíván
 především v~unixových systémech pro záznam událostí jednak prvků součástí systém, jednak pro
 události běžících programů. 

Součástí záznamů býva zpravidla časová značka, zařízení (facility), závažnost záznamu (severity)
 a samotný záznam. V~některých implementacích se může objevit i doménové jméno či IP adresa. Záznam
 může být uložen lokálně (např. soubor \verb|/var/log/messages|) nebo lze záznamy přeposílat
 na vzdálený server. Takové záznamy se pak objeví v syslogu daného serveru. Součástí záznamů může být
 i doménové jméno resp. IP adresa původce záznamu.

V~položce facility je uvedena součást systému (jádro, autentizační zprávy,apod.) nebo
 přímo název programu, případně jeho PID. Kompletní seznam zařízení lze najít ve starším RFC 3164
 \cite{rfc3164}.

Závažnost záznamu je rozdělena na 8 úrovní od nejzávažnější (0) po ladící informace (7). Obvykle
 jsou v syslogu zaznamenávány událostí od úrovně varování (4). Úplný seznam úrovní pak vypadá takto:
\begin{table}[!h]
	\centering
	\begin{tabular}[!h]{|c|l|l|}
	\hline
	\shortstack{Závažnost} & & Popis \\
	\hline
	0 & EMERGENCY & Nejvyšší závažnost, systém je nestabilní\\
	1 & ALERT & Událost vyžadující okamžitou akci.\\
	2 & CRITICAL & Kritická chyba systému (např. HW chyby) \\
	3 & ERROR & Chyba programu.\\
	4 & WARNING & Varování programu.\\
	5 & NOTICE & Důležité informační záznamy (např. změna nastavení)\\
	6 & INFORMATIONAL & Informační zprávy nevyžadující speciální pozornost\\
	7 & DEBUG & Ladicí informace\\
	\hline
	\end{tabular}
\end{table}


%%%%%%%%%%%%%%%%%%%%%%%%%%%%%%%%%%%%%%
\newpage
\section{Simple Network Monitoring Protocol}
\label{snmp}

SNMP \cite{rfc1157} je protokol určený k monitorování stavu zařízení. Protokol podporuje nejen čtení
 provozních paramterů připojených stanic (využití paměti, zatížení CPU apod.), ale i nastevní
 některých paramterů těchto stanic.
Protokol je provozován nad UDP na portech 161 a 162 pro asynchronní zprávy (traps). V případě
 potřeby přenášení větších bloků informací lze SNMP provozovat i nad TCP \cite{rfc3430}.

Stanice zapojené v síti dělíme na "správce" (Network Management station, NMS0 a~sledovaná zařízení.
 Správcovské stanice slouží ke shromažďování informací ze sledovaných zařízení a jejich případné prezentaci.
Požadované informace mohou získat buď synchronní komunikací způsobem požadavek-odpověď, nebo mohou
 získávat informace od sledovaných stanic asynchronně pomocí tzv. traps. Tyto zprávy jsou odesílány
 přímo sledovanými zařízeními. Sbírané informace jsou pak uloženy v databázi MIB
 (Management Information Base). Každa položka je identifikována pomocí OID (Object Identifier),
 podobně jako v~LDAP. Příkladem může být položka sysLocation uložená pod OID {\tt 1.3.6.1.2.1.1.6.0}.
 Seznam OID podle standrardu RFC 1213 \cite{rfc1213} lze najít na stránce
\begin{verbatim}
    http://www.oidview/mibs/0/RFC1213-MIB.html.
\end{verbatim}

Sledované stanice provozují v~síti SNMP programy pro získávání potřebných informací tzv. SNMP agenty.
 Ti sbírají data a zpracovávají požadavky z~NMS.

Zařízení v síti SNMP jsou rozdělena do logických celků, komunit, identifikovaných textovým
 řetězcem. Komunita slouží k autentizaci zařízení v síti a definuje přístup k informacím MIB.
 Určuje tedy, které hodnoty a vlastnosti je povoleno v dané komunitě sledovat, případně nastavovat.
 Příklad takového nastavení může být následující: 
\begin{verbatim}
rocommunity public 10.0.0.0/24
rocommunity public 192.168.0.0/24 1.3.6.1.2.1.1
rwcommunity public localhost
\end{verbatim}
Konfigurace v příkladu umožňuje zařízením v síti 10.0.0.0/24 přístup k MIB komunity public pouze
 pro čtení. Při pokusu o zápis do jakékoli proměnné skončí akce neúspěchem, že proměnná neexistuje.
 Pro stanice v síti 192.168.0.0/24 je přístup pro čtení omezen pouze na základní szstémové informace
 (podstrom system). Při lokálním přístupu však budou záznamy přístupné i pro zápis.

 %%%%%%%%%%%%%%%%%%%%%%%%%%%%%%%%%%%%%%
 \newpage
\section{Cisco NetFlow}
\label{netflow}

Formát záznamu NetFlow umožňuje ukládání informací o IP provozu na síťovém zařízení. Formát byl
 vytvořen firmou Cisco jako funkcionalita jejich zařízení. Ve verzi 9 pak byl standartizován
 jako RFC 3954 \cite{rfc3954}.

NetFlow záznamy jsou využívány k~monitorování a analýze provozu v~síti. Data obsahují záznamy
 o IP adresách, portech, počty přenesených paketů či bajtů apod. Data jsou uchovávána pro
 tzv. toky neboli flow. Toky jsou identifikovány pěticí zdrojová a cílová IP adresa,
 zdrojový a cílový port a protokol transportní vrstvy. Tok je chápán jako jednosměrná komunikace.

Infrastruktura pro sběr a uchovánvání NetFlow dat sestává z~kolektoru a sond. Kolektorem je míněna
 stanice, která sbírá data ze sítě a provádí jejich uchování. Sondou pak může být jakékoli zařízení
 ať už směrovač, přepínač nebo zařízení přímo vyhrazené pro tuto činnost. Sběr dat může probíhat
 pro všechny toky, nebo může být prováděno vzorkování či agregace. 

%%%%%%%%%%%%%%%%%%%%%%%%%%%%%%%%%%%%%%

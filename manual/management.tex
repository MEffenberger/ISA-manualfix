%%%%%%%%%%%%%%%%%%%%%%%%%%%%%%%%%%%%%%
\section{Přenos systémových události službou Syslog}\label{syslog}
Služba Syslog  slouží pro generování a přenos systémových událostí síťových zařízení a koncových stanic na centrální úložiště (server). Systémové událostí jsou generovány operačním systémem či běžícími aplikace. Tyto události se lokálně ukládají do logovacích souborů. Služba Syslog posílá vygenerované události na server. 

Architekturu služby Syslog tvoří protokol Syslog \cite{rfc5424}, který popisuje formát přenosu zpráv, dále Syslog klient ({\tt originator}), které vytvořené události posílá na server, a Syslog server ({\tt collector}), který přijímá poslané události protokolem Syslog od klientů a centrálně je zpracovává. 

Každá vygenerovaná událost se ukládá jako záznam do logovacího souboru buď operačního systému (např. \verb|/var/log/messages|) nebo do logovacího souboru konkrétní aplikace (např. \verb|/var/log/maillog| pro poštovní server, \verb|/var/log/ipfw| pro firewall, \verb|/var/log/apache/access.log| pro HTTP server).  

Logovací záznam události zpravidla obsahuje časovou značku, typ zařízení, které událost vygenerovalo (facility), závažnost události (severity) a slovní popis události. V~některých logovacích souborech se může objevit i doménové jméno či IP adresa stanice, která událost způsobila. Záznamy mohou být uloženy lokálně v souborovém systému (na unixových systémech v adresáři \verb|/var/log|) nebo je můžeme přeposílat protokolem Syslog na vzdálený server. 

Typ zařízení, které událost vygenerovalo, je definováno identifikátorem v záznamu události. Seznam zařízení, které definuje protokol Syslog, je v tabulce \ref{tab:facility}. 
\begin{table}[h]
  \centering
  \small
  \begin{tabular}{|c|l|c|l|}
    \hline
    \bf Code & \bf Facility & \bf Code & \bf Facility\\
    \hline
     0  &  kernel messages                             &  12 &  NTP subsystem           \\
     1  &  user-level messages                         &  13 &   log audit              \\
     2  &  mail system                                 &  14 &   log alert              \\
     3  &  system daemons                              &  15 &   clock daemon (note 2)  \\
     4  &  security/authorization messages             &  16 &   local use 0  (local0)  \\
     5  &  messages generated internally by syslogd    &  17 &   local use 1  (local1)  \\
     6  &  line printer subsystem                      &  18 &   local use 2  (local2)  \\
     7  &  network news subsystem                      &  19 &   local use 3  (local3)  \\
     8  &  UUCP subsystem                              &  20 &   local use 4  (local4)  \\
     9  &  clock daemon                                &  21 &   local use 5  (local5)  \\
     10 &  security/authorization messages             &  22 &   local use 6  (local6)  \\
     11 &  FTP daemon                                  &  23 &   local use 7  (local7)  \\
     \hline
  \end{tabular}
  \caption{Typy zařízení Syslog dle RFC 5424 \cite{rfc5424}}\label{tab:facility}
\end{table}
Protože tento seznam z původního  návrhu protokolu BSD Syslog z roku 2001, obsahuje zařízení, která se dnes už příliš nepoužívají. Z tohoto důvodu se pro aplikace a další typy zdrojů událostí používá lokální označení z identifikátory 16-23. 

Pro hodnocení závažnosti vzniklé události definuje Syslog osm úrovní od nejzávažnější typu Emergency (0) po ladící výpisy typu Debug (7). Pro správu sítí jsou důležité zejména události od úrovně Warning (4). Přehled definovaných úrovní závažnosti je v tabulce \ref{tab:severity}. 
\begin{table}[h]
  \centering
  \small
  \begin{tabular}{|c|l|l|}
    \hline
    \bf Code & \bf Severity & \bf Description \\
    \hline
    0 & EMERGENCY & system is unusable  \\
    1 & ALERT & action must be taken immediately\\
    2 & CRITICAL & critical conditions \\
    3 & ERROR & error conditions\\
    4 & WARNING & warning conditions\\
    5 & NOTICE & normal but significant condition\\
    6 & INFORMATIONAL & informational messages\\
    7 & DEBUG & debug-level messages\\
    \hline
  \end{tabular}
  \caption{Úrovně závažznosti událostí Syslog dle RFC 5424 \cite{rfc5424}}\label{tab:severity}
\end{table}


%%%%%%%%%%%%%%%%%%%%%%%%%%%%%%%%%%%%%%
\clearpage
\section{Monitorování sítě SNMP}\label{snmp}

Služba SNMP (Simple Network Monitoring Protocol) slouží k aktivnímu monitorování zařízení na síti. Protokol SNMP \cite{rfc3412} slouží k přenosu monitorovacích informací mezi řídící stanicí NMS (Network Management Station) a monitorovaným zařízením (SNMP Agent). Protokol slouží nejen ke čtení provozních parametrů (hodnot objektů MIB) připojených stanic (využití paměti, počty přenesených paketů, stav síťových rozhraní, zatížení CPU apod.), ale dá se použít i k  nastavení některých těchto parametrů. Protokol pracuje nad UDP na portech 161 (vyzývání) a 162 (asynchronní zprávy, traps).

Řídicí stanice NMS posílá zprávy SNMP na jednotlivá síťová zařízení z cílem získat hodnoty sledovaných objektů systému. Standardní způsob komunikace pracuje na principu vyzývání (polling), kdy řídicí stanice NMS periodicky posílá zprávy typu dotaz na sledovaná zařízení. Požadované hodnoty (objekty MIB) jsou identifikovány pomocí tzv. OID (Object Identifier), které jsou standardizovány v databázi MIB (Management Information Base).

Příkladem objektu je {\tt sysDescr} (System Description) s OID {\tt 1.3.6.1.2.1.1.6.0}. Standardizované objekty jsou uloženy v databázi MIB-2, která je definována standardem RFC 1213 \cite{rfc1213}.

Pro vyhledávání objektů SNMP lze využít následující portály:
\begin{itemize}
  \item Global OID Reference Database: \url{http://oidref.com/}
  \item Object Identifier (OID) Repository: \url{http://www.oid-info.com/}
\end{itemize}

\subsection{Nástroj pro monitorování zařízení SNMP}
Pro monitorování zařízení SNMP můžeme využít nástroj {\tt snmpwalk}, viz {\tt man snmpwalk}, který slouží k získání objektů z daného zařízení protokolem SNMP. Program {\tt snmpwalk} vrátí buď požadový objekt MIB zadaný identifikátorem OID nebo množinu objektů MIB, které patří do zadaného podstromu MIB.

Formát a příklad příkazu {\tt snmpwalk}:
{\small
\begin{verbatim}
snmpwalk -v <protocol> -c <community string> <host> <object>

snmpwalk -v2c -c public isa.fit.vutbr.cz sysDescr         ; vyhledá objekt System Description
snmpwalk -v2c -c public isa.fit.vutbr.cz 1.3.6.1.2.1.1.1  ; ekvivalentní zápis pomocí OID

snmpwalk -v2c -c public isa.fit.vutbr.cz system           ; vyhledá objekty podstromu System
snmpwalk -v2c -c public isa.fit.vutbr.cz 1.3.6.1.2.1.1    ; ekvivalentní zápis pomocí OID
\end{verbatim}
}

\subsection{Konfigurace agentů SNMP}
Zařízení v síti SNMP jsou rozdělena do logických celků, komunit, identifikovaných textovým  řetězcem. Komunita slouží k autentizaci zařízení v síti a definuje přístup k informacím MIB.

V konfiguraci monitorovaného zařízení (agenta SNMP) můžeme definovat, kdo může přistupovat k jakým objektů SNMP. Můžeme definovat způsob přístupu k objektům SNMP (pro čtení: {\tt ro},  pro čtení a zápis: {\tt rw}), můžeme omezit přístup na určité zdrojové adresy, či můžeme definovat část podstromu MIB, do kterého má daný uživatel přístup.  
Příklad nastavení: 
\begin{verbatim}
rocommunity public 10.0.0.0/24
rocommunity public 192.168.0.0/24 1.3.6.1.2.1.1
rwcommunity public localhost
\end{verbatim}
Výše uvedená konfigurace umožňuje zařízením v síti 10.0.0.0/24 přístup k objektům MIB komunity {\tt public} pouze
 pro čtení. Stanice v síti 192.168.0.0/24 mají přístup pro čtení omezen pouze na základní systémové informace uložené v podstromu MIB-2 {\tt system} (OID 1.3.6.1.2.1.1). 
 (podstrom system). Pro lokálního uživatele budou objekty MIB přístupné pro čtení i pro zápis.

 %%%%%%%%%%%%%%%%%%%%%%%%%%%%%%%%%%%%%%
 \newpage
\section{Monitorování toků NetFlow}\label{netflow}
Služba NetFlow byla navržena pro monitorování toků v síti. Její architekturu tvoří sonda NetFlow (exportér), která monitoruje síťový provoz, analyzuje jednotlivé pakety na úrovni L3 a L4, a vytváří tzv. záznamy (flow records) o probíhajícím provozu. Záznamy jsou uložený v lokální paměti sondy, který je po skončení toku přepošle protokole NetFlow na tzv. kolektor. Kolektor ukládá záznamy z různých sond do centrálního úložiště, kde je můžeme vyhledávat, analyzovat, identifikovat potenciálně nebezpečný provoz apod.

Příklad záznamu o toku:
{\footnotesize
\begin{verbatim}
Date flow start       Duration Proto Src IP Addr:Port    Dst IP Addr:Port   Flags Tos Packets Bytes Flows
2005-08-30 06:53:53.370 63.545 TC  113.138.32.15225   ->  222.33.70.124:3575.AP.SF 0    62     3512   1
2005-08-30 06:53:53.370 63.545 TC  222.33.70.124:3575 ->  113.138.32.152:25 .AP.SF 0    58     3300   1
\end{verbatim}
}
Tento záznam obsahuje dva toky TCP, z nichž první tvoří 62 paketů o velikosti 3.512 B, které byly poslány z adresy 113.138.32 a portu 15225 an adresu 222.33.70.124, port 3575. Délka přenosu trvala 63,545 sekund. 


\subsection{Program {\tt nfdump}}
Program {\tt nfdump} slouží jako kolektor systému NetFlow, viz {\tt man nfdump}. Program také umožňuje zobrazovat uložené záznamy, vyhledávat v nich, provádět agregaci, filtrování a další. 

\begin{verbatim}
  Format:   nfdump [options] [filter] 
\end{verbatim}

Příklady použití:
\begin{verbatim}
nfdump -r nfcapd.202004221040  # vypíše všechny záznamy o tocích z daného souboru

nfdump -r nfcapd.202004221040 'host 147.229.5.2'  # vypíše všechny toky dané stanice
 
nfdump -r nfcapd1 -c 100 'proto tcp and (src ip 172.16.17.18 or dst ip 172.16.17.19)'
             # vybere ze souboru 100 záznamů, které vyhovují zadanému filtru

nfdump -r nfcapd1 -s record/bytes -n 10
             # vygeneruje statistiku Top 10 záznamů s největším počtem přenesených bytů

nfdump -r nfcapd1 -s srcip/packets -n 10
             # vygeneruje statistiku Top 10 IP adres s největším počtem odeslaných paketů

nfdump -r nfcapd1 -A srcip,dstport
             # vypíše agregované toky podle zdrojové IP adresy a cílového portu

nfdump -r nfcapd1 -A srcip 'host 147.229.5.2'
             # vypíše agregované toky podle zdrojové IP adresy pro daný filtr

nfdump -r nfcapd1 -t 2020/04/22.12:19:00-2020/04/22.12:20:00 'port 53'
             # vyhledá všechny záznamy s komunikací DNS v daném časovém okně 
\end{verbatim}

%% 
%% Formát záznamu NetFlow umožňuje ukládání informací o IP provozu na síťovém zařízení. Formát byl
%%  vytvořen firmou Cisco jako funkcionalita jejich zařízení. Ve verzi 9 pak byl standartizován
%%  jako RFC 3954 \cite{rfc3954}.
%% 
%% NetFlow záznamy jsou využívány k~monitorování a analýze provozu v~síti. Data obsahují záznamy
%%  o IP adresách, portech, počty přenesených paketů či bajtů apod. Data jsou uchovávána pro
%%  tzv. toky neboli flow. Toky jsou identifikovány pěticí zdrojová a cílová IP adresa,
%%  zdrojový a cílový port a protokol transportní vrstvy. Tok je chápán jako jednosměrná komunikace.
%% 
%% Infrastruktura pro sběr a uchovánvání NetFlow dat sestává z~kolektoru a sond. Kolektorem je míněna
%%  stanice, která sbírá data ze sítě a provádí jejich uchování. Sondou pak může být jakékoli zařízení
%%  ať už směrovač, přepínač nebo zařízení přímo vyhrazené pro tuto činnost. Sběr dat může probíhat
%%  pro všechny toky, nebo může být prováděno vzorkování či agregace. 
%% 
%%%%%%%%%%%%%%%%%%%%%%%%%%%%%%%%%%%%%%

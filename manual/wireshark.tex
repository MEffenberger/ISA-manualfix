\section{Analýza síťového provozu programem Wireshark}
\label{wireshark}

Wireshark je aplikace sloužící k~analýze protokolů a zachytávání paketů. Zachytává síťový provoz z~jednotlivých síťových zařízení (jak fyzických, tak virtuálních) a umožňuje inspekci jednotlivých polí v~daných paketech.

Alternativou ke grafickému Wireshark je command line nástroj \texttt{tcpdump}. Wireshark má oproti \texttt{tcpdump} například integrované volby pro řazení, filtrování a jiné.


\subsection{Zachytávání provozu}
Zachytávat provoz můžeme z~libovolného síťového zařízení, dokonce z~více zařízení najednou.

Odchytíme pomocí dialogu dostupného pomocí \texttt{Capture -> Interfaces...},
kde vybereme zařízení, která chceme snímat.
V dialogu je možné nastavit filtrace provozu již při zachytávání. Tato vlastnost může být užitečná při dlouhodobějším zachytávání, především z~hlediska redukce množství ukládaných paketů.
Tzv. {\emph Capture filter} (zachytávací filtr) můžeme aplikovat před spuštěním
zachytávání. Použijeme \texttt{Options} a v~poli \texttt{Capture Filter:} můžeme
zvolit jeden z~předdefinovaných filtrů, případně specifikovat vlastní (vizte
také \texttt{man 7 pcap-filter}).
Záchyt spustíme pomocí tlačítka
\texttt{Start} -- v tomto nastavení odchytíme veškerý provoz na vybraných zařízeních.

Odchycený provoz můžeme uložit do souboru (přípona \texttt{.pcap}). Později je možné tyto soubory znovu analyzovat.


\subsection{Zobrazovací filtr}
V některých situacích, například pokud zachytáváme veškerý provoz, je velmi vhodné použít filtrování. Jelikož zde pracujeme s velkým množstvím různých paketů, můžeme si tak vyfiltrovat pouze ty relevantní. K tomu slouží tzv. display filtry, jejichž syntax je popsána v Tabulce 1.
\begin{center}
\begin{table}[h]
\centering
\def\arraystretch{1.2}
\begin{tabular}{|l|l|}
\hline
\textbf{Porovnávání} & \texttt{==, >=, <=, !=, contains}\\
\textbf{Logické operátory} & \texttt{||, or, \&\&, and, !, not}\\
\textbf{Kombinace filtrů} & \texttt{(ip.src==192.168.0.105 and udp.port==53)}\\
\textbf{Filtrování na základě existence pole} & \texttt{http.cookie or http.set\_cookie}\\
\textbf{Filtrování specifických bytů} & \texttt{eth.src[4:2]==22:1b}\\
\textbf{Regex filtrování} & \texttt{http.host \&\& !http.host matches "\.com\$"}\\
\hline
\end{tabular}
\caption{Filtrovací operátory}
\end{table}
\end{center}


\subsection{Flow graph}
Wireshark umožňuje zobrazit vybranou komunikaci v~\texttt{flow graph}. Zde můžeme přehledně vidět, které stroje, rozlišené IP adresou, spolu komunikují a kdo komu, zasílá kterou zprávu v~jakém pořadí.
Flow graph si zobrazíme pomocí \texttt{Statistics -> Flow Graph..}, zde si vybereme, co chceme zobrazit.
\subsection{Zobrazení streamu}
Zachycenou komunikaci vidíme v~podobě jednotlivých paketů. Pokud například zachytíme HTTP komunikaci, jeden paket nese většinou pouze část obsahu HTTP zprávy. V tomto případě můžeme pro zobrazení celé zprávy (celého streamu) použít volbu \texttt{Flow <protokol> Stream}. Zobrazení provedeme tak, že vybereme jeden zachycený paket, označíme ho a následně \texttt{Analyze -> Follow <protokol> Stream}.
\subsection{Čas}
V rámci zachytávání paketů nese každý záznam informaci o čase. Defaultně je zde čas zobrazen v~sekundách, od počátku zachytávání komunikace. Nicméně v~\texttt{View -> Time Display Format} můžeme vybrat konkrétní zobrazení času a zjistit tak například, ve který den byl daný paket zachycen.

%%%%%%%%%%%%%%%%%%%%
\section{Vzdálený přístup pomocí SSH (Secure Shell)}
\label{ssh}

Základní funkcí protokolu Secure Shell (SSH) \cite{rfc4253} je vytvořit bezpečný přístup ke vzdálenému počítači přes nezabezpečenou síť. Díky obecnému návrhu protokolu SSH, lze pomocí něj zabezpečit i další služby, jako je např. X Window, přístup ke vzdálenému  souborovému systému (SFTP, sshfs, scp), tunelování portů TCP formou tunelování aplikačních dat skrze protokol SSH . Protokol SSH zajišťuje šifrování dat, autentizaci, integritu dat a volitelně také kompresi. V předmětu ISA budeme protokol SSH využívat pouze k připojení na vzdálený terminál pomocí klíčů. 

Jednou z~nejčastěji používaných implementací protokolu SSH je sada programů OpenSSH\footnote{Viz \url{https://www.openssh.com/}.}. OpenSSH obsahuje kromě klienta {\tt ssh} i serverovou aplikaci {\tt sshd}, program
 pro generování SSH klíčů {\tt ssh-keygen}, agenta pro usnadnění práce s klíči {\tt ssh-agent}
 a další. Podrobnosti ohledně konfigurace serveru lze najít v manuálových stránkách  {\tt sshd(8)} a {\tt sshd\_config(5)}.

\subsection{Připojení ke~vzdálenému počítači}
Pro připojení se ke vzdálenému počítači se jménem \emph{h01} je možné použít příkaz:
\begin{verbatim}
ssh h01
\end{verbatim}
Zde jsou nejčastěji používané parametry příkazu {\tt ssh}, podrobnosti viz {\tt man ssh}:
\begin{itemize}
  \item {\tt -l} přihlašovací jméno na vzdáleném počítači, např. {\tt ssh -l xnovak02 merlin}
  \item {\tt -p} zadání nestandardního portu ssh serveru, např. {\tt ssh -p 6222 merlin}
  \item {\tt -v} výpis podrobností ohledně přihlašování. např. {\tt ssh -v merlin}
  \item {\tt -X} tunelování komunikace protokolu X11  s X serverem přes SSH, např. {\tt ssh -X xnovak02@merlin}
  \item {\tt -L} přesměrování komunikace na daném lokálním portu na vzdálený port přes SSH, např {\tt ssh -N -f -L 25:localhost:25 user@mailserver.example.com}, které vytvoří zabezpečené spojení z lokálního portu 25 na vzdálený port 25. Pro vytvoření tunelu se využije uživatelský účet {\tt user} na vzdáleném mailserveru. Po dobu vytvoření tohoto spojení bude veškerý provoz z lokálního portu 25 tunelován na vzdálený port 25 přes SSH. 
%% %    
%%  Z~těch nejčastěji používaných zmíníme alespoň změnu uživatelského jména ({\tt -l}), specifikování
%%  TCP portu vzdáleného serveru ({\tt -p}), zvýšení výřečnosti programu ({\tt -v}, tento parametr
%%  lze použít i vícekrát), zapnutí tunelování protokolu X Windows ({\tt -X}, {\tt -Y})
%%  a přesměrování portů ({\tt -L}). Často zadávané parametry se specifikují v konfiguračním souboru
%%  \verb|~/.ssh/config|. Popis tohoto souboru obsahuje manuálová stránka {\tt ssh\_config(5)}.
\end{itemize}
 
Po připojení ke vzdálenému serveru jsou informace o~použitém spojení dostupné na serveru v proměnných prostředí typu SSH.  Lze je zobrazit  příkazem {\tt env | grep SSH}, viz následující ukázka
\begin{verbatim}
local $ ssh merlin6.fit.vutbr.cz                          # připojení na server merlin
merlin $ env | grep SSH                                   # výpis proměnných
SSH_CLIENT=2001:67c:1220:80c:e138:4d11:c04c:c675 54514 22 # informace o klientovi
SSH_TTY=/dev/pts/30                                       # lokální terminál
SSH_CONNECTION=2001:67c:1220:80c:e138:4d11:c04c:c675 \    # informace o spojení
  54514 2001:67c:1220:8b0::93e5:b013 22
\end{verbatim}

Z~výpisu vidíme, že klient SSH s adresou IPv6 {\tt 2001:67c:1220:80c:e138:4d11:c04c:c675} se připojil na adresu SSH serveru s adresou IPv6 {\tt 2001:67c:1220:8b0::93e5:b013}. Klientská aplikace běžela na portu 54514 a server na standardním portu 22. Po připojení využíval vzdálený uživatel lokální terminál s identifikátorem 30. 

\subsection{Kopírování souborů mezi počítači}
Pro kopírování souborů protokolem SSH lze využít příkaz {\tt scp} s formátem
\begin{verbatim}
# kopírování lokálního souboru na vzdálený server
  scp <local-file> <username>@<ssh-server>:<path/remote-file>

# kopírování vzdáleného souboru na lokální server 
  scp <username>@<ssh-server>:<path/remote-file> <local-file>
\end{verbatim}
Následující příkaz zkopíruje lokální soubor {\tt isa.txt} na vzdálený počítač {\tt h01} do adresáře {\t fit} umístěném v~domovském adresáři uživatele {\tt student}.
\begin{verbatim}
scp isa.txt student@h01:~/fit/
\end{verbatim}

\subsection{Tunelování provozu přes SSH}
Pomocí protokolu SSH je možné tunelovat provoz z daného zařízení na vzdálený počítač. Tunelování SSH zabezpečí původně nešifrovaný provoz, který je vložen do šifrované komunikace SSH. Následující příkaz otevře spojení SSH, které pak můžeme použít jako proxy ve webovém prohlížeči.
\begin{verbatim}
  ssh -D 12345 -N student@sshserver
\end{verbatim}
Po otevření tohoto tunelu SSH stačí v~prohlížeči zadat nastavení proxy serveru {\tt localhost} s~portem 12345. Provoz pak bude přesměrován na lokální počítač, poté půjde spojením SSH na  {\tt sshserver}, kde bude převeden na běžný HTTP/HTTPS provoz a~poslán dále. 

\subsection{Generování klíčů SSH}
Pro autentizaci využívá SSH asymetrickou kryptografii, konkrétně veřejné a soukromé klíče (tzv. Public Key Infrastructure, PKI). 
%% Klíče protokolu SSH mají několik výhod. Díky nim není nutné posílat heslo pro přístup ke vzdálenému
%%  stroji přes síť, byť v~šifrované podobě. Délka používaných klíčů znesnadňuje potenciálnímu
%%  útočníkovi útok hrubou silou, protože síla klíče bývá typicky vyšší než síla běžně používaných
%%  hesel. Ve spojení s~agentem pro správu klíče může uživatel přistupovat ke vzdálenému serveru bez
%%  nutnosti opakovaného zadávání hesla. Agent může být nastaven, aby pro použitý klíč nevyžadoval
%%  heslo po zbytek sezení, či po určitý počet minut.
%% 
SSH klíč se generuje příkazem {\tt ssh-keygen}. Po spuštění bez parametrů je vytvořen pár klíčů (veřejný a soukromý klíč) algoritmem RSA o~délce 2048\,B. Uživatel je dotázán na umístění souborů s klíči v souborovém systému (standardně adresář {\tt ~/.ssh}. Dále zadává uživatel heslo (passphrase), které chrání přístup k soukromému klíči, tzn. použití tohoto klíče. Parametr {\tt -t} nastavuje typ šifrovacího algoritmu (implicitně RSA, může být DSA, EC DSA a další). Parametr {\tt -b} specifikuje délku klíče (počet bitů klíče), pro RSA je implicitní hodnota nyní 3072. Parametr {\tt -f} specifikuje jméno souboru, do kterého se klíč ukládá, parametr {\tt -C} přidává komentář ke klíči. Podrobnější informace je možno nalézt v  manuálové stránce {\tt ssh-keygen(1)}.

Následující příklad vygeneruje klíč o~délce 4096\,B, který nebude chráněn žádným heslem. Klíč bude uložen v souboru {\tt .ssh/nopass}:
\begin{verbatim}
ssh-keygen -t rsa -b 4096 -N "" -f ~/.ssh/nopass -C nopass
\end{verbatim}

Po vytvoření klíčů vzniknou dva soubory. Klíč v souboru bez přípony je soukromý klíč, který by měl zůstat utajený a uživatel by ho neměl dále distribuovat. V souboru s~příponou {\tt .pub} je veřejný klíč, který slouží k ověřování a je určen k zveřejnění\footnote{Popis používání klíčů je například na stránce na \url{https://www.digitalocean.com/community/tutorials/understanding-the-ssh-encryption-and-connection-process}}.

\subsection{Použití klíčů SSH}
Pro správné použití zabezpečení SSH je potřeba vygenerovaný veřejný klíč umístit na vzdálený počítač, aby tento vzdálený počítač mohl správně ověřit připojujícího se uživatel. Veřejný klíč v souboru se jménem např.  {\tt id\_rsa.pub} je potřeba přenést na vzdálených počítač a uložit (přidat) do souboru \verb| ~/.ssh/authorized_keys|, kde ho bude hledat server SSH při pokusu o přihlášení.

Na vzdálený počítač můžeme veřejný klíč přenést programem {\tt scp}. Do souboru {\tt authorized\_keys} ho můžeme přidat příkazem {\tt cat}, viz následující příklad. Je potřeba striktně dodržovat názvy souborů -- pokud dojde k překlepu ve jménu souboru (např. {\tt authorized\_key}), server SSH klíče nenajde a nebude provádět autentizaci vůči klíčům SSH. 
\begin{verbatim}
cat id_rsa.pub >> ~/.ssh/authorized_keys
\end{verbatim}

Nyní se již můžeme připojit ke vzdálenému počítači pomocí vytvořených klíčů. Pokud je soukromý klíč SSH na lokálním počítači umístěn ve výchozím nastavení, použije se automaticky. Pokud jsme  zvolili jiné umístění, je potřeba program {\tt ssh} informovat o~umístění soukromého klíče parametrem  {\tt -i}, nebo volbou {\tt IdentityFile} v~konfiguračním souboru. Pořadí hledání a použití klíčů u klienta SSH lze vypsat parametrem {\tt -v}, např. {\tt ssh -v merlin}.


\documentclass[a4paper,11pt]{article}

\usepackage[utf8]{inputenc}
\usepackage[czech]{babel}
\usepackage[left=2cm,top=3cm,text={17cm,24cm}]{geometry}
\usepackage{graphicx}
\usepackage{listings}
\usepackage{url}

\title{Základní konfigurace síťových zařízení a analýza síťového provozu programem Wireshark\\
{\bf\large ISA - Laboratorní cvičení č.1}\\
{\bf\large Protokol ke cvičení}}

\author{Vysoké učení technické v Brně}

\date{\url{https://github.com/nesfit/ISA/tree/master/lab1-konfigurace}}

\setlength\parindent{0pt}

\begin{document}

{\let\newpage\relax\maketitle}

%Jméno:\\
Login:\\
Skupina cvičení:\\
Dnešní datum:\\

\section{Zjišťování konfigurace}
\textbf{1.}\\
Rozhraní \texttt{enp2s0}\\
\\
\begin{tabular}{|l|r|}
\hline
MAC adresa: & \hspace{12em} \\
\hline
IPv4 adresa: & \\
\hline
Délka prefixu: & \\
\hline
Adresa sítě: & \\
\hline
Broadcastová adresa: & \\
\hline
\end{tabular}
\vspace{1em}
\\
\textbf{2.}\\
IPv4 adresa výchozí brány:\\
MAC adresa výchozí brány:\\
\\
\textbf{3.}\\
Použité příkazy: \\
\\
\\
\textbf{4.}\\
Implicitní DNS servery:\\
\vskip 2em
\textbf{5.}\\
Úpravený záznam:\\
\\
\textbf{6.}\\
Záznam + popis:\\
\vskip 8em

\textbf{8.}\\
Použitý příkaz:\\
\\
\textbf{9.}\\
Použitý příkaz:\\
Chybová zpráva:

\section{Wireshark}
\textbf{1.}\\
Capture filter:\\
\\
\textbf{2.}
\\
\vspace{-1em}
\\
\begin{tabular}{|l|c|c|c|c|}
\hline
& \multicolumn{2}{|c|}{\textbf{Požadavek HTTP}} & \multicolumn{2}{|c|}{\textbf{Odpověď HTTP}}\\
\hline
\textbf{Hodnota} & \textbf{Adresa} & \textbf{Role zařízení} & \textbf{Adresa} & \textbf{Role zařízení}\\
\hline
Cílová MAC adresa & \hspace{8em} & \hspace{6em} & \hspace{8em} & \hspace{6em} \\
\hline
Cílová IPv4 adresa & & & & \\
\hline
Zdrojová MAC adresa & & & & \\
\hline
Zdrojová IPv4 adresa & & & & \\
\hline
\end{tabular}
\\
\\
Komu patří nalezené IPv4 adresy a MAC adresy? Vypisovali jste již některé z nich? Proč tomu tak je?\\
\\


\textbf{3.}\\
Display Filtr:\\
Zamyslete se nad tím jaké výhody a nevýhody má použití Capture a Display
Filteru:
\\
\\
\\
\\
\textbf{4.}\\
Jaká je závislost a návaznost paketů protokolů DNS a HTTP(S) při komunikaci?\\
\\
\\
\\
\textbf{6.}\\
Jaký je formát zobrazených dat funkcí \emph{Follow TCP stream} a \emph{Follow HTTP stream}? Slovně popište jejich význam:\\
\\

\section{Konfigurace IPv4 a IPv6}

\begin{tabular}{|l|r|}
\hline
Zvolená maska sítě pro IPv4: & \hspace{15em} \\ \hline
Použité adresy IPv4: & \\ \hline
Maximální velikost skupiny: & \\ \hline \hline
Zvolená adresa sítě pro IPv6: & \\ \hline
Použité adresy IPv6: & \\ \hline
Maximální velikost skupiny: & \\ \hline
\end{tabular}


%\thispagestyle{empty}

Jméno a příjmení:\\
Login:\\
Skupina (číslo nebo čas):\\
Datum:\\

\section{Zjišťování konfigurace}
\textbf{1.1}
Konfigurace rozhraní \texttt{enp2s0}\\
\\
\begin{tabular}{|l|r|}
\hline
MAC adresa: & \hspace{20em} \\
\hline
IPv4 adresa: & \\
\hline
Délka prefixu (v bitech): & \\
\hline
Adresa sítě: & \\
\hline
Broadcastová adresa: & \\
\hline
\end{tabular}
\vspace{1em}
\\
\textbf{1.2} Výchozí brána (default gateway)
\medskip

IPv4 adresa:\\
MAC adresa:\\
\\
\textbf{1.4} Implicitní DNS servery:\\
\vskip 2em

\textbf{1.5} Přidaný záznam v {\tt /etc/hosts}:\\
\vskip 2em

\textbf{1.6} Příklad aktivního spojení + vysvětlení položek:\\
\vskip 8em

\textbf{1.9} Vyhledání chybové zprávy v systémovém logu
\medskip

Použitý příkaz:\\

Chybová zpráva:

\section{Wireshark}
\textbf{2.1} Capture filter:\\
\vskip 1em

\textbf{2.5} Analýza komunikace HTTP\\

\begin{tabular}{|l|c|c|c|c|}
\hline
& \multicolumn{2}{|c|}{\textbf{Požadavek HTTP}} & \multicolumn{2}{|c|}{\textbf{Odpověď HTTP}}\\
\hline
\textbf{Hodnota} & \textbf{Adresa} & \textbf{Typ zařízení} & \textbf{Adresa} & \textbf{Typ zařízení}\\
\hline
Zdrojová MAC adresa & & & & \\
\hline
Zdrojová IPv4 adresa & & & & \\
\hline
Cílová MAC adresa & \hspace{8em} & \hspace{6em} & \hspace{8em} & \hspace{6em} \\
\hline
Cílová IPv4 adresa & & & & \\
\hline
\end{tabular}

\bigskip
U typu zařízení popište, k jakému zařízení je daná adresa přiřazena, např. webový klient, klientský počítač, brána apod. 

\bigskip

\textbf{2.6} Popište formát výstupu funkcí {\em Follow TCP stream} a {\em Follow HTTP stream}. \\
\medskip

\emph{Follow TCP stream:}\\
\vskip 2em

\emph{Follow HTTP stream:}\\
\vskip 2em 

\textbf{2.10} Nastavení filtru pro zobrazení (display filter) v programu Wireshark:\\
\medskip

\textbf{2.11} Popište souvislost a návaznost odchycených paketů DNS a HTTP(S):\\
\vskip 5em

\section{Konfigurace IPv4 a IPv6}

\begin{tabular}{|l|r|}
\hline
Zvolená maska podsítě pro IPv4: & \hspace{20em} \\ \hline
Použitá adresa IPv4: & \\ \hline
Maximální počet zařízení v podsíti: & \\ \hline \hline
Zvolený prefix IPv6 sítě: & \\ \hline
Použitá adresa IPv6: & \\ \hline
Maximální počet zařízení v podsíti: & \\ \hline
\end{tabular}

\end{document}

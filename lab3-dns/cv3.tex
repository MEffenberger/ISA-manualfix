\documentclass[a4paper,11pt]{article}
\usepackage[utf8]{inputenc}
\usepackage[czech]{babel}
\usepackage[left=2cm,top=3cm,text={17cm,24cm}]{geometry}
\usepackage{graphicx}
\usepackage{listings}
\usepackage{url}
\usepackage{xr}
\externaldocument{../manual/manual}

\title{Systém DNS\\
{\bf\large ISA - Laboratorní cvičení č.3}}

\author{Vysoké učení technické v Brně}

\date{\url{https://github.com/nesfit/ISA/tree/master/lab3-dns}}

\begin{document}

{\let\newpage\relax\maketitle}

\section*{Cíl laboratoře}
\begin{itemize}
  \item Seznámit se se službou DNS
  \item Nakonfigurovat server DNS s vlastní doménou.
\end{itemize}

\section*{Obecné pokyny}
\begin{itemize}
  \item Vypněte si mobily a uschovejte do tašek. Není dovoleno používat mobily během výuky.
  \item Přihlaste se do OS Linux (F3) jako uživatel {\tt user}, heslo {\tt user4lab}.
  \item Otevřete si terminálové okno pro uživatele {\tt user}.
  \item Otevřete si terminálové okno pro uživatele {\tt root} příkazem {\tt su}
    (switch user).
  \item Pro editaci konfiguračních souborů použijte libovolný editor (nano, mcedit, vim, gedit).
\end{itemize}

\section{Služba DNS}
Služba DNS slouží k ukládání a vyhledávání záznamů DNS v globálním distribuovaném prostoru DNS. Pro komunikaci využívá protokol DNS. Podrobnější informace ke službě DNS, záznamům DNS a konfiguraci serveru najdete v laboratorním manuálu, kapitola \ref{dns}.
\begin{enumerate}
    \item Spusťte program Wireshark\footnote{Aplikaci Wireshark spusťte jako uživatel \texttt{root} z~příkazové řádky příkazem "\texttt{wireshark}".} a zachytávejte komunikaci DNS na rozhraní \texttt{enp2s0}.
    \item Otevřete si nový terminál a pomocí příkazu \texttt{dig www.vutbr.cz a} zjistěte, na jakou IPv4 adresu se překládá doménové jméno \texttt{www.vutbr.cz}. Zjištěnou IPv4 adresu zapište do protokolu.
    \item Zastavte zachytávání v programu Wireshark a prohlédněte si komunikaci DNS. Zjistěte, na jaký server DNS směřují dotazy DNS z vašeho počítače a v jaké sekci paketu DNS je uvedena hledaná doména. Výsledky zapište do protokolu.
    \item Z hlavičky DNS, část \texttt{FLAGS} zjistěte, jaký je typ odeslaného dotazu (rekurzivní či iterativní) a zda je dotazovaný server autoritativní (viz příznak v odpovědi DNS). Výsledek zapište do protokolu.
    \item Pomocí programu {\tt dig} vyhledejte informace o doménách {\tt fit.vutbr.cz, www.fit.vutbr.cz} a zapište je do protokolu. Využijte záznamy \texttt{A, AAAA, SOA, MX} a {\tt NS}.
\end{enumerate}

% \section{Služba Whois}
% \begin{enumerate}
%     \item Zadejte do terminálu příkaz \texttt{whois vutbr.cz}. Prohlédněte si zobrazené informace o této doméně. Zapište do protokolu, ve kterém roce byla doména v systému Whois zaregistrována.
%     \item Zjistěte IP adresu vašeho počítače. Využijte například  webovou stránku \url{www.fit.vut.cz}, která zobrazí vaši veřejnou IP adresu vpravo dole. Pomocí příkazu \texttt{ip address show} zjistěte nakonfigurovanou IP adresu vašeho počítače. Adresy zapište do protokolu a vysvětlete, proč se liší.
%     \item Pomocí služby Whois zjistěte, do jakého rozsahu IP adres patří vaše veřejná IP adresa, který regionální registrátor (RIR) daný rozsah spravuje a jakému koncového uživateli je tento rozsah IP adres přidělen. Použijte příkaz \texttt{whois <IP-adresa>}. Výsledek zapište do protokolu.
% \end{enumerate}

\section{Konfigurace vlastního DNS serveru}
\begin{enumerate}
  \item Prostudujte si začátek podkapitoly \ref{sec:vlastni_zona} z laboratorního manuálu. V následující části budeme konfigurovat vlastní DNS zónu na DNS serveru BIND.
  \item Zvolte si název vlastní doménu ve tvaru {\tt xlogin00.cz}, kde místo řetězce {\tt xlogin00} použijte svůj vlastní login, například {\tt xnovak01.cz}.

  \item Jako uživatel {\tt root} \textbf{zkopírujte} (příkaz \texttt{cp}) šablonu zónového souboru {\tt /root/isa3/template.dns.zone} do adresáře {\tt /var/named}. Přejmenujte soubor se šablonou na název vaší domény {\tt xlogin00.cz}.
  \item Upravte zónový soubor vaší domény {\tt xlogin00.cz} tak, aby obsahoval následující informace:
    \begin{enumerate}
      \item hodnota \texttt{TTL},
      \item název primárního serveru DNS bude  {\tt ns1.xlogin00.cz},
      \item e-mailová adresa na správce DNS bude {\tt admin@xlogin00.cz},
      \item sériové číslo záznamu SOA zapište ve tvaru {\tt yyyymmdd}, které odpovídá datu poslední změny,
      \item vytvořte překlad doménového jména serveru {\tt ns1.xlogin00.cz} na IP adresu vašeho počítače,
      \item vytvořte překlad doménového jména {\tt pcuc.xlogin00.cz} na IP adresu {\tt 10.10.10.1},
      \item vytvořte alias pro DNS server se jménem {\tt server.xlogin00.cz}.
    \end{enumerate}
  \item Tyto úpravy zahrnují vytvoření či úpravu záznamů typu SOA, NS, A a CNAME. Příklad zónového souboru pro doménu {\tt example00.cz} je níže:
\begin{verbatim}
$TTL   3600
@      IN  SOA    ns1.example00.cz. admin.example00.cz. (
                     2016072701        ; serial number
                     3600              ; refresh period
                     600               ; retry period
                     604800            ; expire time
                     1800              ; minimum ttl
                  )
@       IN  NS    ns1.example00.cz.
ns1     IN  A     192.168.0.1
pc1     IN  A     192.168.0.2
webapp  IN  CNAME ns1.example00.cz.
\end{verbatim}

\item Upravte konfigurační soubor {\tt /etc/named.conf} serveru DNS. Vložte do něho odkaz na vytvořený zónový soubor pro vaši doménu {\tt xlogin00.cz}. Zónu vložte na konec souboru před klíčové slovo {\tt include}. Při vytvoření nové zóny v konfiguračním souboru DNS se můžete inspirovat ukázkovým konfiguračním souborem {\tt /root/isa3/named.conf.local}, případně následujícím příkladem:
\begin{verbatim}
zone "xlogin00.cz" IN {
       type master;
       file "/var/named/xlogin00.cz";
};
\end{verbatim}
  \item Spusťte server DNS příkazem {\tt systemctl start named.service}. Příkazem {\tt systemctl status named} ověřte, zda služba správně běží\footnote{Pro kontrolu syntaxe zónového souboru a konfiguračního souboru DNS lze také použít nástroje {\tt named-checkzone} a {\tt named-checkconf}, viz {\tt man named-checkzone} a {\tt man named-checkconf}.}. Chyby související s IPv6 můžete ignorovat. Pokud budete chtít službu restartovat, použijte příkaz {\tt systemctl restart named.service}.

  \item V souboru {\tt /etc/resolv.conf} nastavte ručně adresu lokálního serveru DNS na IP adresu {\tt 127.0.0.1}. Řádek {\tt search netlab.fit.vutbr.cz} smažte.
  \item Ověřte vytvořenou konfigurace následujícími příkazy. Výstupy vybraných příkazů zapište do laboratorního protokolu.
    \begin{itemize}
        \item {\tt dig xlogin00.cz soa}
        \item {\tt dig xlogin00.cz ns}
        \item {\tt dig server.xlogin00.cz}
        \item {\tt dig pcuc.xlogin00.cz}
        \item {\tt ping server.xlogin00.cz}
        \item {\tt ping pcuc.xlogin00.cz}
    \end{itemize}

  \item Nakonfigurujte také reverzní záznamy pro vaši doménu. Protože počítače ve vaší doméně ukazují na adresy v rozsahu 10.10.10.0/8, bude potřeba vytvořit doménu pro reverzní záznamy se jménem \texttt{10.10.10.in-addr.arpa}. Pro vytvoření využijte šablonu v souboru {\tt /root/isa3/db.127}, kterou \textbf{zkopírujte} (příkaz \texttt{cp}) do adresáře {\tt /var/named} pod názvem {\tt 10.10.10}.
  \item Upravte zónový soubor pro reverzní překlad tak, aby obsahoval následující informace:
    \begin{enumerate}
      \item jméno primárního serveru DNS bude {\tt ns1.xlogin00.cz},
      \item e-mailová adresa na správce DNS bude {\tt admin@xlogin00.cz},
      \item sériové číslo záznamu SOA zapište ve tvaru {\tt yyyymmdd}, které odpovídá datu poslední změny,
      \item vytvořte záznam pro překlad IP adresy vašeho počítače na doménové jméno {\tt ns1.xlogin00.cz},
      \item vytvořte překlad IP adresy {\tt 10.10.10.1} na doménové jméno {\tt pcuc.xlogin00.cz}.
    \end{enumerate}
  \item Tyto úpravy zahrnují záznamy SOA, NS a PTR. Příklad reverzního zónového souboru pro doménu {\tt 0.168.192.in-addr.arpa}:
    \vspace{-3mm}
\begin{verbatim}
$TTL 3600
@   IN SOA  ns1.example00.cz. admin.example00.cz. (
               2016072701       ; serial number
               3600             ; refresh period
               600              ; retry period
               604800           ; expire time
               1800             ; minimum ttl
            )
@   IN  NS  ns1.example00.cz.
1   IN  PTR ns1.example00.cz.
\end{verbatim}
  \enlargethispage{2mm}
\item Do konfiguračního souboru {\tt /etc/named.conf} vložte novou zónu pro doménu "10.10.10.in-addr.arpa". Zónu vložte na konec konfiguračního souboru před klíčové slovo {\tt include}. Příklad reverzní zóny pro doménu {\tt 0.168.192.in-addr.arpa}:
\begin{verbatim}
zone "0.168.192.in-addr.arpa" IN {
       type master;
       file "/var/named/192.168.0";
};
\end{verbatim}


  \item Restartujte server DNS příkazem {\tt systemctl restart named.service}. Ověřte, zda služba běží správně.

    \item Následujícími příkazy otestujte nastavení. Výstupy vybraných příkazů zapište do laboratorního protokolu.
    \begin{itemize}
        \item {\tt dig -x 10.10.10.1xx}, kde {\tt xx} je číslo vašeho počítače;
        \item {\tt dig -x 10.10.10.1}
    \end{itemize}

  \item Předveďte výsledky konfigurace vyučujícímu pomocí následujících testů:
\begin{verbatim}
  dig xlogin00.cz soa       # řetězec xlogin00.cz nahraďte svou doménou
  dig pcuc.xlogin00.cz
  dig server.xlogin00.cz

  dig -x 10.10.10.xxx       # zde vložte IP adresu svého počítače
  dig -x 10.10.10.1
\end{verbatim}

%  \item Upravte IP adresu výchozího DNS serveru tímto postupem:
%  \begin{itemize}
%    \item Na konec souboru {\tt /etc/sysconfig/network-scripts/ifcfg-Wired\char`_connection\char`_1} přidejte následující dva řádky:\\
%          \verb|PEERDNS=no|\\
%          \verb|DNS1=127.0.0.1|
    %    \item Vypněte a znovu zapněte síťové rozhraní {\tt enp2s0} příkazy: {\tt ifdown Wired\char`_connection\char`_1}; {\tt ifup Wired\char`_connection\char`_1}
%  \end{itemize}

  % \item V~programu Wireshark začněte zachytávat  komunikaci DNS na rozhraní {\tt loopback}.
  % \item V~terminálu zadejte {\tt ping PCUC.xlogin00.cz} a ověřte, zda došlo ke správnému překladu doméno\-vého jména na IP adresu. Prohlédněte si jednotlivé části dotazu a odpovědi DNS zachycené programem Wireshark.
  % \item Otestujte funkčnost dalších záznamů DNS:\\
  %   {\tt ping server.xlogin00.cz},\\
  %   {\tt ping PC01.xlogin00.cz},\\
  %   {\tt nslookup -type=soa xlogin00.cz},\\
  %   {\tt dig xlogin00.cz},\\
  %   {\tt dig xlogin00.cz NS}.
  % \item Do protokolu zapište výstupy následujících příkazu:\\
  %   {\tt nslookup 10.10.10.1},\\
  %   {\tt dig PCUC.xlogin00.cz},\\
  %   {\tt dig -x 10.10.10.1xx}, kde {\tt xx} je číslo vašeho počítače.
  % \item Demonstrujte funkčnost vašeho řešení cvičícímu. Ukažte výstupy z následujících dotazů: \\
  %   {\tt nslookup PCUC.xlogin00.cz}, \\
  %   {\tt nslookup -type=SOA xlogin.cz}, \\
  %   {\tt nslookup -type=NS xlogin.cz}, \\
  %   {\tt nslookup 10.10.10.1xx}.

  \item V~terminálu zadejte příkaz \texttt{dig nes.fit.vutbr.cz +trace +nodnssec}, který provede rezoluci daného doménového jména a vypíše  její podrobný průběh.

  \item \label{wireshark_step} Spusťte záchyt paketů v programu Wireshark.

  \item V~terminálu zadejte příkaz \texttt{dig nes.fit.vutbr.cz +trace +nodnssec} znovu.

  \item Zastavte zachytávání provozu DNS v programu Wireshark.

  \item Proveďte analýzu rezoluce na základě výpisu programu {\tt dig}. Do protokolu zapište, na které servery se dotazy DNS posílali a v jakém pořadí. Do protokolu zapište IP adresy a doménová jména serverů. Výstup můžete také zkontrolovat podle zachyceného provozu ve Wiresharku, využijte zobrazení toků v menu {\tt Statistics/Flow Graph}. Pokud diagram nesedí s diagramem v protokolu, vraťte se na bod \ref{wireshark_step}.

%%zachycený DNS provoz. Do protokolu doplňte abstraktní sekvenční diagram procesu této rezoluce. Jako názvy jednotlivých procesů doplňte názvy kontaktovaných serverů. Do protokolu také uveďte, jaký byl typ DNS rezoluce z pohledu programu \texttt{dig}.
%%
%%
%%
%% zpráv si \texttt{dig} a \texttt{bind} vyměnili na rozhraní \texttt{localhost}?
\end{enumerate}

\section{Ukončení práce v~laboratoři}
\begin{itemize}
  \item Jako uživatel {\tt root} spusťte dávkou {\tt /root/isa3/clean}, která vyčistí vytvořenou konfiguraci a vypne počítač.
\end{itemize}

\end{document}

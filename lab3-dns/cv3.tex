\documentclass[a4paper,11pt]{article}
\usepackage[utf8]{inputenc}
\usepackage[czech]{babel}
\usepackage[left=2cm,top=3cm,text={17cm,24cm}]{geometry}
\usepackage{graphicx}
\usepackage{listings}
\usepackage{url}
\usepackage{xr}
\externaldocument{../manual/manual}

\title{Systém DNS\\
{\bf\large ISA - Laboratorní cvičení č.3}}

\author{Vysoké učení technické v Brně}

\date{\url{https://github.com/nesfit/ISA/tree/master/lab3-dns}}

\begin{document}

{\let\newpage\relax\maketitle}

\section*{Cíl laboratoře}
\begin{itemize}
  \item Seznámit se se službou DNS a databází Whois.
  \item Nakonfigurovat server DNS s vlastní doménou. 
\end{itemize}

\section*{Obecné pokyny}
\begin{itemize}
  \item Vypněte si mobily a uschovejte do tašek. Není dovoleno používat mobily během výuky. 
  \item Přihlaste se do OS Linux (F3) jako uživatel {\tt user}, heslo {\tt user4lab}.
  \item Otevřete si terminálové okno pro uživatele {\tt user}.
  \item Otevřete si terminálové okno pro uživatele {\tt root} příkazem {\tt su}
    (switch user).
  \item Pro editaci konfiguračních souborů použijte libovolný editor (nano, mcedit, vim, gedit).
\end{itemize}

\section{Služba DNS}
Služba DNS slouží k ukládání a vyhledávání záznamů DNS v globálním distribuovaném prostoru DNS. Pro komunikaci využívá protokol DNS. Podrobnější informace ke službě DNS, záznamům DNS a konfiguraci serveru najdete v laboratorním manuálu, kapitola \ref{dns}. 
\begin{enumerate}
    \item Spusťte program Wireshark\footnote{Aplikaci Wireshark spuťte jako uživatel \texttt{root} z~příkazové řádky příkazem "\texttt{wireshark \&}".} a zachytávejte komunikaci DNS na rozhraní \texttt{enp2s0}.
    \item Otevřete terminál a pomocí příkazu \texttt{nslookup -type=a www.vutbr.cz} zjistěte, na jakou IPv4 adresu se překládá doménové jméno \texttt{www.vutbr.cz}. Zjištěnou IPv4 adresu zapište do protokolu.
    \item Prohlédněte si strukturu dotazů a odpovědí DNS v~programu Wireshark. Podle čeho klient DNS pozná, která odpověď náleží k jakému dotazu? 
    \item Příkazem \texttt{nslookup -type=soa vutbr.cz} zobrazte záznam SOA naší univerzity. Prohlédněte si políčka, která záznam SOA obsahuje. 
    \item V~programu Wireshark znovu spusťte zachytávání komunikace DNS.
    \item Pomocí příkazu \texttt{nslookup -type=<X> <doména>} zjistěte autoritativní servery DNS pro doménu \texttt{vutbr.cz}. Hodnotu \texttt{<X>} nahraďte typem záznamu DNS, které obsahují názvy autoritativních serverů dané domény. Výsledek dotazu zapiště do protokolu.
    \item Zastavte zachytávání komunikace v~programu Wireshark.
    \item V~zachyceném provozu prozkoumejte pakety obsahující komunikaci poslední rezoluce DNS.
    \item Podle příznaku v poli {\tt Flags} zjistěte, zda se jedná o rekurzivní či iterativní dotaz DNS. Zjištěný údaj zapište do protokolu.
    \item Uveďte, na jakou IP adresu daný dotaz směřoval. Jak zjistí klient adresu serveru DNS, na který bude dotazy směrovat?    \item Zobrazte záznamy typu \texttt{MX} pro doménu \texttt{fit.vutbr.cz}. Zapište do protokolu, jaký je primární e-mailový server pro doménu \texttt{fit.vutbr.cz}.
\end{enumerate}

\section{Služba Whois}
\begin{enumerate}
    \item Zadejte do terminálu příkaz \texttt{whois vutbr.cz}. Prohlédněte si zobrazené informace o doméně. Zapište do protokolu, ve kterém roce byla doména v systému Whois zaregistrována.
    \item Zjistěte veřejnou IP adresu vašeho počítače. Využijte například  webovou stránku \url{www.fit.vut.cz}, která zobrazí vaši veřejnou IP adresu vpravo dole. Nalezenou adresu zapiště do protokolu. Pomocí příkazu \texttt{ip address show} zjistěte nakonfigurovanou IP adresy vašeho počítače. Nalezené adresy zapište do protokolu a vysvětlete, proč se liší. 
    \item Pomocí služby Whois zjistěte, do jakého rozsahu IP adres vaše veřejná IP adresa patří a komu byl přidělen. Použijte příkaz \texttt{whois <vase-verejna-IP-adresa>}. Výsledek zapište do protokolu.
\end{enumerate}

%\section{Blokování vybraných domén}
%\begin{enumerate}
%	\item Do souboru \texttt{/etc/hosts} přidejte následující řádek:\\
%    \verb|0.0.0.0    www.facebook.com|
%    \item Ve webovém prohlížeči zadejte do adresního řádku \url{www.facebook.com}. Podařilo se Vám stránku zobrazit? Pokud ano, restartujte webový prohlížeč, aby se vymazala DNS mezipaměť webového prohlížeče, a zkuste znovu.
%    \item Zamyslete se nad slabinami tohoto řešení omezení přístupu na webové stránky. Jak lze toto řešení (jednoduše) obejít?
%\end{enumerate}

\section{Konfigurace vlastního DNS serveru}
\begin{enumerate}
  \item Prostudujte si začátek podkapitoly \ref{sec:vlastni_zona} z laboratorního manuálu.
  \item Zvolte si jméno vlastní doménu ve tvaru {\tt xlogin00.cz}, například {\tt xnovak01.cz}. Název domény, tj. řetězec {\tt xlogin00.cz}, bude využívat v názvech souborů a v konfiguraci.

\item Pro svou doménu vytvořte zónový soubor dle následujících instrukcí (jako uživatel {\tt root}):
    \begin{enumerate}
      \item Šablonu přímého zónového souboru najdete v souboru {\tt /root/isa3/template.dns.zone}. Zkopírujte tento soubor do složky {\tt /var/named} pod novým jménem "xlogin00.cz".
      \item V~nově vytvořeném souboru {\tt /var/named/xlogin00.cz} upravte záznam SOA pro svou doménu {\tt xlogin00.cz}. Jméno primárního serveru DNS bude "{\tt ns1.xlogin00.cz.}". \rm E-mailovou adresu správce nastavte na hodnotu "{\tt admin.xlogin00.cz.}"\rm \footnote{V záznamu SOA se znak "@" v~e-mailové adrese nahrazuje znakem ".".}.
      \item V záznamu SOA aktualizujte sériové číslo, aby odpovídalo dnešnímu datu ve tvaru {\tt yyyymmdd}.
      \item Vytvořte záznam NS pro svou doménu. Záznam ukazuje na váš server "{\tt ns1.xlogin00.cz.}"\rm.
      \item Pro server "{\tt ns1.xlogin00.cz.}"\rm \; vytvořte záznam A, který se odkazuje na IP adresu Vašeho počítače, tj. adresu na rozhraní {\tt enp2s0}. 
      \item Do zónového souboru přidejte další záznam A, který ukazuje na učitelský počítač v~laboratoři. Záznam zadejte ve tvaru: \verb|PCUC    IN    A    10.10.10.1|
      \item Přidejte další záznamy A, které ukazují na tři libovolné počítače v~laboratoři, např. PC01, PC02 a PC03.
      \item Přidejte záznam CNAME, který vytvoří alias s názvem {\tt server} pro počítač {\tt ns1.xlogin00.cz.}
      \item V~případě zájmu nakonfigurujte pro doménu překlad na adresy IPv6 (záznamy AAAA).
    \end{enumerate}
  
  \item Nyní upravte konfigurační soubor {\tt /etc/named.conf}. Při úpravách se můžete inspirovat ukázkovým konfiguračním souborem {\tt /root/isa3/named.conf.local}. Proveďte následující úpravy:
    \begin{itemize}
      \item Vytvořte novou zónu pro vaši doménu {\tt xlogin.cz}. Zónu vložte téměř na konec souboru před klíčové slovo {\tt include}.
      \item Doplňte absolutní cestu k~vašemu zónovému souboru \texttt{xlogin00.cz}.
    \end{itemize}
  \item Spusťte server DNS příkazem {\tt systemctl start named.service}. Příkazem {\tt systemctl status named} ověříte, zda služba správně běží. Chyby související s IPv6 můžete ignorovat.
  
  \item Po úspěšném spuštění serveru DNS nakonfigurujte reverzní překlad pro vaši doménu. Zónový soubor pro reverzní doménu {\tt 10.10.10.in-addr.arpa.} vytvořte následujícím způsobem:
    \begin{enumerate}
      \item Jako šablonu zónového souboru pro reverzní překlad použijte soubor {\tt /root/isa3/temp\-late.dns.zone}.
            Zkopírujte tento soubor do složky {\tt /var/named} pod názvem "10.10.10".
      \item V~nově vytvořeném souboru {\tt /var/named/10.10.10} upravte záznam SOA pro doménu {\tt 10.10.10.in-addr.arpa.} Autoritativní server bude opět "{\tt ns1.xlogin00.cz.}"\rm. E-mailová adresa správce zůstává "{\tt admin.xlogin00.cz.}"\rm.
      \item V záznamu SOA aktualizujte sériové číslo, aby odpovídalo dnešnímu datu.
      \item Vytvořte záznam NS, který ukazuje na autoritativní server "{\tt ns1.xlogin00.cz.}"\rm.
      \item Přidejte záznam PTR, který mapuje IP adresu vašeho počítače na doménové jméno autoritativního serveru {\tt ns1.xlogin00.cz.}
      \item Přidejte další záznam PTR, který ukazuje na učitelský počítač v~laboratoři. Záznam zadejte ve tvaru:
            \verb|1    IN    PTR    PCUC.xlogin00.cz.|
      \item Přidejte další tři záznamy PTR, které ukazují na vybrané tři počítače v~laboratoři, např. PC01, PC02 a PC03. IP adresy počítačů musí odpovídat názvům, které jsou uvedeny v záznamech A zónového souboru {\tt xlogin.cz}.
      \item V~případě zájmu nakonfigurujte záznamy PTR pro revezní překlad z~adres IPv6.
    \end{enumerate} 

  \item V~souboru {\tt /etc/named.conf} vytvořte novou zónu "10.10.10.in-addr.arpa". Vložte ji téměř na konec souboru před klíčové slovo {\tt include} a doplňte absolutní cestu k příslušnému zónovému souboru.
  
  \item Restartujte DNS server příkazem {\tt systemctl restart named.service}.
    Příkazem {\tt systemctl status named} opět ověřte, zda služba běží. Chyby spjaté s IPv6 můžete ignorovat.

%  \item Upravte IP adresu výchozího DNS serveru tímto postupem:
%  \begin{itemize}
%    \item Na konec souboru {\tt /etc/sysconfig/network-scripts/ifcfg-Wired\char`_connection\char`_1} přidejte následující dva řádky:\\
%          \verb|PEERDNS=no|\\
%          \verb|DNS1=127.0.0.1|
    %    \item Vypněte a znovu zapněte síťové rozhraní {\tt enp2s0} příkazy: {\tt ifdown Wired\char`_connection\char`_1}; {\tt ifup Wired\char`_connection\char`_1}
%  \end{itemize}
  \item V konfiguraci počítače nastavte manuálně adresu serveru DNS na lokální IP adresu {\tt 127.0.0.1}. Vypněte a zapněte síťové rozhraní. Ověřte, že došlo ke změně výchozího serveru DNS výpisem souboru {\tt /etc/resolv.conf}.
  \item V~programu Wireshark začněte zachytávat  komunikaci DNS na rozhraní {\tt loopback}.
  \item V~terminálu zadejte {\tt ping PCUC.xlogin00.cz} a ověřte, zda došlo ke správnému překladu doméno\-vého jména na IP adresu. Prohlédněte si jednotlivé části dotazu a odpovědi DNS zachycené programem Wireshark.
  \item Otestujte funkčnost dalších záznamů DNS:\\
    {\tt ping server.xlogin00.cz},\\
    {\tt ping PC01.xlogin00.cz},\\
    {\tt nslookup -type=soa xlogin00.cz},\\
    {\tt dig xlogin00.cz},\\
    {\tt dig xlogin00.cz NS}. 
  \item Do protokolu zapište výstupy následujících příkazu:\\
    {\tt nslookup 10.10.10.1},\\
    {\tt dig PCUC.xlogin00.cz},\\
    {\tt dig -x 10.10.10.1xx}, kde {\tt xx} je číslo vašeho počítače.
  \item Demonstrujte funkčnost vašeho řešení cvičícímu. Ukažte výstupy z následujících dotazů: \\
    {\tt nslookup PCUC.xlogin00.cz}, \\
    {\tt nslookup -type=SOA xlogin.cz}, \\
    {\tt nslookup -type=NS xlogin.cz}, \\
    {\tt nslookup 10.10.10.1xx}. 
  \item Ve dvou instancích Wiresharku začněte zachytávat DNS komunikaci na rozhraních {\tt enp2s0} a {\tt Loopback: lo}. Případně můžete použít jedinou instanci Wiresharku, ale musíte zachytávat na obou rozhraních zároveň \footnote{Příkaz \texttt{Ctrl + left\_click} k výběru více rozhraní v programu Wireshark.}. Zachycený provoz seřaďte podle časových razítek.
  
  \item V~terminálu zadejte rezoluci na doménové jméno, které váš server DNS nezná, např. {\tt dig www. google.com}. Pozor na záznamy uložené DNS cache.
  
  \item Zastavte zachytávání provozu DNS v programu Wireshark.
  \item Analyzujte zachycený provoz DNS. Zjistěte časovou posloupnost dotazů a odpovědí. Do sekvenčního diagramu v  protokolu vyplňte informace, ke kterým serverům DNS dané dotazy směrovaly. Zamyslete se nad tím, který dotaz byl iterativní a který rekurzivní. 
\end{enumerate}

\section{Ukončení práce v~laboratoři}
\begin{itemize}
  \item Jako uživatel {\tt root} spuťte dávkou {\tt /root/isa3/clean}, která vyčistí vytvořenou konfiguraci a vypne počítač.
\end{itemize}

\end{document}

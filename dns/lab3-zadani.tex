\section*{Cíle laboratoře}
\begin{itemize}
  \item Seznámení se systémem a protokolem DNS a databází Whois.
  \item Konfigurace a spuštění serveru DNS obsluhujícího vlastní doménu.
\end{itemize}

\section*{Základní instrukce}
\begin{itemize}
  \item Přihlaste se do OS CentOS (F3), user/password: {\tt user}/{\tt user4lab}.
  \item Otevřete si příkazovou řádku pro uživatele {\tt user}.
  \item Otevřete si příkazovou řádku pro uživatele {\tt root} příkazem {\tt su}
    (switch user).
  \item Pro editaci konfiguračních souborů použijte libovolný editor (např.
    nano, vim, gedit).
  \item {\bf Odpovědi pište do protokolu.}
\end{itemize}

\section*{Úkoly}

\section{Seznámení s~DNS}
\begin{enumerate}
    \item Spusťte program Wireshark (vždy jako \texttt{root} z~příkazové řádky příkazem \texttt{wireshark \&}) a začněte zachytávat komunikaci DNS na rozhraní, pomocí kterého jste připojeni k~Internetu (\texttt{enp2s0}).
    \item Otevřete terminál a pomocí příkazu \texttt{nslookup -type=a www.vutbr.cz} zjistěte, na jakou IPv4 adresu se překládá doménové jméno \texttt{www.vutbr.cz}. Zjištěnou IPv4 adresu zapište do protokolu ke cvičení.
    \item Zachycený dotaz DNS a odpověď si prohlédněte v~programu Wireshark.
    \item Příkazem \texttt{nslookup -type=soa vutbr.cz} si zobrazte záznam SOA naší univerzity.
    \item Záznam SOA si dobře prohlédněte, později budete sami vytvářet záznam SOA pro vlastní doménu.
    \item V~programu Wireshark spusťte zachytávání komunikace DNS znovu. Dříve zachycený provoz můžete zahodit.
    \item Pomocí příkazu \texttt{nslookup -type=<X> <doména>} zjistěte autoritativní servery DNS pro doménu \texttt{vutbr.cz}. Hodnotu \texttt{<X>} nahraďte typem záznamu DNS pro zjištění autoritativních serverů. Výsledek dotazu zapiště do protokolu ke cvičení.
    \item Zastavte zachytávání komunikace v~programu Wireshark.
    \item V~zachyceném provozu prozkoumejte pakety obsahující komunikaci poslední vámi provedené rezoluce DNS.
    \item Zjistěte, zda se jedná o rekurzivní či iterativní dotaz DNS, viz políčko \texttt{Flags} v dotazu DNS. Zapište odpověď do protokolu.
    \item Zjistěte a zapište do protokolu, na jakou IP adresu směřoval paket s dotazem DNS.
    \item Zobrazte záznamy \texttt{MX} pro doménu \texttt{fit.vutbr.cz}. Jaký je primární e-mailový server pro doménu \texttt{fit.vutbr.cz}? Odpověď zapiště do protokolu.
\end{enumerate}

\section{Seznámení s~Whois}
\begin{enumerate}
    \item Zadejte do terminálu příkaz \texttt{whois vutbr.cz}. Prohlédněte si zobrazené informace. V jakém roce byla doména registrována? Zapište odpověď do protokolu.
    \item Zjistěte, jaká je veřejná IP adresa vašeho počítače. Otevřte si například v prohlížeči webovou stránku \url{www.fit.vut.cz}. Vaši IP adresu můžete najít na stránce vpravo dole. Nalezenou adresu zapiště do protokolu. Následně zjistěte v~terminálu pomocí příkazu \texttt{ip address show} IP adresy vašeho počítače (rozhraní \texttt{enp2s0}). Nalezené adresy zapište do protokolu. Vysvětlete, proč se vaše veřejná adresa neshoduje s adresou na rozhraní. 
    \item Pomocí služby {\tt  whois}, příkaz \texttt{whois <vase-verejna-IP-adresa>} zjistěte, do kterého rozsahu vaše veřejná IP adresa patří a komu je tento rozsah přidělen. Obě informace zapište do protokolu.
\end{enumerate}

%\section{Blokování vybraných domén}
%\begin{enumerate}
%	\item Do souboru \texttt{/etc/hosts} přidejte následující řádek:\\
%    \verb|0.0.0.0    www.facebook.com|
%    \item Ve webovém prohlížeči zadejte do adresního řádku \url{www.facebook.com}. Podařilo se Vám stránku zobrazit? Pokud ano, restartujte webový prohlížeč, aby se vymazala DNS mezipaměť webového prohlížeče, a zkuste znovu.
%    \item Zamyslete se nad slabinami tohoto řešení omezení přístupu na webové stránky. Jak lze toto řešení (jednoduše) obejít?
%\end{enumerate}

\section{Konfigurace vlastního DNS serveru}
\begin{enumerate}
  \item Prostudujte si začátek kapitoly 9.2.3 z~manuálu k~laboratořím.
  \item Zvolte si jméno vlastní doménu ve tvaru {\tt xlogin00.cz}, například {\tt xnovak01.cz}. Název domény, tj. řetězec {\tt xlogin00.cz}, bude využívat v názvech souborů a v konfiguraci, viz níže.

\item Pro svou doménu vytvořte zónový soubor dle následujících instrukcí:
    \begin{itemize}
      \item Soubor {\tt /root/isa3/template.dns.zone} je ukázkový zónový soubor. Zkopírujte tento soubor do složky {\tt /var/named} pod novým jménem "xlogin00.cz".
      \item V~nově vytvořeném souboru {\tt /var/named/xlogin00.cz} upravte záznam SOA pro svou doménu {\tt xlogin00.cz.}. Jméno primárního serveru DNS bude {\tt ns1.xlogin00.cz.} E-mail správce nastavte na hodnotu {\tt admin.xlogin00.cz.} Pozor, v~e-mailové adrese se znak "@" nahrazuje znakem ".".
      \item V záznamu SOA aktualizujte sériové číslo, aby odpovídalo dnešnímu datu ve tvaru {\tt yyyymmdd}.
      \item Vytvořte záznam NS pro svou doménu. Záznam ukazuje na váš server {\tt ns1.xlogin00.cz.}
      \item Pro server {\tt ns1.xlogin00.cz.} vytvořte záznam A, který se odkazuje na IP adresu Vašeho počítače, tj. adresu na rozhraní {\tt enp2s0}. 
      \item Přidejte další záznam A, který ukazuje na učitelský počítač v~laboratoři. Záznam zadejte v~tomto tvaru: \verb|PCUC    IN    A    10.10.10.1|
      \item Přidejte další záznamy A, které ukazují na tři libovolné počítače v~laboratoři, např. PC01, PC02 a PC03.
      \item Přidejte záznam CNAME, který vytvoří alias s názvem {\tt server} pro počítač {\tt ns1.xlogin00.cz.}
      \item V~případě zájmu nakonfigurujte pro doménu překlad na adresy IPv6 (záznamy AAAA).
    \end{itemize}
  
  \item Nyní upravte konfigurační soubor {\tt /etc/named.conf}. Při úpravách se můžete inspirovat ukázkovým konfiguračním souborem {\tt /root/isa3/named.conf.local}. Proveďte následující úpravy:
    \begin{itemize}
      \item Vytvořte novou zónu pro vaši doménu {\tt xlogin.cz}. Zónu vložte téměř na konec souboru před klíčové slovo {\tt include}.
      \item Doplňte absolutní cestu k~vašemu zónovému souboru \texttt{xlogin00.cz}.
    \end{itemize}
  \item Zkuste spustit server DNS příkazem {\tt systemctl start named.service}. Příkazem {\tt systemctl status named} ověříte, zda služba správně běží. Chyby související s IPv6 můžete ignorovat.
  
  \item Po úspěšném spuštění serveru DNS nakonfigurujte reverzní překlad pro vaši doménu. Opět začněte vytvoření zónového souboru. 
    
  \item Zónový soubor pro reverzní doménu {\tt 10.10.10.in-addr.arpa.} vytvořte následujícím způsobem:
    \begin{itemize}
      \item Jako šablonu zónového souboru pro reverzní překlad použijte soubor {\tt /root/isa3/temp\-late.dns.zone}.
            Zkopírujte tento soubor do složky {\tt /var/named} pod názvem "10.10.10".
      \item V~nově vytvořeném souboru {\tt /var/named/10.10.10} upravte záznam SOA pro doménu {\tt 10.10.10.in-addr.arpa.} Autoritativní server bude opět {\tt ns1.xlogin00.cz.} E-mail na správce bude {\tt admin.xlogin00.cz.}
      \item V záznamu SOA aktualizujte sériové číslo, aby odpovídalo dnešnímu datu.
      \item Vytvořte záznam NS, který ukazuje na autoritativní server {\tt ns1.xlogin00.cz.}
      \item Přidejte záznam PTR, který mapuje IP adresu vašeho počítače na doménové jméno autoritativního serveru {\tt ns1.xlogin00.cz.}
      \item Přidejte další záznam PTR, který ukazuje na učitelský počítač v~laboratoři. Záznam zadejte ve tvaru:
            \verb|1    IN    PTR    PCUC.xlogin00.cz.|
      \item Přidejte další tři záznamy PTR, které ukazují na vybrané tři počítače v~laboratoři, např. PC01, PC02 a PC03. IP adresy počítačů musí odpovídat názvům, které jsou uvedeny v záznamech A zónového souboru {\tt xlogin.cz}.
      \item V~případě zájmu nakonfigurujte záznamy PTR pro revezní překlad z~adres IPv6.
    \end{itemize} 

  \item V~souboru {\tt /etc/named.conf} vytvořte novou zónu "10.10.10.in-addr.arpa". Vložte ji téměř na konec souboru před klíčové slovo {\tt include} a doplňte absolutní cestu k příslušnému zónovému souboru.
  
  \item Restartujte DNS server příkazem {\tt systemctl restart named.service}.
    Příkazem {\tt systemctl status named} opět ověřte, zda služba běží. Chyby spjaté s IPv6 můžete ignorovat.

%  \item Upravte IP adresu výchozího DNS serveru tímto postupem:
%  \begin{itemize}
%    \item Na konec souboru {\tt /etc/sysconfig/network-scripts/ifcfg-Wired\char`_connection\char`_1} přidejte následující dva řádky:\\
%          \verb|PEERDNS=no|\\
%          \verb|DNS1=127.0.0.1|
    %    \item Vypněte a znovu zapněte síťové rozhraní {\tt enp2s0} příkazy: {\tt ifdown Wired\char`_connection\char`_1}; {\tt ifup Wired\char`_connection\char`_1}
%  \end{itemize}
  \item V konfiguraci počítače nastavte manuálně adresu serveru DNS na lokální IP adresu {\tt 127.0.0.1}. Vypněte a zapněte síťové rozhraní. Ověřte, že došlo ke změně výchozího serveru DNS v~souboru {\tt /etc/resolv.conf}.
  \item V~programu Wireshark začněte zachytávat  komunikaci DNS na rozhraní {\tt loopback}.
  \item V~terminálu zadejte {\tt ping PCUC.xlogin00.cz} a ověřte, zda došlo ke správnému překladu doménového jména na IP adresu. Prohlédněte si zachycenou komunikaci DNS ve Wiresharku. Prohlédněte si jednotlivé části dotazu a příslušnou odpovědi.
  \item Otestujte funkčnost dalších záznamů DNS -- například:  {\tt ping server.xlogin00.cz},\\
    {\tt ping PC01.xlogin00.cz}, {\tt nslookup -type=soa xlogin00.cz}, {\tt dig xlogin00.cz} či \\
    {\tt dig xlogin00.cz NS}. 
  \item Do protokolu zapište výstup příkazu {\tt nslookup 10.10.10.1}.
  \item Dále zapište do protokolu výstup příkazů {\tt dig PCUC.xlogin00.cz} a {\tt dig -x 10.10.10.1xx}, kde {\tt xx} je číslo vašeho počítače.
  \item Demonstrujte funkčnost vašeho řešení cvičícímu. Ukažte výstupy z následujících dotazů: \\
    {\tt nslookup PCUC.xlogin00.cz}, \\
    {\tt nslookup -type=SOA xlogin.cz}, \\
    {\tt nslookup -type=NS xlogin.cz} a \\
    {\tt nslookup 10.10.10.1xx}. 
  \item Ve dvou instancích Wiresharku začněte zachytávat DNS komunikaci na rozhraních {\tt enp2s0} a {\tt Loopback: lo}. Případně můžete použít jedinou instanci Wiresharku, ale musíte zachytávat na obou rozhraních zároveň. (\texttt{Ctrl + left\_click} k výběru více rozhraní). V takovém případě nezapomeňte zachycený provoz seřadit podle časových razítek.
  
  \item V~terminálu zadejte {\tt dig www.google.com} či zvolte jiné doménové jméno, které váš server DNS nezná. Pozor na záznamy uložené DNS cache!
  
  \item Zastavte zachytávání provozu DNS v programu Wireshark a analyzujte zachycený provoz. Zjistěte časovou posloupnost dotazů a odpovědí, například pomocí časových razítek v sekci \texttt{frame}. Do sekvenčního diagramu v  protokolu vyplňte informace, ke kterým serverům DNS dané dotazy směřovaly. Zamyslete se nad tím, který dotaz byl iterativní a který rekurzivní. 
\end{enumerate}

\section{Ukončení práce v~laboratoři}
\begin{itemize}
  \item Počítač vypněte dávkou {\tt /root/isa3/clean}.
\end{itemize}


\section*{Cíle laboratoře}
\begin{itemize}
  \item Seznámení se systémem a protokolem DNS a s~databází Whois.
  \item Konfigurace a spuštění DNS serveru obsluhujícího vlastní doménu.
\end{itemize}

\section*{Základní instrukce}
\begin{itemize}
  \item Přihlaste se do OS CentOS (F3), user/password: {\tt user}/{\tt user4lab}.
  \item Otevřete si příkazovou řádku pro uživatele {\tt user}.
  \item Otevřete si příkazovou řádku pro uživatele {\tt root} příkazem {\tt su}
    (switch user).
  \item Pro editaci konfiguračních souborů použijte libovolný editor (např.
    nano, vim, gedit).
  \item {\bf Odpovědi pište do protokolu.}
\end{itemize}

\section*{Úkoly}

\section{Seznámení s~DNS}
\begin{enumerate}
    \item Spusťte program Wireshark (vždy jako \texttt{root} z~příkazové řádky příkazem \texttt{wireshark \&}) a začněte zachytávat DNS komunikaci na rozhraní, pomocí kterého jste připojeni k~Internetu (\texttt{enp2s0}).
    \item Otevřete terminál a pomocí příkazu \texttt{nslookup -type=a www.vutbr.cz} zjistěte, na jakou IPv4 adresu se překládá doménové jméno \texttt{www.vutbr.cz}. Zjištěnou IPv4 adresu zapište do protokolu ke cvičení.
    \item Zachycený DNS dotaz a odpověď si prohlédněte v~programu Wireshark.
    \item Příkazem \texttt{nslookup -type=soa vutbr.cz} si zobrazte SOA záznam naší Alma mater\footnote{\textbf{Alma mater} (latinsky \textbf{matka živitelka}) je původně antické označení pro bohyni matku. Ve středověké poezii se spojení někdy užívalo i pro Pannu Marii jako \emph{Matku Boží} (např. v~hymnu \emph{Alma Redemptoris Mater}). Nejstarší evropská univerzita v~Boloni užívá motto \emph{Alma mater studiorum}. Odtud dnes spojení \textbf{alma mater} metaforicky označuje univerzitu nebo vysokou školu, na které student získal své vzdělání. \emph{Zdroj: Ottův slovník naučný}}.
    \item SOA záznam si dobře prohlédněte (později budete sami vytvářet SOA záznam pro vlastní doménu).
    \item V~programu Wireshark spusťte zachytávání DNS komunikace znovu (dříve zachycený provoz můžete zahodit).
    \item Pomocí příkazu \texttt{nslookup -type=<X> <doména>} zjistěte autoritativní DNS servery pro doménu \texttt{vutbr.cz}, kde za \texttt{<X>} doplňte vhodný typ DNS záznamu pro zjištění autoritativních serverů a za \texttt{<doména>} doplňte vhodnou doménu. Nenechte se zmást tím, že jste dostali neautoritativní odpověď. Nahlédněte do slidů z~přednášky, abyste si ujasnili pojmy (ne)autoritativní odpověď a autoritativní DNS server pro danou doménu.
    \item Zastavte zachytávání komunikace v~programu Wireshark.
    \item V~zachyceném provozu nalezněte pakety obsahující komunikaci poslední Vámi provedené DNS rezoluce a prozkoumejte je.
    \item Kolik paketů souvisejících s~Vaší DNS rezolucí bylo zachyceno?
    \item Byl proveden rekurzivní nebo iterativní DNS dotaz? Podívejte se do \texttt{Flags} v~DNS dotazu a odpovědi.
    \item Na jakou IP adresu směřoval paket s~DNS dotazem? Kde je počítač s~touto IP adresou umístěný?
    \item Zobrazte \texttt{MX} záznamy pro doménu \texttt{fit.vutbr.cz}. Jaký je primární e-mailový server pro doménu \texttt{fit.vutbr.cz}?
\end{enumerate}

\section{Seznámení s~Whois}
\begin{enumerate}
    \item Zadejte do Terminálu příkaz \texttt{whois vutbr.cz}. Prohlédněte si zobrazené informace. Od jakého roku je doména registrována?
    \item Zjistěte, jaká je veřejná IP adresa Vašeho počítače - například v patičce webu \url{https://www.fit.vutbr.cz/}. Následně si v~Terminálu pomocí příkazu \texttt{ip addres show} zobrazte IP adresy na rozhraních Vašeho počítače. Proč ani na jednom rozhraní nevidíte svoji veřejnou IP adresu? (nápověda: NAT)
    \item Informace o své veřejné IP adrese zjistěte příkazem \texttt{whois <vase-verejna-IP-adresa>}. Do jakého rozsahu IP adresa patří a kdo má tento rozsah IP adres přidělen?
\end{enumerate}

\section{Blokování vybraných domén}
\begin{enumerate}
	\item Do souboru \texttt{/etc/hosts} přidejte následující řádek:\\
    \verb|0.0.0.0    www.facebook.com|
    \item Ve webovém prohlížeči zadejte do adresního řádku \url{www.facebook.com}. Podařilo se Vám stránku zobrazit? Pokud ano, restartujte webový prohlížeč, aby se vymazala DNS mezipaměť webového prohlížeče, a zkuste znovu.
    \item Zamyslete se nad slabinami tohoto řešení omezení přístupu na webové stránky. Jak lze toto řešení (jednoduše) obejít?
\end{enumerate}

\section{Konfigurace vlastního DNS serveru}
\begin{enumerate}
  \item Prostudujte si začátek kapitoly 9.2.3 z~manuálu k~laboratořím.
  \item Zvolte si vlastní doménu, kterou bude Váš budoucí DNS server spravovat. Například {\tt xlogin00.cz.} V zadání bude Vaše vybraná doména označována jako {\tt xlogin00.cz}. Vy ale řetězec {\tt xlogin00.cz} v názvech souborů a v konfiguraci nahrazujte svojí vlastní doménou.
  \item Nejdříve upravte konfigurační soubor {\tt /etc/named.conf}. Při úpravách se můžete inspirovat ukázkovým konfiguračním souborem {\tt /root/isa3/named.conf}. Proveďte následující úpravy:
    \begin{itemize}
      \item Vytvořte novou dopřednou zónu (téměř na konci souboru před klíčovým slovem {\tt include}) pro Vaši doménu {\tt xlogin00.cz}.
            \item Cestu k~zónovému souboru nastavíte později~--~až po vytvoření tohoto souboru.
    \end{itemize}
  
  \item Nyní je potřeba pro nově registrovanou zónu vytvořit zónový soubor. V~něm používejte \textbf{FQDN}! (Tedy: Doménová jména musí mít na konci tečku.)
  \item Zónový soubor pro doménu {\tt xlogin00.cz} vytvořte dle následujících instrukcí:
  
    \begin{itemize}
      \item Soubor {\tt /root/isa3/cz.vutbr.fit.netlab} je ukázkový zónový soubor.\\
            Zkopírujte tento soubor do složky {\tt /var/named} pod novým jménem {\tt cz.xlogin00}.
      \item V~nově vytvořeném souboru {\tt /var/named/cz.xlogin00} upravte SOA záznam domény\\ {\tt xlogin00.cz.} Autoritativní server bude {\tt ns1.xlogin00.cz.}
            Email správce bude\\ {\tt admin.xlogin00.cz.} (nelekněte se, že se v~e-mailové adrese místo znaku '{\tt @}' používá znak '{\tt .}').
      \item V SOA záznamu aktualizujte sériové číslo, aby odpovídalo dnešnímu datumu ve tvaru {\tt yyyymmdd}.
      \item Upravte NS záznam, aby ukazoval na autoritativní server {\tt ns1.xlogin00.cz.}
      \item Pro autoritativní server {\tt ns1.xlogin00.cz.} vytvořte A~záznam, který bude ukazovat na IP adresu Vašeho počítače (na rozhraní {\tt enp2s0}).
            Uvědomte si, že nyní jste pomocí SOA, NS a A~záznamu nastavili, že Váš počítač je tím autoritativním DNS serverem pro doménu {\tt xlogin00.cz} (tj. Váš počítač spravuje zónový soubor domény).
      \item Přidejte další A~záznam, který bude ukazovat na učitelský počítač v~laboratoři. Záznam zadejte v~tomto tvaru:
            \verb|PCUC    IN    A    10.10.10.1|
      \item Přidejte další A~záznamy, které budou ukazovat na tři libovolné počítače v~laboratoři (např. PC01, PC02 a PC03).
      \item Přidejte záznam typu CNAME pro jméno {\tt server} ukazující na {\tt ns1.xlogin00.cz.}
      \item V~případě zájmu nakonfigurujte pro doménu překlad na adresy IPv6 (záznamy AAAA).
      \item Nepotřebné záznamy smažte!
    \end{itemize}

  \item Nezapomeňte nyní zaregistrovat zónový soubor v~souboru {\tt /etc/named.conf} (doplnit cestu k~zónovému souboru).
  \item Zkuste spustit DNS server příkazem {\tt systemctl start named.service}.
    Příkazem {\tt systemctl status named} ověřte, zda byla služba správně spuštěna.
  \item Po úspěšném spuštění DNS serveru nakonfigurujte ještě reverzní překlad pro Vaši doménu.
  \item Vytvořte v~souboru {\tt /etc/named.conf} novou zónu (téměř na konci souboru před klíčovým slovem {\tt include}) pro reverzní překlad ({\tt 10.10.10.in-addr.arpa}).
  \item Zónový soubor pro reverzní překlad vytvořte dle následujících instrukcí:
  
    \begin{itemize}
      \item Jako šablonu zónového souboru pro reverzní překlad využijte soubor {\tt /root/isa3/10.10.10}.
            Zkopírujte tento soubor do složky {\tt /var/named} pod stejným jménem.
      \item V~nově vytvořeném souboru {\tt /var/named/10.10.10} upravte SOA záznam domény\\ {\tt 10.10.10.in-addr.arpa.} Autoritativní server bude opět {\tt ns1.xlogin00.cz.}
            Email správce bude také opět {\tt admin.xlogin00.cz.}
      \item V SOA záznamu aktualizujte sériové číslo, aby odpovídalo dnešnímu datumu ve tvaru {\tt yyyymmdd}.
      \item Vytvořte NS záznam, který bude ukazovat na autoritativní server {\tt ns1.xlogin00.cz.}
      \item Přidejte PTR záznam, který bude mapovat IP adresu Vašeho počítače (na rozhraní {\tt enp2s0}) na doménové jméno autoritativního serveru {\tt ns1.xlogin00.cz.}
      \item Přidejte další PTR záznam, který bude ukazovat na učitelský počítač v~laboratoři. Záznam zadejte v~tomto tvaru:
            \verb|1    IN    PTR    PCUC.xlogin00.cz.|
      \item Přidejte další tři PTR~záznamy, které budou ukazovat na vybrané tři počítače v~laboratoři (např. PC01, PC02 a PC03).
      \item V~případě zájmu nakonfigurujte pro doménu překlad z~adres IPv6 (záznamy PTR).
      \item Nepotřebné záznamy smažte!
    \end{itemize} 
    
  \item Nezapomeňte nyní ještě zaregistrovat zónový soubor pro reverzní překlad v~souboru {\tt /etc/named.conf} (doplnit cestu k~zónovému souboru).
  \item Restartujte DNS server příkazem {\tt systemctl restart named.service}.
    Příkazem {\tt systemctl status named} opět ověřte, zda byla služba správně spuštěna.

  \item Upravte IP adresu výchozího DNS serveru tímto postupem:
  \begin{itemize}
    \item Na konec souboru {\tt /etc/sysconfig/network-scripts/ifcfg-Wired\char`_connection\char`_1} přidejte následující dva řádky:\\
          \verb|PEERDNS=no|\\
          \verb|DNS1=127.0.0.1|
    \item Vypněte a znovu zapněte síťové rozhraní {\tt enp2s0} příkazy: {\tt ifdown enp2s0}; {\tt ifup enp2s0}
    \item Ověřte, že došlo ke změně výchozího DNS serveru v~souboru {\tt /etc/resolv.conf} a že nyní je výchozím DNS serverem Váš počítač.
  \end{itemize}
  \item V~programu Wireshark začněte zachytávat DNS komunikaci na rozhraní {\tt Loopback: lo}.
  \item V~Terminálu zadejte {\tt ping PCUC.xlogin00.cz}. Došlo k~přeložení doménového jména na IP adresu? Pokud ano, gratuluji!
  \item Prohlédněte si zachycenou DNS komunikaci ve Wiresharku. Nalezněte DNS dotaz a DNS odpověď provedených při {\tt pingu}.
  \item Otestujte funkčnost dalších DNS záznamů -- například:\\
        {\tt ping server.xlogin00.cz}, {\tt ping PC01.xlogin00.cz}, {\tt nslookup -type=soa xlogin00.cz}, {\tt dig xlogin00.cz}, {\tt dig xlogin00.cz NS}, {\tt nslookup 10.10.10.1}, {\tt nslookup 10.10.10.101}
  \item V~Wiresharku začněte zachytávat DNS komunikaci na rozhraních {\tt enp2s0} a {\tt Loopback: lo}.
  \item V~Terminálu zadejte {\tt ping www.google.com} (nebo zvolte jiné doménové jméno, které Váš vlastní DNS server nezná).
  \item Zastavte zachytávání DNS provozu v programu Wireshark a analyzujte, co se stalo, když jste potřebovali přeložit doménové jméno, které Váš DNS server nezná. Koho se zeptal? Zejména si povšimněte, kdy byl DNS dotaz rekurzivní a kdy iterativní.
  \item {\bf Pochlubte se svými výsledky a zjištěním cvičícímu}.
\end{enumerate}




\section{Ukončení práce v~laboratoři}
\begin{itemize}
  \item Počítač vypněte dávkou {\tt /root/isa3/clean}.
\end{itemize}

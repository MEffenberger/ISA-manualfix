\section*{Cíle laboratorního cvičení}
\begin{itemize}
  \item Seznámit se s DNS a databází Whois.
  \item Prozkoumat data přenášená v protokolu DNS pomocí programu Wireshark.
  \item Vyzkoušet zabezpečení DNS over HTTPS.
  \item Rozšifrovat zachycenou komunikaci z webového prohlížeče Mozilla Firefox.
  \item Nastavit šifrovanou komunikaci DNS over TLS.
  \item Nastavit šifrovanou komunikaci DNS přes DNS over TLS a zároveň filtrace DNS reklamních a malware domén.
\end{itemize}

\section*{Pokyny}
\begin{itemize}
  \item Přihlaste se do OS CentOS (\texttt{F3} - při bootování) jako uživatel \texttt{user} (heslo: \texttt{user4lab}).
  \item Otevřete si příkazovou řádku pro uživatele {\tt user}.
  \item Otevřete si příkazovou řádku pro uživatele {\tt root} příkazem {\tt su} (switch user). Heslo: \texttt{root4lab}
  \item Pro editaci konfiguračních souborů použijte libovolný editor (např. \texttt{nano}, \texttt{vim}, \texttt{gedit}).
  \item Odpovědi pište do \textbf{protokolu} ke cvičení.
\end{itemize}

\section{Seznámení s DNS}
\begin{enumerate}
    \item Otevřete ve webovém prohlížeči (např. Mozilla Firefox) online nástroj pro vyhledávání DNS záznamů \url{https://www.nslookup.io/}.
    \item Zadejte do vyhledávacího pole doménu \texttt{vutbr.cz}. Prohlédněte si zobrazené DNS záznamy k této doméně. Pokud některému ze zobrazených DNS záznamů nerozumíte, nechte si jeho význam vysvětlit od spolužáka nebo cvičícího. Pochopení významu základních DNS záznamů je klíčové pro celé cvičení.
    \item Do protokolu ze cvičení zapište A a MX záznamy domény \texttt{vutbr.cz} v korektním formátu (dle RFC 1034 a RFC 1035). TTL (time-to-live) můžete zvolit libovolnou reálnou hodnotu. Jaký je správný formát záznamů si můžete ověřit na příkladech ve slidech z přednášky.
    \item Vyzkoušejte zobrazení DNS záznamů pro další libovolné domény (např. \texttt{www.fit.vutbr.cz} nebo \texttt{seznam.cz}).
    \item Na nové kartě webového prohlížeče otevřete pokročilejší nástroj pro vyhledávání DNS záznamů \url{https://centralops.net/co/}.
    \item Opět zadejte do vyhledávacího pole doménu \texttt{vutbr.cz} a prohlédněte si zobrazené informace k doméně naší Alma mater\footnote{\textbf{Alma mater} (latinsky \textbf{matka živitelka}) je původně antické označení pro bohyni matku. Ve středověké poezii se spojení někdy užívalo i pro Pannu Marii jako \emph{Matku Boží} (např. v hymnu \emph{Alma Redemptoris Mater}). Nejstarší evropská univerzita v Boloni užívá motto \emph{Alma mater studiorum}. Odtud dnes spojení \textbf{alma mater} metaforicky označuje univerzitu nebo vysokou školu, na které student získal své vzdělání. \emph{Zdroj: Ottův slovník naučný}}. Povšimněte si zejména PTR záznamu. K čemu tento záznam slouží?
\end{enumerate}

\section{Seznámení s Whois}
\begin{enumerate}
    \item Otevřete ve webovém prohlížeči online nástroj pro prohledávání databáze Whois\\ \url{https://www.nic.cz/whois/}. Btw. Odkud znáte \emph{CZ.NIC}? Co toto sdružení zajišťuje?
    \item Zadejte do vyhledávacího pole doménu \texttt{vutbr.cz}. Prohlédněte si zobrazené informace. Do protokolu zapište, od jakého roku je doména registrována.
    \item Zjistěte, jaká je veřejná IP adresa Vašeho počítače - například pomocí webového nástroje\\ \url{https://www.whatismyip.com/}. Následně si v Terminálu pomocí příkazu \texttt{ifconfig} zobrazte IP adresy na rozhraních Vašeho počítače. Proč ani na jednom rozhranní nevidíte svoji veřejnou IP adresu? (nápověda: NAT) Požádejte cvičícího, ať vám ukáže, jakou IP adresu má na rozhraní \texttt{re0} nastaven učitelský počítač. Je to stejná IP adresa jako Vaše veřejná?
    \item Na nové kartě webového prohlížeče otevřete nástroj \url{https://lookup.icann.org/} a svoji veřejnou IP adresu zadejte do vyhledávacího pole. Do protokolu zapište, do jakého rozsahu IP adresa patří a kdo má tento rozsah IP adres přidělen. Btw. Odkud znáte \emph{ICANN}? Co tato organizace zajišťuje?
\end{enumerate}

\section{Prozkoumání protokolu DNS}
\begin{enumerate}
    \item Otevřete terminál a pomocí příkazu \texttt{nslookup -type=a vutbr.cz} zjistěte, na jakou IPv4 adresu se překládá doména \texttt{vutbr.cz}. Odpovídá IP adresa předchozímu hledání A záznamu přes webový nástroj?
    \item Spusťte program Wireshark (vždy jako \texttt{root} z příkazové řádky příkazem \texttt{wireshark \&}) a začněte zachytávat komunikaci na rozhraní, pomocí kterého jste připojeni k Internetu (\texttt{enp2s0}).
	\item Vyfiltrujete pouze pakety související s DNS provozem a do odpovědního archu zapište \emph{display filter}, který jste pro to použili.
    \item Otevřete terminál a pomocí příkazu \texttt{nslookup -type=ns vutbr.cz} zjistěte autoritativní DNS servery pro doménu \texttt{vutbr.cz} a zapište je do odpovědního archu. (Nenechte se zmást tím, že jste dostali neautoritativní odpověď. Diskutujte, abyste si ujasnili pojmy (ne)autoritativní odpověď a autoritativní DNS server pro danou doménu.)
    \item Zastavte zachytávání komunikace v programu Wireshark.
    \item V zachyceném provozu nalezněte pakety obsahující komunikaci Vámi provedeného dotazu na doménu \texttt{vutbr.cz} a prozkoumejte je.
	\item Kolik paketů souvisejících s Vaším dotazem na doménu bylo zachyceno? Číslo zapište do odpovědního archu.
	\item Byl proveden rekurzivní nebo iterativní DNS dotaz? Jak jste to zjistili ze zachyceného provozu? Zapište do odpovědního archu.
	\item Na jakou IP adresu směřoval paket s DNS dotazem? Komu náleží tato IP adresa? Pokud netušíte, jakému zařízení IP adresa náleží, zkuste se podívat do souboru \texttt{/etc/resolv.conf}, poté zadejte v Terminálu příkaz \texttt{ip route} a prohlédněte si výpis a nakonec požádejte cvičícího, ať vám ukáže, jakou IP adresu má na rozhraní \texttt{em0} nastavený učitelský počítač. Diskutujte, jak spolu tyto informace souvisí a proč jste na všech třech místech viděli stejnou IP adresu.
\end{enumerate}

\section{Zabezpečení a překlad pomocí DNS over HTTPS}
\begin{enumerate}
    \item Spusťe prohlížeč Firefox.
    \item V prohlížeči přistupte do \texttt{Preferences} pak sescrollujte dolů na položku \texttt{Network Settings} a klikněte na tlačítko \texttt{Settings}. V dialogovém okně najděte položku \texttt{Enable DNS over HTTPS} a zaškrtněte. Provider nastavte na \texttt{Custom} a vyplňte nově vzniklé pole tímto url\\ \texttt{https://odvr.nic.cz/doh}.
    \item Potvrďte změny kliknutím na tlačítko \texttt{OK} a prohlížeč zavřete.
    \item Spusťte program Wireshark jako uživatel \texttt{root} z terminálu příkazem \texttt{wireshark \&} a začněte zachytávat provoz na všech rozhraních.
    \item Pro pozdější rozšifrování HTTPS komunikace z prohlížeče je nutné nastavit proměnnou prostředí\\ \texttt{SSLKEYLOGFILE=<cesta\_k\_souboru>}, na kterou prohlížeče Firefox, Chrome a případně další reagují.
    \item Otevřete terminál a jako uživatel \texttt{user} zadejte\\ \texttt{export SSLKEYLOGFILE=/home/user/Desktop/keylogfile.log}.
    \item Následně ve stejném okně terminálu, ve kterém jste nastavili proměnnou prostředí \texttt{SSLKEYLOGFILE}, spusťte jako uživatel \texttt{user} prohlížeč Firefox příkazem \texttt{firefox \&}.
    \item Přistupte na pár webových stránek, které Vás napadnou. Následně prohlížeč zavřete a tuto akci ještě jednou či vícekrát zopakujte, pokaždé ideálně s jinými webovými stránkami. Nakonec prohlížeč zavřete.
    \item Následně zastavte zachytávání v programu Wireshark.
    \item Zachycenou komunikaci uložte do souboru \texttt{cv3-DoH.pcap}, který budete odevzdávat.
	\item Představ si sebe jako útočníka, který zachytil neznámou komunikaci do souboru \texttt{cv3-DoH.pcap} a nemá k ní žádné další informace. Dokázali byste v tuto chvíli zjistit ze zachyceného DNS provozu, jaké domény byly přes prohlížeč Firefox navštíveny? Proč? Odvěď uveďte do odpovědního archu.
    \item V programu Wireshark otevřete \texttt{Edit > Preferences} a zde v levém sloupci rozklikněte \texttt{Protocols} a zde nalezněte položku \texttt{TLS}. Na této kartě je potřeba nastavit \texttt{(Pre)-Master-Secret log filename} na váš keylogfile (\texttt{/home/user/Desktop/keylogfile.log}). Aplikujte změnu kliknutím na \texttt{OK}. Nyní by mělo proběhnout rozšifrování provozu.
    \item Pomocí \emph{display filteru} vyfiltrujte pouze TLS provoz. Do odpovědního archu zapište, jaký \emph{display filter} jste zadali. Pokuste se nalézt pakety protokolu DoH (DNS over HTTPS).
    \item Pokud se Vám to podařilo, podívejte se jak vypadá obsah paketu.
    \begin{itemize}
        \item Pokud se vám nepodařilo nalézt pakety s DoH, ve složce \texttt{/home/user/doh-pcaps/} naleznete \texttt{.zip} soubor po jehož rozbalení objevíte soubor \texttt{doh-decrypted.pcapng}.
		Když ve Wiresharku otevřete tento soubor, měli byste narazit na již dešifrovaný TLS provoz, ve kterém DoH pakety již určitě naleznete.
    \end{itemize}
	\item Vyberte si libovolnou zachycenou DoH odpověď a do odpovědního archu opište jeden řádek z položky \texttt{Answers}.
	\item Jaká je cílová IP adresa pro pakety s DoH dotazy? Jaké doménové jméno patří k této IP adrese? Zapište do odpovědního archu.
    \item Před postupem k další části cvičení nezapomeňte otevřít prohlížeč a na stejném místě v \texttt{Preferences} položku \texttt{Enable DNS over HTTPS} opět vypnout a prohlížeč zavřít.
\end{enumerate}


\section{Zabezpečení a překlad pomocí DNS over TLS}
\label{sec:dot}
\begin{enumerate}
    \item Deaktivujte SELinux jako uživatel \texttt{root} pomocí příkazu \texttt{setenforce 0}.
	\item Zároveň v souboru \texttt{/etc/selinux/config} zkontrolujte a případně upravte řádek s proměnnou \texttt{SELINUX} následovně \texttt{SELINUX=disabled}. NErestartujte v tuto chvíli počítač.
    \item Nainstalujete Unbound DNS caching resolver pomocí příkazu \texttt{yum install unbound -y}.
    \item V souboru \texttt{/etc/unbound/unbound.conf} najděte příslušnou část pro úpravu forward zón (pod řádkem začínajícím \texttt{\# Forward zones} a přidejte následující:
    
\begin{verbatim}
forward-zone:
    name: "."
    forward-ssl-upstream: yes
    # Cloudflare DNS
    forward-addr: 1.1.1.1@853
    forward-addr: 1.0.0.1@853
\end{verbatim}

    \item Jakmile je soubor upraven, uložte jej a restartujte službu pomocí příkazu \texttt{systemctl restart unbound}.
    \item Ověřte zdali služba běží bez problémů pomocí \texttt{systemctl status unbound}.
    \item V souboru \texttt{/etc/resolv.conf} nastavte nameserver na IP adresu \texttt{127.0.0.1}.
    \item Nyní spusťte program Wireshark a spusťte zachytávání na rozhraní, pomocí kterého jste připojeni k Internetu.
    \item Pokuste se vygenerovat nějaký DNS provoz pomocí webového prohlížeče, a to konkrétně přístupem na web \texttt{idnes.cz}.
    \item Vyčkejte než se stránka celá načte a důkladně si ji prohlédněte.
    \item Následně zavřete tab s načtenou stránkou i prohlížeč samotný.
    \item Zastavte zachytávání provozu. Zachycený provoz uložte jako soubor \texttt{cv3-DoT.pcap}, který budete odevzdávat.
    \item V programu Wireshark pomocí \emph{display filteru} vyfiltrujte pouze pakety, které využívají port 853 nad protokolem TCP. Jaký filtr přesně jste použili? Zapište do odpovědního archu.
    \item Následně vyfiltrujte pakety, které obsahují port 53 nad protokolem TCP nebo UDP. Jaký filtr přesně jste použili zde? Opět zapište do odpovědního archu.
	\item Jaká služba běží nad portem 53? Kolik paketů se zdrojovým nebo cílovým portem 53 bylo zachyceno? Odpovězte do odpovědního archu a zamyslete se, proč právě takové číslo.
\end{enumerate}


\section{Blokování reklam}
\begin{enumerate}
	\item Stáhněte pi-hole instalační skript pomocí příkazu\\ \texttt{wget -O basic-install.sh https://install.pi-hole.net}.\footnote{Pihole je populární open-source DNS sink-hole, jehož zdrojové kódy můžete nalézt na adrese\\
	\texttt{https://github.com/pi-hole/pi-hole/}, ostatně taktéž tento instalační skript se stahuje z tohoto oficiálního GitHub repozitáře}
	\item Instalační skript si prohlédněte, zdali neobsahuje žádný škodlivý kus zdrojového kódu.\footnote{Tento krok jsme již provedli a proto ho můžete v tomto případě přeskočit.}
	\item Spusťte instalační skript pi-hole pomocí příkazu \texttt{sudo bash basic-install.sh}.
    \item Instalací pi-hole vás bude provázet dialogové okno, ve kterém bude nutné vybrat následující možnosti:
    \begin{enumerate}
        \item V dialogovém okně budete dotázáni na instalaci PHP. V tomto případě vyberte možnost \texttt{no}.
		\item Potvrďte, že souhlasíte, aby se Vaše zařízení stalo blokátorem reklam.
		\item Potvrďte informaci, že se jedná o software zdarma.
		\item Potvrďte, že rozumíte nutnosti přidělení statické IP adresy v lokální síti.
        \item Při výběru síťového rozhraní nechte zaškrtnuté ethernetové (\texttt{enp0s3}).
        \item U výběru "Upstream DNS Provider" sescrollujte dolů, kde zvolte možnost \texttt{Custom} a následně pro adresu serveru zadejte \texttt{127.0.0.1\#5335} a potvrďte klávesou \texttt{Enter}. Také v dalším okně potvrďte adresu serveru.
        \item Dále při výběrů blocking listů ponechte zaškrtnuté oba dva listy.
        \item Zbytek nastavení ponecháme ve výchozím stavu a jenom vždy odsouhlasíme buď \texttt{yes} nebo \texttt{ok}.
    \end{enumerate}
    \item Na začátek souboru \texttt{/etc/unbound/unbound.conf} pod řádek obsahující \texttt{server:} přidejte následující:
\begin{verbatim}
    interface: 127.0.0.1
    port: 5335
    do-ip4: yes
    do-udp: yes
    do-tcp: yes
    do-ip6: yes
\end{verbatim}
    \item Restartujte službu unbound DNS caching serveru pomocí příkazu \texttt{systemctl restart unbound}.
    \item Pomocí příkazu \texttt{systemctl enable unbound} nastavte povolení služby na systému.
	
    \item Restartujte systém pomocí příkazu \texttt{reboot}.
    \item Přihlašte se do systému a zkontrolujte zdali pi-hole běží v pořádku:
    \begin{enumerate}
        \item Spusťte příkaz \texttt{pihole status}.
        \item Zkontrolujte obsah souboru \texttt{/etc/resolv.conf}, měl by obsahovat záznam pro nameserver ukazující na \texttt{127.0.0.1\#5335}.
    \end{enumerate}
    \item Spusťte prohlížeč Firefox a přistupte znovu na stránku \texttt{idnes.cz}. Jaký rozdíl jste vypozorovali? Zapište do odpovědního archu.
	\item Pořiďte snímek obrazovky s načtenou webovou stránkou \texttt{idnes.cz} tak, aby vypozorovaný rozdíl byl na snímku obrazovky viditelný, a uložte snímek obrazovky jako obrázek s názvem\\ \texttt{cv3-idnes.\{png|jpg|jpeg|...\}} v dostatečně nízkém rozlišením, aby ho bylo možné odevzdat do WIS-u.
\end{enumerate}

\section*{Poznámka ke cvičení}
Použití nasazení nástroje pi-hole lokálně na zařízení není úplně standardní a je v této konfiguraci pouze z demonstračních důvodů. Standardně je tato aplikace určena pro nasazení na samostatném serveru (například Raspberry Pi) běžícím v lokální síti a použití tohoto serveru všemi zařízeními v síti jednotně. Je možné jej pak zkombinovat i s tunelováním DNS přes TLS jako v tomto cvičení (viz úloha \ref{sec:dot}). Pro případné nasazení na serveru v lokální síti bude potřeba pár drobných změn v tomto cvičení.

\section*{Odevzdávané soubory}
Zkontrolujte, zda máte všechny soubory které se budou odevzdávat:
\begin{itemize}
  \item \texttt{protokol.md}
  \item \texttt{cv3-dns.pcap}
  \item \texttt{cv3-DoH.pcap}
  \item \texttt{cv3-DoT.pcap}
  \item \texttt{cv3-idnes.\{png|jpg|jpeg|...\}}
\end{itemize}

\section*{Ukončení práce v laboratoři}
\begin{itemize}
	\item Do WIS-u odevzdejte vyplněný \texttt{protokol.md}, všechny zachycené \texttt{pcap} soubory a snímek obrazovky.
	\item Vypňete virtuální stroj a obnovte jeho snapshot vytvořený na začátku této laboratoře.
\end{itemize}

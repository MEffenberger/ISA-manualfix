%\include{macro_lecture}
\section*{Cíl laboratorního cvičení}
\begin{itemize}
  \item Seznámit se se systémem pro logování událostí Syslog a nástrojem {\tt rsyslog}.
  \item Seznámit se se systémem Netflow pro monitorování toků a nástrojem {\tt nfdump}.
  \item Seznámit se s nástroji pro správu sítě SNMP.
  \item Vyzkoušet si nástroje pro monitorování sítě založené na protokolu ICMP.
\end{itemize}

\section*{Pokyny}
\begin{itemize}
  \item Do zadání nepište, slouží pro další skupiny. Zadání cvičení i šablony konfiguračních souborů lze najít v systému Moodle VUT u předmětu ISA.
    
%  \item Po ukončení práce v laboratoři nezapomeňte na poslední bod, tj. na {\em ukončení práce v laboratoři}.
  
  \item Zapněte si počítač s operačním systémem Linux/CentOS a přihlaste se jako uživatel {\tt user} s heslem {\tt user4lab}.
\end{itemize}

\newpage
%\section{Laboratorní úlohy}

%SYSLOG
\section{Syslog}
\begin{itemize}
  \item Úkol: 
    \begin{itemize}
      \item Seznámit se s protokolem Syslog, který slouží pro přenos
        logovacích zpráv ze spravovaných zařízení. Názvem Syslog se také označuje programové vybavení implementující přenos, třídění a ukládání zpráv na disk.
      \item Pracujte ve dvojicích, kde jedna stanice bude v~roli {\bf serveru} a~druhá v~roli {\bf klienta}.
      \item Pro práci využijte nástroj {\tt rsyslogd}, který bude sloužit jako server i klient.
        K otestování komunikace využijte nástroj {\tt logger}.
      \item Na klientovi následně omezte přeposílání zpráv Syslog pouze na zprávy konkrétního typu.
      \item Pro spuštění a konfiguraci služby Syslog musíte mít práva uživatele {\tt root}.
    \end{itemize}
  \item Příkazy:
    \begin{itemize}
      \item {\tt rsyslogd(8)} -- démon pro Syslog
      \item {\tt rsyslog.conf(5)} -- popis konfigurace rsyslog démona
      \item {\tt logger(1)} -- nástroj pro generování Syslog zpráv
    \end{itemize}
  \item Postup:
    \begin{enumerate}
      \item {\bf Na serveru} povolte komunikaci Syslog na daném portu. Do souboru {\tt /etc/rsyslog.conf} přidejte následující řádky, případně je odkomentujte:
\begin{verbatim}
  module(load="imudp")
  input(type="imudp" port="514")
\end{verbatim}
      \item Na serveru otevřete port 514 ve firewallu pro přijímání komunikace Syslog.
\begin{verbatim}
  firewall-cmd --add-port=514/udp
\end{verbatim}
      \item {\bf Na klientovi} nakonfigurujte démona {\tt rsyslog} tak, aby odesílal veškeré zprávy
        z klienta na serveru pomocí UDP.
        Do souboru {\tt /etc/rsyslog.conf} přidejte na konec konfiguračního souboru následující pravidlo:
\begin{verbatim} 
  *.*  @<doménové_jméno_serveru>:<číslo_portu>
\end{verbatim}
      \begin{itemize}
        \item Doménové jméno serveru zadejte ve formátu \texttt{hXX.netlab.fit.vutbr.cz}, kde \texttt{XX} je číslo vašeho počítače, např. \texttt{h10.netlab.fit.vutbr.cz}.
      \end{itemize}
      \item {\bf Na serveru i klientovi} restartujte proces {\tt rsyslog}: 
\begin{verbatim}
  systemctl restart rsyslog
\end{verbatim} 
      \item Ověřte na straně serveru, zda běží na daném portu proces {\tt rsyslog}, např. pomocí příkazu \verb|netstat -ulnp|.
      \item Pro ověření funkčnosti pošlete {\bf z klienta} testovací zprávu Syslog nástrojem {\tt logger}:
\begin{verbatim} 
  logger -d <obsah_zprávy>
\end{verbatim} 
      \item Na serveru lze zprávu najít na konci souboru {\tt /var/log/messages}, který obsahuje
        systémové události (jméno klienta zadávejte ve tvaru hXX).
\begin{verbatim} 
  tail -f /var/log/messages | grep <doménové_jméno_klienta>
\end{verbatim} 
      \item Na serveru i klientovi sledujte zprávy Syslog pomocí nástroje {\tt wireshark}. Zjistěte, na jakém portu a jakým protokolem jsou zprávy Syslog zasílány. Na klientovi se v novém terminálu odhlaste a přihlaste, čímž vygenerujete zprávy Syslog, které jsou zaslány na server. Typ a hodnotu vygenerovaných zpráv zapište do protokolu.
    \end{enumerate}
\end{itemize}

%NETFLOW
\section{NetFlow}
Seznamte se s měřením provozu pomocí NetFlow. NetFlow slouží pro sběr a přenos statistik o jednotlivých tocích, které komunikují po síti. Záznamy NetFlow, s~nimiž budete během cvičení pracovat, jsou pořízeny ze sítě VUT a~anonymizovány.
\subsection*{Postup}
\begin{enumerate}
  \item Na počítači se v adresáři {\tt /root/isa5/} nachází anonymizovaný soubor s daty NetFlow, viz soubor \texttt{nfcapd.202211242155.anon}. Tento soubor bude vstupem pro program  {\tt nfdump}, který využijte pro prohledávání záznamů NetFlow.
  \item Prostudujte manuálovou stránku nástroje {\tt nfdump}.
    \begin{itemize}
      \item Zjistěte v manuálové stránce, k čemu slouží přepínače {\tt -r, -s, -n, -O, -A}.
      \item V části {\tt EXAMPLES} manuálové stránky se podívejte na příklady použití nástroje {\tt nfdump}.
    \end{itemize}
  \item Dotažte se na TOP 20 IP adres podle počtu přenesených bajtů. 
\begin{verbatim}
  nfdump -r /root/isa5/nfcapd.202211242155.anon -s srcip/bytes -n 20
\end{verbatim}
  \item Zjistěte, kolik dat bylo přeneseno na portech 80 (HTTP) a 443 (HTTPS). Jaký je poměr využití těchto protokolů?
\begin{verbatim}
  nfdump -r /root/isa5/nfcapd.202211242155.anon -s port/bytes 
\end{verbatim}
   \begin{itemize}
     \item Všimněte si rozdílů v podílech podle toků a podle přenesených bajtů. Vypište a porovnejte statistiky portů řazené podle počtu toků.
   \end{itemize}
\end{enumerate}
%\end{itemize}

%SNMP
\section{SNMP}
Protokolem SNMP získejte základní informace o zvoleném stanici. Pro čtení dat SNMP využijte klienta \texttt{snmpwalk}, viz \texttt{man snmpwalk}. Formát použití:

\begin{verbatim}
snmpwalk -v<protokol> -c <community string> <host> <object>
\end{verbatim}

\subsection*{Postup}
\begin{enumerate}
  \item Vyhledejte základní informace o serveru {\tt isa2.netlab.fit.vutbr.cz}.  Použijte příkaz {\tt snmpwalk}\footnote{Pokud příkaz {\tt snmpwalk} není dostupný, doinstalujte balíček \texttt{net-snmp-utils}, např. pomocí příkazu \texttt{yum install net-snmp-utils}.} a SNMP objekt \texttt{system}. Pro připojení použijte privátní IP adresu serveru \texttt{10.10.10.1}.
\begin{verbatim}
  snmpwalk -v2c -c public isa2.netlab.fit.vutbr.cz system
\end{verbatim}
  \item Pomocí protokolu SNMP vypište tabulku ARP (mapování IP adresa na MAC adresy) na serveru  {\tt isa2}. V dané tabulce zjistěte  hodoty IP adresy a MAC adresy počítače vašeho souseda. Pro vyhledání použijte objekt SNMP \texttt{ipNetToMedia}. Přehlednější výpis dat získáte použitím příkazu \texttt{snmptable}, který má stejnou syntax jako \texttt{snmpwalk}.
\begin{verbatim}
  snmptable -v2c -c public isa2.netlab.fit.vutbr.cz ipNetToMedia
\end{verbatim}
\item Zjistěte {\bf název} síťového rozhraní serveru {\tt isa2}, které obsahuje mapování mezi IP adresou a MAC adresou vašeho souseda.

  {\small Nápověda: použijte objekt SNMP \texttt{interface} a příkaz \texttt{snmpwalk}}.
\end{enumerate}

\section{ICMP}
Protokol ICMP slouží pro dohledávání problémů v počítačové síti na vrstvě IP. Standardně ho využívají nástroje \texttt{ping} a \texttt{traceroute}, nebo pokročilejší nástroje jako například \texttt{mtr}. Pomocí těchto nástrojů vyhledejte informace o cestě mezi vaším počítačem a vybraným serverem.

\begin{enumerate}
  \item Zjistěte pomocí nástroje \texttt{ping} průměrnou dobu odezvy serveru \texttt{google.com}.
  \item Jaká je maximální velikost IP datagramu povolena na cestě k serveru\\ \texttt{www.fit.vutbr.cz}? Jak se tato velikost liší pro IPv4 a IPv6?
  \begin{itemize}
    \item Použijte nástroj {\tt ping} s parametrem \texttt{-s} + testovaná velikost datagramu, a s parametrem \texttt{-M do}, který zakáže fragmentaci IP datagramu, tj. nastaví bit \texttt{Don't Fragment}.
\begin{verbatim}
  ping -s 1500 -M do www.fit.vutbr.cz
\end{verbatim}
\item Pro testování maximální velikosti IP datagramu pro IPv6 se přihlašte z terminálu pomocí protokolu SSH na svůj studentský účet na serveru {\tt merlin.fit.vutbr.cz}. Z tohoto serveru otestujte cestu k serveru \texttt{www.fit.vutbr.cz} a určete maximální velikost IPv6 datagramu. 
  \end{itemize}
  \item Zachyťte provoz ICMP nástrojem {\tt wireshark} a zjistěte zapouzdření zpráv ICMP a velikost jednotlivých vrstev.
  \item Pomocí nástroje \texttt{traceroute} zjistěte, přes které {\em autonomní systémy} je směrován provoz z vašeho PC na servery \texttt{google.com} a \texttt{idnes.cz}.
\begin{verbatim}
  traceroute -A google.com
  traceroute -A idnes.cz
\end{verbatim}
  \item Zapište, kteří operátoři provozují dané autonomní systémy. 

  {\small Nápověda: pro řešení využijte nástroj {\tt whois} nebo stránky \texttt{stat.ripe.net}, případně \texttt{bgp.he.net}.}
\end{enumerate}

\section*{Ukončení práce v laboratoři}
\begin{itemize}
  \item Počítač vypněte skriptem {\tt /root/isa5/clean}.
\end{itemize}

\section*{Cíle cvičení}
\begin{itemize}
	\item Seznámení se se základní prací v~OS Linux.
	\item Seznámení se se základními nástroji pro zjišťování konfigurace zařízení.
	\item Analýza síťového provozu pomocí programu Wireshark.
	\item Seznámit se s manuální konfigurací IPv4 a IPv6 na OS Linux.
\end{itemize}

\section*{Pokyny}
\begin{itemize}
\item Do zadání nepište, slouží pro další skupiny. Výsledky zapisujte do protokolu, který na konci cvičení odevzdáte cvičícímu. Cvičící může použít zápisky v záznamovém archu při hodnocení cvičení.
\item Pro práci v laboratoři budeme používat OS Linux, při bootu vyberte volbu F3.
\item Uživatelé a hesla pro přihlášení: \texttt{user - user4lab}, \texttt{root - root4lab}.
\item Přihlaste se jako uživatel \texttt{user}. Veškeré potřebné příkazy následně spouštějte jako \texttt{root}.
\end{itemize}

\subsection*{Příprava laboratoře}
Váš počítač je připojený kabelem přes patch panel
pro kabely EXX (XX je číslo vašeho počítače) vepředu laboratoře. Na PC budete
pracovat s rozhraním {\bf enp2s0}.

\section{Zjišťování konfigurace}
V případě, že se v OS Linux úplně neorientujete, přečtěte si kapitolu \ref{basics} v
laboratorním manuálu --- Základní konfigurace linuxového serveru. V~této části cvičení se
budeme zabývat převážně zjišťováním síťové konfigurace systému. Veškeré potřebné
informace, které budete potřebovat ke splnění této části cvičení, naleznete
v~sekci \ref{basic_ipconfig} laboratorního manuálu --- Konfigurace síťových zařízení. V~případě, že si nejste jistí některým příkazem, neváhejte nahlédnout do manuálové stránky.

\begin{enumerate}
\item Vypište konfiguraci vašeho stroje, konkrétně MAC adresu, IPv4 adresu, masku, síť a broadcastovou adresu ({\tt ip address show}).
\item Zobrazte si záznamy ve~směrovací a ARP tabulce. Zapište IPv4 adresu
  výchozí brány a přiřaďte k ní MAC adresu ({\tt ip route}, {\tt ip neighbour}).
\item Otestujte konektivitu k~výchozí bráně a následně konektivitu do Internetu ({\tt ping}).
\item Vypište implicitní servery DNS ze souboru {\tt /etc/resolv.conf}.
\item Upravte soubor {\tt /etc/hosts} tak, aby po spuštění příkazu \texttt{ping gw}, byl ping proveden vůči IPv4 adrese 10.10.10.1. Zapište, jak jste soubor upravili a jaký záznam jste přidali.
\item Vypište aktivní TCP spojení, vyberte jeden záznam, zapište si ho a popište
  význam jednotlivých položek. Pokud se žádné TCP spojení nezobrazuje, nějaké
    vygenerujte, například pomocí webového prohlížeče ({\tt ss}).
\item Zobrazte systémové události pomocí programu \texttt{journalctl}.
\item Zobrazte pouze události týkající se NetworkManager (\texttt{journalctl}).
\item Pokuste se jako uživatel user spustit Wireshark s pomocí programu \texttt{sudo}. Následně vyhledejte v~logu zprávu, která byla zaznamenána v~případě správně zadaného hesla, ale odepřeného přístupu (\texttt{journalctl}).

\end{enumerate}

\section{Wireshark}
V~této části cvičení se budeme zabývat analýzou a zachytáváním provozu
v~programu Wireshark. Spuštění Wireshark provedete příkazem \texttt{wireshark}
pod uživatelem \texttt{root}. Veškeré potřebné informace, které budete
potřebovat k~této části cvičení, naleznete v~kapitole \ref{wireshark} laboratorního manuálu
--- Analýza síťového provozu programem Wireshark.

\begin{enumerate}
  \item Pomocí programu Wireshark začněte \textbf{zachytávat} pouze HTTP komunikaci na výchozím portu
    (výchozí porty je možné nalézt např. v \texttt{/etc/services}). Zapište
    použitý \emph{capture filter} do odpovědního archu. Spusťte si prohlížeč a
    načtěte stránku \texttt{http://cphoto.fit.vutbr.cz} (po zapnutí zachytávání
    provozu). Zachycený provoz uložte do adresáře \texttt{/tmp} (přípona \texttt{.pcap}).
\item Vypište zdrojovou a cílovou IPv4 adresu a MAC adresu odeslaného a
  přijatého paketu. Zamyslete se nad tím, co vypsané MAC adresy a IPv4 adresy
    identifikují --- nalezené identifikátory porovnejte s identifikátory
    vypsanými v předchozích bodech zadání.
\item Zahajte znovu zachytávání komunikace, nyní bez použití filtru pro HTTP,
  zachytávejte veškerou komunikaci. V~příkazové řádce odstraňte ARP záznamy v systému
    (zadejte příkaz \texttt{ip neighbor flush dev enp2s0}). Vygenerujte ICMP komunikaci. Ve
    Wiresharku zobrazte pouze komunikaci
    protokoly ARP a~ICMP. Analyzujte obsah ARP paketů. Zapište, jaký filtr jste
    použili.
\item Zachyťte pouze HTTP, HTTPS a DNS provoz (na výchozích portech). Ve webovém prohlížeči zkuste otevřít několik stránek na různých URL adresách. Analyzujte obsah a posloupnost DNS paketů a HTTP(S) paketů, zaměřte se na návaznost/závislost paketů.

\item Znovu si otevřete dříve uložený soubor s nešifrovanou komunikaci pomocí
  HTTP, zobrazte si TCP a HTTP stream této komunikace (\emph{Follow TCP stream} a \emph{Follow HTTP stream}).
    Najděte dotaz, který odpovídá stránce zobrazené v prohlížeči.

\item Do odpovědního archu slovně popište význam funkce \emph{Follow TCP stream} a \emph{Follow HTTP stream},
    zamyslete se nad formátem zobrazených dat těchto funkcí
    a rozdílem oproti výchozímu pohledu na data ve formě paketů.
\end{enumerate}

\section{Konfigurace IPv4 a IPv6}

\subsection{Příprava}

Pracujte samostatně nebo spolupracujte se svým spolužákem v okolí, který se také
chystá plnit tento bod zadání. Můžete spolupracovat i ve větších skupinách.
Ze své skupiny vyberte velitele, jehož číslo počítače (pracovního místa)
použijete níže jako konstantu $N$.

Teorie \emph{IPv4 adresování} je popsaná v sekci \ref{adresy_ipv4} laboratorního manuálu --- Adresy v IPv4 síti.
Možnosti manuální konfigurace IP adres jsou v sekci \ref{ipconfig} --- Konfigurace síťování koncových zařízení.

Teorie \emph{IPv6 adresování} je popsaná v sekci \ref{adresy_ipv6} --- Adresy v IPv6 síti.
Možnosti manuální konfigurace IP adres jsou stejné jako pro IPv4.

\begin{enumerate}
    \item Zvolte nejdelší možný prefix IPv4 sítě tak, aby síť obsahovala prostor pro 100 koncových stanic.
    \item Jako adresu sítě použijte \texttt{192.168.N.0}.
    \item Každému členu skupiny přiřaďte jednu IPv4 adresu z vybraného adresního prostoru.
    \item Zvolte si adresu sítě IPv6 vhodnou pro použití v privátních lokálních
        sítích s prefixem délky 64 bitů. Prvních 48 bitů zvolte podle popisu v
        sekci \ref{adresy_ipv6}. Pro vygenerování unikátního Global ID můžete
        použít web \url{https://cd34.com/rfc4193/}, zbývajících 16 bitů (Subnet
        ID) zvolte libovolně. Pokud pracujete ve skupině, použijte
        všichni shodné Global ID i Subnet ID.
      \item Unikátnost vygenerovaného Global ID můžete (částečně)
        zkontrolovat na \url{https://www.sixxs.net/tools/grh/ula/list}.
\end{enumerate}

\subsection{Manuální konfigurace IPv4 a IPv6}

\begin{enumerate}
    \item Na svém počítači
      nakonfigurujte IPv4 a IPv6 adresy zvolené výše. Klikněte na ikonu sítě vpravo nahoře v GUI CentOS,
      vyberte drátové Ethernetové připojení a položku nastavení.
      Klikněte na ozubené kolo u tohoto připojení. Na záložkách IPv4 a IPv6
      manuálně vyplňte adresy a prefixy. Konfiguraci uložte.
    \item Aplikujte manuálně nastavené adresy dočasnou deaktivací rozhraní (vypnout rozhraní) a
      následnou aktivací (zapnout rozhraní) s pomocí menu v pravém horním rohu obrazovky.
    \item Správnou konfiguraci si ověřte příkazem {\tt ping} a {\tt ping6}. Pokud
      spolupracujete ve skupině, ověřte konektivitu ke všem spolupracujícím
      členům skupiny.
\end{enumerate}

\section{Ukončení práce v~laboratoři}
Jakmile máte veškerou práci hotovou, ohlaste se u vyučujícího, který Vám
konfiguraci zkontroluje. Jakmile Vám vyučující práci zkontroluje, spusťte skript {\tt /root/isa1/clean} (jako
\texttt{root}), který smaže Vaši konfiguraci a vypne počítač.

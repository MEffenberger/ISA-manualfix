\documentclass[a4paper,11pt]{article}

\usepackage[utf8]{inputenc}
\usepackage[czech]{babel}
\usepackage[left=2cm,top=3cm,text={17cm,24cm}]{geometry}
\usepackage{graphicx}
\usepackage{listings}
\usepackage{url}

\title{Zabezpečený přenos dat\\
{\bf\large ISA - Laboratorní cvičení č.2}\\
{\bf\large Odpovědní arch}}

\author{Vysoké učení technické v Brně}

\date{\url{https://github.com/nesfit/ISA/tree/master/encrypted_transfers}}

\setlength\parindent{0pt}

\begin{document}

{\let\newpage\relax\maketitle}

Jméno:\\
Login:\\
Skupina cvičení:\\
Datum:

\section{Network Time Protocol (NTP)}
NTP server: \underline{\hspace{4cm}}\hspace{1cm} stratum: \underline{\hspace{1cm}} \\
NTP server: \underline{\hspace{4cm}}\hspace{1cm} stratum: \underline{\hspace{1cm}} \\
NTP server: \underline{\hspace{4cm}}\hspace{1cm} stratum: \underline{\hspace{1cm}} \\
NTP server: \underline{\hspace{4cm}}\hspace{1cm} stratum: \underline{\hspace{1cm}} \\

\section{Vzdálený terminál - Secure Shell (SSH)}
\textbf{2.} Co můžeme ze zachycené komunikace vyčíst? Jsou v této komunikaci vidět zadávané příkazy a jejich výstup?\\

\underline{\hspace{16cm}}
~\\
\textbf{3.} Vyplňte požadované informace o veřejném a privátním klíči:\\
~\\
\begin{tabular}{|r|c|c|c|}
\hline
~ & Název souboru & Oprávnění souboru \\
\hline
Veřejný klíč: & & \\
\hline
Privátní klíč: & & \\
\hline
\end{tabular}
\vspace{1em}

\textbf{4.} Do tabulky vyplňte hesla požadovaná při obou přihlášeních:\\
~\\
\begin{tabular}{|l|c|c|}
\hline
~ & user & root \\
\hline
vyžadované heslo při distribuci klíčů: & \hspace{3cm} & \hspace{3cm} \\
\hline
vyžadované heslo při druhém přihlášení: & \hspace{3cm} & \hspace{3cm} \\
\hline
\end{tabular}

\bigskip
K čemu slouží soubor \texttt{.ssh/authorized\_keys}?
~\\
\underline{\hspace{15cm}} \\
Proč nemohou uživatelé ze skupiny zapisovat do souboru \texttt{.ssh/authorized\_keys}?
~\\
\underline{\hspace{15cm}}

Bylo možné po neúspěšné autentizaci klíčem použit jinou alternativu? Jakou?
~\\
~\\
\underline{\hspace{15cm}}

\textbf{5.} Aplikovalo se omezené využití klíče? \underline{\hspace{1cm}} ~~~Pokud ano, uveďte, jak se projevilo:
~\\
~\\
~\\
\underline{\hspace{16.5cm}}
~\\


\textbf{6.} Bylo při opakovaném přihlášení nutné znovu zadávat heslo? \underline{\hspace{1cm}} \\


\section{Transport Layer Security (TLS)}

\textbf{1. Nezabezpečený přenos dat}\\
Je možné přečíst obsah komunikace? \underline{\hspace{1cm}}~~~Proč?~\underline{\hspace{8cm}}

~\\
\textbf{2. Přenos dat zabezpečený TLS} \\
Lze z komunikace zjistit jméno serveru? \underline{\hspace{1cm}}\\
~\\
Je možné přečíst obsah komunikace? \underline{\hspace{1cm}}~~~Proč?~\underline{\hspace{8cm}}\\
~\\
Jak se jmenuje použitá šifrovací sada? \underline{\hspace{7cm}}\\
~\\
Vyplňte informace o použité šifrovací sadě:\\

\renewcommand\arraystretch{1.3}
\begin{tabular}{|l|r|}
%\hline
%\textbf{Šifrovací sada (Cipher suite)}: & \hspace{25em} \\ \hline
\hline
Algoritmus pro výměnu klíčů: & \hspace{25.2em} \\ \hline
Algoritmus pro zajištění autentizace: & \\ \hline
Algoritmus pro šifrování přenosu: & \\ \hline
Délka šifrovacího klíče: & \\ \hline
Hešovací algoritmus: & \\ \hline
\end{tabular}
\renewcommand\arraystretch{1}
\vspace{0.5cm}

\textbf{3. Zabezpečený přenos dat TLS v prohlížeči} \\
~\\
Jak se jmenuje použitá šifrovací sada? \underline{\hspace{7cm}}\\
~\\
Vyplňte informace o certifikátu:\\

\renewcommand\arraystretch{1.3}
\begin{tabular}{|l|r|}
\hline
Vydavatel certifikátu: & \hspace{30em} \\ \hline
Doba platnosti: & \\ \hline
\end{tabular}
\renewcommand\arraystretch{1}
\vspace{0.5cm}
~\\
Příklady důvěryhodných CA: \underline{\hspace{11cm}}\\


\textbf{4. Certificate transparency} \\
~\\
K čemu certificate transparency log slouží? Jak informace z logu může využít správce?
~\\
~\\
~\\
\underline{\hspace{16cm}}
%~\\
%~\\
%Co obsahuje soubor \texttt{Atom}?~\underline{\hspace{12.2cm}}
%\section{Ukončení práce v laboratoři}
%\textbf{Nezapomeňte na závěr jako root spustit:} \texttt{/root/isa2/clean} \textbf{(!)}

\thispagestyle{empty}

\end{document}
%% END OF FILE

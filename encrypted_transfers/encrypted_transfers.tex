\documentclass[a4paper,11pt]{article}

\usepackage[utf8]{inputenc}
\usepackage[czech]{babel}
\usepackage[left=2cm,top=3cm,text={17cm,24cm}]{geometry}
\usepackage{graphicx}
\usepackage{listings}
\usepackage{url}
\usepackage{graphicx}
\graphicspath{ {./images/} }
\title{ISA - Laboratorní cvičení č.\,2\\
{\bf\large Zabezpečený přenos dat}}

\author{Vysoké učení technické v Brně}

\date{\url{https://github.com/nesfit/ISA/tree/master/encrypted_transfers}}

\begin{document}

{\let\newpage\relax\maketitle}

\section*{Cíl laboratorního cvičení:}
\begin{itemize}
  \item Základní seznámení se se synchronizací času a protokolem NTP.
  \item Naučit se pracovat s nástrojem SSH a správou klíčů.
  \item Porozumět komunikaci protokolem TLS.
  \item Seznámit se službou Certificate Transparency Log.
\end{itemize}

\section*{Pokyny}
\begin{itemize}
  \item Přihlaste se do OS GNU/Linux (F3), user/password {\tt user}/{\tt user4lab}.
  \item V případě potřeby se přepněte na uživatele {\tt root} příkazem {\tt su}
  (switch user), heslo {\tt root4lab}.
  \item V případě potřeby si otevřete další terminál v novém okně.
  \item Pro editaci konfiguračních souborů použijte libovolný editor (např.
  nano, mcedit, vim, gedit).
\item Pracujte ve dvojicích (váš počítač je dále v textu označen jako hXX a
  počítač souseda hYY, kde XX a YY je číslo počítače 01--20).
  %Zkontrolujte, že máte počítač propojen
  % přímým kabelem s přístupovým přepínačem (rozhraní enp2s0, zdířka E na patch
  %panelu).
  Pokud souseda nemáte, zapněte si jeden z volných počítačů.

\end{itemize}

%\newpage
%\section{Laboratorní úlohy}
\section{NTP}

Vaším úkolem je zajistit synchronizaci hodin počítače. V rámci bezpečnosti je
důležité provozovat počítač se správným časem kvůli časovým razítkům
omezujícím platnost klíčů, certifikátů či podpisů.

\begin{enumerate}
  \item Příkazem {\tt systemctl status chronyd.service} zjistěte, zda běží služba
    \texttt{chronyd} pro synchronizaci času pomocí protokolu NTP. Pomocí {\tt
    chronyc -n sources} zkontrolujte seznam serverů, které váš počítač používá
    pro synchronizaci času.

  \item Do protokolu vyplňte adresu nakonfigurovaného \textbf{NTP serveru} a jeho \textbf{hodnotu stratum}.

  \item \textit{(volitelně)} Pod uživatelem \texttt{root} otevřete
    konfiguračního soubor {\tt /etc/chrony.conf}. V souboru
    nahraďte využívání poolu serveru Cent OS ({\tt *.centos.pool.ntp.org}) za
    server na FIT ({\tt ntp.fit.vutbr.cz}).

    Po úpravě restartujte démona pro NTP příkazem {\tt systemctl restart chronyd.service}.
    Ověřte pomocí příkazu {\tt systemctl status chronyd.service}, že služba běží. Pokud
    neběží, zobrazte chybu příkazem {\tt journalctl -u chronyd}. Chybu v konfiguraci opravte a službu {\tt chronyd} restartujte. Zadejte znovu {\tt chronyc -n sources} a zkontrolujte, zda je čas správně synchronizován s NTP serverem na FIT.

\end{enumerate}

\section{Vzdálený terminál -- SSH, Secure Shell}

%\noindent {\bf V průběhu řešení jsou tučně vyznačené otázky, na které si připravte odpovědi.
%  Na konci úkolu je sdělíte cvičícímu.}
%~\\~\\
Poznámka: namísto řetězce {\tt <login>} používejte v této úloze své studentské přihlašovací jméno, například {\tt xstudent00}.

\begin{enumerate}

  \item Otevřete si dvě terminálová okna: příkazovou řádku pro uživatele {\tt user} a další
    příkazovou řádku pro uživatele {\tt root} příkazem {\tt su} (switch user).
Pomocí příkazu {\tt whoami} vypište jméno aktivního uživatele. 

    V~obou otevřených terminálech dočasně vypněte podporu pro SSH agenta:
  \begin{lstlisting}
  [user@hXX]$ unset SSH_AUTH_SOCK
  [root@hXX]# unset SSH_AUTH_SOCK
  \end{lstlisting}

  Pokud budete otevírat v průběhu řešení úkolu nové terminály, nezapomeňte
  v nich vypnout podporu agenta SSH.

  \item {\bf Bezpečné připojení na vzdálený počítač bez autentizačních klíčů}

    \begin{enumerate}

      \item Spusťte program Wireshark a zachytávejte komunikaci na portu 22
        (rozhraní enp2s0).

      \item Přihlaste se službou {\tt ssh} na počítač souseda hYY (např. h01) příkazem \\
        {\tt ssh user@hYY.netlab.fit.vutbr.cz}. Zadejte heslo \textbf{user4lab}.

      \item Na serveru zadejte libovolný příkaz, který znáte (např. zobrazte
        obsah manuálové stránky ssh ({\tt man ssh}), vypište
        obsah adresáře ({\tt ls}) či login aktuálního uživatele ({\tt whoami}).

      \item Příkazem {\tt exit} nebo stiskem {\tt Ctrl-D} spojení ukončete.

      \item V programu Wireshark zobrazte obsah komunikace (pomocí Follow TCP stream).
        Podívejte se, \textbf{co lze z uvedené komunikace vyčíst}. Jsou v komunikaci vidět
        \textbf{zadávané příkazy a jejich výstup?} Vaše odpovědi zapište do protokolu.
    \end{enumerate}

  \item {\bf Vytvoření veřejného a privátního klíče}

    \begin{enumerate}

      \item Jako uživatel {\tt user} vygenerujte příkazem \verb|$ ssh-keygen -C <login>@user|
        implicitní klíč pro uživatele
        {\tt user}. Ponechte implicitní názvy souborů pro uložení klíčů. Zvolte heslo o délce alespoň osmi znaků, například \texttt{fitvutisa}.

      \item Jako uživatel {\tt root} vygenerujte příkazem \verb|# ssh-keygen -N "" -C <login>@root|
        implicitní klíč pro uživatele {\tt root} bez hesla.

      \item Ověřte obsah a přístupová práva u nově vzniklých souborů (\verb|ls -l ~/.ssh|).

      \item Do protokolu uveďte \textbf{v jakých souborech} jsou uloženy klíče a~jak je nastaveno \textbf{oprávnění} souborů (unixová práva).

    \end{enumerate}

  \item {\bf Distribuce klíčů}

    \begin{enumerate}

      \item Oba veřejné klíče zkopírujte na vzdálený počítač hYY do
        souboru \verb|.ssh/authorized_keys|. Můžete využít následující postup. 
        \begin{itemize}
          \item Klíč uživatele {\tt user} zkopírujte následujícím příkazem: \\
            {\verb&cat ~user/.ssh/*.pub | ssh user@hYY "cat >> .ssh/authorized_keys"&}
          \item Na vzdáleném počítači nastavte správně přístupová práva k souboru:\\
            {\verb&ssh user@hYY "chmod g-w .ssh/authorized_keys"&} 
          \item Zkopírujte na vzdálený počítač také veřejný klíč uživatele {\tt root}: \\
            {\verb&cat /root/.ssh/*.pub | ssh root@hYY "cat >> .ssh/authorized_keys"&}
        \end{itemize}

      \item Zkuste se znovu přihlásit jako \texttt{user} i jako \texttt{root} na daný vzdálený počítač hYY.

      \item Do tabulky v~protokolu vyplňte, \textbf{jaká hesla bylo nutné zadat}. Vysvětlete význam souboru \texttt{authorized\_keys}.

      \item Zkuste \textbf{opakovaně zadat špatné heslo}. Do protokolu napište, co se stalo.
      Při experimentu použijte režim verbose (\texttt{ssh -v}).

    \end{enumerate}

  \item {\bf Omezení použití klíčů}

    Nyní omezíme použití klíče uživatele {\tt root} pouze na spuštění konkrétního příkazu na vzdáleném počítači.
    \begin{enumerate}
      \item Přihlaste se jako uživatel {\tt root} na sousední počítač hYY, kam jste nakopírovali své veřejné klíče. Na vzdáleném počítači upravte soubor s autorizovanými veřejnými klíči tak, že na začátek řádku s klíčem uživatele {\tt root} (řádek poznáte tak, že končí řetězcem {\tt <login>@root}) napíšete příkaz 
        \verb|command="chronyc -n sources" | následovaný mezerou a původním obsahem řádku.
      \item Odhlaste se ze vzdáleného počítače.
      \item Znovu se přihlaste na sousední počítač hYY z účtu {\tt root} jako \texttt{root}. \textbf{K čemu došlo?} Zapište výsledek operace do protokolu.
    \end{enumerate}


  \item {\bf Opakované použití klíče zabezpečeného heslem}

    \begin{enumerate}

      \item Otevřete si nový unixový terminál pod uživatelem {\tt user}. Pomocí příkazu \verb|env| ověřte, jak je nastavená proměnná \verb|SSH_AUTH_SOCK|.
      \item Přihlaste se znovu k sousednímu počítači hYY.
      \item Odhlaste se a přihlašte ještě jednou na sousední počítač. \textbf{Jak proběhlo přihlášení?} Výsledek zapište do protokolu.
    \end{enumerate}
\end{enumerate}

\section{Zabezpečení transportní vrstvy TLS, Transport Layer Security}

%{\bf V průběhu řešení jsou tučně vyznačené otázky, na které si připravte odpovědi.
%  Na konci úkolu je sdělíte cvičícímu.}

\begin{enumerate}

  \item {\bf  Nezabezpečený přenos dat}
    \begin{enumerate}
      \item Spusťte program Wireshark a zachytávejte komunikaci na portu 80.
      \item Pomocí programu {\tt telnet} se připojte k fakultnímu webovému
        serveru: \\ \verb|telnet nes.fit.vutbr.cz 80|.
      \item V programu Wireshark pozorujte navázání spojení TCP pomocí
        trojcestné synchronizace.
      \item Zašlete serveru požadavek protokolem HTTP, např. \verb|GET / HTTP/1.0|. Dotaz ukončete prázdným řádkem. V terminálu pozorujte   odpověď.
      \item Zobrazte komunikace HTTP v programu Wireshark.  Do~protokolu uveďte, zda je možné \textbf{přečíst obsah komunikace}.
    \end{enumerate}

  \item {\bf Přenos dat zabezpečený protokolem TLS}

    \begin{enumerate}
      \item Spusťte program Wireshark a zachytávejte komunikaci na portu 443.
      \item Pomocí programu {\tt openssl} se připojte k fakultnímu webovému
        serveru: \\ \verb|openssl s_client -connect www.fit.vutbr.cz:443 -tls1_2|.
        Všimněte si řádku \emph{Verify return code}. Co říká o certifikátu?

      \item V programu Wireshark pozorujte navázání spojení TCP a TLS. Jak je možné zjistit z komunikace \textbf{jméno serveru}? Zapište odpověď do protokolu.

      \item Pomocí programu \texttt{openssl} v rámci navázaného spojení zašlete serveru požadavek protokolem HTTP, např. \verb|GET / HTTP/1.0|. Dotaz ukončete prázdným řádkem. V terminálu pozorujte odpověď.

      \item Zobrazte tuto komunikaci programem Wireshark.  Do~protokolu zapište, zda je možné či nikoliv \textbf{přečíst obsah komunikace}, případně proč. 

      \item Z výstupu aplikace \texttt{openssl} určete, jaká \textbf{šifrovací sada (cipher suite)} se používá.
      Identifikátor šifrovací sady zapište do protokolu.

      \item Na stránce \url{https://ciphersuite.info/} vyhledejte detaily k dané šifrovací sadě. Do protokolu \textbf{vyplňte názvy použitých algoritmů pro zabezpečení komunikace TLS}.
    \end{enumerate}

  \item {\bf Zabezpečený přenos dat TLS v prohlížeči}

    \begin{enumerate}

      %\item Spusťte program Wireshark, zachytávejte komunikaci na portu 443.

      \item V prohlížeči Firefox otevřete stránku  \verb|wis.fit.vutbr.cz|.

      \item Klikněte na ikonku zámku nalevo od URL a~zjistěte, jaká \textbf{šifrovací
      sada} je použita u připojení na daný server. Zobrazte \textbf{informace o použitém certifikátu}.
      Požadované údaje vyplňte do protokolu.

      \item V menu \emph{Edit » Settings » Privacy \& Security} v sekci
        \emph{Certificates} zobrazte certifikáty autorit, kterým důvěřujete. {\bf Vypište názvy alespoň tří z nich} do protokolu.

      %\item V programu Wireshark pozorujte navázání spojení TCP a TLS. {\bf Jaké
      %  informace lze vyčíst z~handshaku TLS? Je možné zjistit jméno serveru?
      %  Nalezené jméno serveru na konci úkolu ukažte cvičícímu}.

    \end{enumerate}

  \item {\bf Certificate Transparency}

    \begin{enumerate}

      \item Příkazem \verb|host -t CAA fit.vutbr.cz| si zobrazte certifikační
        autority oprávněné vydávat certifikáty pro doménu \url{fit.vutbr.cz}.

      \item V prohlížeči se připojte na stránku
        \url{https://certstream.calidog.io/}, klikněte na tlačítko \emph{Open
        Fire Hose}. Pozorujte, \textbf{jaké informace} se zde zobrazují.
        Zamyslete se, k čemu mohou tyto informace sloužit, případně jak se dají zneužít. {\bf Zapište výsledek do protokolu.}
      \item V prohlížeči se připojte na stránku \url{https://crt.sh}.

      \item Zobrazte certifikáty vydané pro servery v  doméně \emph{kralovopole.brno.cz}. Podívejte se na jejich obsah, vydavatele a platnost. 

%      \item Najděte na stránce ikonu směřují k souboru \textbf{Atom}.
%      Podívejte se, co soubor obsahuje a zodpovězte otázku v protokolu.
%      Zamyslete se, k čemu by se takový obsah dal použít.
    \end{enumerate}

\end{enumerate}


\section{Ukončení práce v laboratoři}
\begin{itemize}
  \item Odevzdejte vypracovaný a podepsaný protokol vyučujícímu. 
  \item Jako uživatel \texttt{root} vypněte počítač dávkou {\tt /root/isa2/clean}.
\end{itemize}

\end{document}
%% END OF FILE

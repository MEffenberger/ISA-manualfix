\documentclass[a4paper,11pt]{article}

\usepackage[utf8]{inputenc}
\usepackage[czech]{babel}
\usepackage[left=2cm,top=3cm,text={17cm,24cm}]{geometry}
\usepackage{graphicx}
\usepackage{listings}
\usepackage{url}

\title{Správa a monitorování sítě\\
{\bf\large ISA - Laboratorní cvičení č.5}\\
{\bf\large Protokol ke cvičení}}

\author{Vysoké učení technické v Brně}

\date{\url{https://github.com/nesfit/ISA/tree/master/lab5-management}}

\setlength\parindent{0pt}

\begin{document} 

{\let\newpage\relax\maketitle}

Jméno a příjmení:\\
Login:\\
Skupina (číslo nebo čas):\\
Datum:\\

\section{Logování událostí v systému Syslog}

\begin{itemize}
    \item [1.10] Vypište IP adresy a porty klienta a serveru Syslog. Uveďte transportní protokol služby Syslog.
    \begin{itemize}
        \item IP adresa a port serveru:
        \item IP adresa a port klienta:
        \item Transportní protokol: 
    \end{itemize}
    \item [1.11] Vypište hodnoty políček {\tt FACILITY} a {\tt SEVERITY} ve vámi vybraném paketu Syslog.
    \vspace{15mm}
\end{itemize}

\section{Monitorování toků NetFlow}

\begin{itemize} 
\item[2.3-2.4] Vypište tři stanice s největším počtem odeslaných a přijatých bytů.  

\begin{tabular}{|p{2cm}|p{3cm}|c||p{2cm}|p{3cm}|c|}
  \hline
  \multicolumn{3}{|c||}{Top 3 IPs by sent bytes} & \multicolumn{3}{|c|}{Top 3 IPs received bytes} \\\hline
IP adresa  & Přeneseno bytů  & \% provozu & IP adresa  & Přeneseno bytů  & \% provozu \\ \hline
      &                 &           &&&   \\   \hline
      &                 &           &&&   \\  \hline
      &                 &           &&&   \\   \hline
\end{tabular}

\item[2.6] Vypište tři nejvytíženější porty podle počtu přenesených toků a bytů: 

\begin{tabular}{|p{2cm}|p{3cm}|c||p{2cm}|p{3cm}|c|}
\hline
  \multicolumn{3}{|c||}{Top 3 ports by flows} & \multicolumn{3}{|c|}{Top 3 ports by bytes} \\\hline
Port  & Počet toků  & \% provozu & Port & Počet bytů & \% provozu \\ \hline
      &                  &       &&& \\  \hline
      &                  &       &&& \\  \hline
      &                  &       &&& \\  \hline
\end{tabular}
\end{itemize}

\section{Monitorování sítě pomocí SNMP}
\begin{itemize}
    \item [3.1] Základní informace o systému:
    \begin{itemize}
        \item Jméno systému:
         \vspace{3mm}
        \item Název a verze OS:
         \vspace{3mm}
        \item Doba běhu systému (ve dnech, hodinách, min a sec):
         \vspace{3mm}
    \end{itemize}
    \item [3.2] Vypište IP adresu a MAC adresu tří sousedních počítačů získané ze SNMP:
    
\begin{tabular}{|p{5cm}|p{7cm}|}
\hline
IP adresa  & MAC adresa   \\ \hline
      &     \\   \hline
      &     \\   \hline
      &     \\   \hline
\end{tabular}
    \item[3.3] Název síťového rozhraní na serveru {\tt isa2} připojeného do internetu: 
\end{itemize}

\section{Monitorování sítě pomocí ICMP}
\begin{itemize}
  \item [4.1] Zapište průměrnou odezvu serveru \texttt{google.com} (v ms):
  \item [4.2] Vypište velikost hlaviček IP, ICMP a maximální velikost obsahu zachycených ICMP paketů:

    \vspace{2mm}
    \begin{tabular}{|c|c|c|c|p{3cm}|}  \hline
     Protokol  & Hlavička IP (B)  & Hlavička ICMP (B)  & Obsah ICMP (B) & Součet (B) \\ \hline
     IPv4      &     &       &         &   \\   \hline
     IPv6      &     &       &         &     \\  \hline
\end{tabular}

    \vspace{2mm}
    \item [4.4] Zapište posloupnost ISP a čísel jejich AS na trase z vašeho PC do zadané destinace (AS-Path):
        \begin{itemize}
            \item \texttt{google.com}:
            \vspace{3cm}
            \item \texttt{idnes.cz}:
            \vspace{3cm}
        \end{itemize}
\end{itemize}
\end{document}

\documentclass[a4paper,11pt]{article}

\usepackage[utf8]{inputenc}
\usepackage[czech]{babel}
\usepackage[left=2cm,top=3cm,text={17cm,24cm}]{geometry}
\usepackage{graphicx}
\usepackage{listings}
\usepackage{url}
\usepackage{graphicx}
\graphicspath{ {./images/} }
\title{ISA - Laboratorní cvičení č.\,2\\
{\bf\large Zabezpečený přenos dat}}

\author{Vysoké učení technické v Brně}

\date{\url{https://github.com/nesfit/ISA/tree/master/encrypted_transfers_online}}

\begin{document}

{\let\newpage\relax\maketitle}

\section*{Cíl laboratorního cvičení:}
\begin{itemize}
  \item Základní seznámení se synchronizací času a protokolem NTP.
  \item Naučit se práci s nástrojem SSH a správu klíčů.
  \item Naučit se základy protokolu TLS.
  \item Seznámit se s Certificate Transparency Log.
\end{itemize}

\section*{Pokyny}
\begin{itemize}

  \item Pro práci v cvičení budeme používat virtuální stroje v programu VirtualBox\footnote{\url{https://nes.fit.vutbr.cz/isa/ISA2020.ova}}.

  \item Cvičení předpokládá funkční interní síť s nakonfigurovanými IPv4 adresami mezi dvěma virtuálními stroji z prvního cvičení (Sekce 3.).

  \item Odpovědi pište do odpovědního archu \texttt{protokol.md} který odevzdáte do WIS-u.
    Dostupný je na adrese \url{https://github.com/nesfit/ISA/blob/master/encrypted_transfers_online/protokol.md}.

  \item Do WIS-u budete také odevzdávat všechny zachycené \texttt{pcap} soubory.

  \item Nastartujte oba stroje a přihlaste se do OS GNU/Linux, user/password {\tt user}/{\tt user4lab}.

  \item V případě potřeby se přepněte na uživatele {\tt root} příkazem {\tt su} (switch user), heslo {\tt root4lab}.

  \item V případě potřeby si otevřete další terminál v novém okně.

  \item Pro editaci konfiguračních souborů použijte libovolný editor (např. nano, vim, gedit).

\end{itemize}

%\newpage
%\section{Laboratorní úlohy}
\section{NTP}

Vašim úkolem je zajistit synchronizaci hodin počítače.
V rámci bezpečnosti je důležité provozovat počítač se správným časem kvůli časovým razítkům označujícím platnost a neplatnost klíčů, certifikátů, podpisů apod.

\begin{enumerate}

  \item Zkontrolujte obsah konfiguračního souboru {\tt /etc/ntp.conf}.
    V souboru by měli být použity pooly serveru CentOS ({\tt *.centos.pool.ntp.org}).

  \item Zapněte démona pro NTP ({\tt systemctl start ntpd.service}).
    Ověřte pomocí {\tt systemctl status ntpd.service}, že služba běží;
    pokud neběží, zobrazte chyby pomocí příkazu {\tt journalctl -u ntpd}, nalezené chyby opravte a službu {\tt ntpd} restartujte.

  \item Sledujte postup synchronizace příkazem {\tt watch ntpq -p}.
    Nemusíte čekat na dokončení synchronizace, nechte okno spuštěné a pokračujte dalšími úkoly.

\end{enumerate}

\section{Vzdálený terminál -- SSH, Secure Shell}

Namísto řetězce {\tt <login>} používejte své studentské přihlašovací jméno.

\begin{enumerate}

  \item Otevřete si dvě okna: příkazovou řádku pro uživatele {\tt user} a další příkazovou řádku pro uživatele {\tt root} příkazem {\tt su} (switch user).
Pomocí příkazu {\tt whoami} vypište jméno aktivního uživatele, ověřte, že je očekávané.

    V~obou otevřených terminálech dočasně vypněte podporu pro SSH agenta:
  \begin{lstlisting}
  [user@localhost]$ unset SSH_AUTH_SOCK
  [root@localhost]# unset SSH_AUTH_SOCK
  \end{lstlisting}

  Pokud budete otevírat v průběhu řešení úkolu nové terminály, nezapomeňte
    v nich také vypnout podporu agenta SSH.

  \item {\bf Bezpečné připojení na vzdálený počítač bez autentizačních klíčů.}

    \begin{enumerate}

      \item Spusťte program wireshark a zachytávejte komunikaci na portu 22 (rozhraní enp0s8).

      \item Přihlaste se příkazem {\tt ssh user@<IP adresa druhého stroje>} a zadejte heslo \textbf{user4lab}.

      \item Na serveru zadejte libovolný příkaz, který znáte (např. zobrazte obsah manuálové stránky ssh příkazem {\tt man ssh}, vypište obsah adresáře příkazem {\tt ls}, obsah souborů příkazem {\tt ls}, aktuálního uživatele příkazem {\tt whoami} apod.).

      \item Příkazem {\tt exit} nebo stiskem {\tt Ctrl-D} spojení ukončete.

      \item V programu Wireshark zobrazte obsah komunikace (follow TCP stream) a zjistěte informace o spojení --- verze programu {\tt ssh} na serveru i klientovi, podporované šifry, autentizační mechanismus HMAC, mechanismus pro výměnu klíčů.
        Ověřte, že pomocí Wiresharku nevidíte příkazy zadané v předchozích bodech a jejich výstup.
        Můžete na základě zachycené komunikace říct něco o jejím obsahu?
        Wireshark nechejte dále zachytávat komunikaci.

    \end{enumerate}

  \item {\bf Vytvoření veřejného a privátního klíče.}

    \begin{enumerate}

      \item Jako uživatel {\tt user} vygenerujte příkazem \verb|$ ssh-keygen -C <login>@user| implicitní klíč pro uživatele {\tt user}.
        Neměňte jeho název a zvolte heslo o délce alespoň osmi znaků, například \textbf{fitvutisa}.

      \item Jako uživatel {\tt root} vygenerujte příkazem \verb|# ssh-keygen -N "" -C <login>@root| implicitní klíč pro uživatele {\tt root} bez hesla.

      \item Ověřte obsah a přístupová práva u nově vzniklých souborů (\verb|ls -l ~/.ssh|).
        Jak se liší práva mezi souborem s privátním a veřejným klíčem?

    \end{enumerate}

  \item {\bf Distribuce klíčů}

    \begin{enumerate}

      \item Oba veřejné klíče zkopírujte na druhý počítač do souboru \verb|.ssh/authorized_keys| pro klíč uživatele {\tt user} např.: \\
        {\verb&cat ~user/.ssh/*.pub | ssh user@<IP> "cat >> .ssh/authorized_keys"&} \\
        pro klíč uživatele {\tt root} např.: \\
        {\verb&cat /root/.ssh/*.pub | ssh root@<IP> "cat >> .ssh/authorized_keys"&}. \\

        Pokud složka {\tt .ssh} na vzdáleném stroji neexistuje, příkaz selže.
        V takovém případě se přihlaste pomocí hesla nebo přepněte do druhého stroje a složku vytvořte.
        Můžete také rovnou vytvořit soubor {\tt .ssh/authorized\_keys} (např. příkazem {\tt touch}).
        {\bf Je důležité aby složka měla přístupová práva {\tt 700} a soubor {\tt 600}, jinak nebude možné zkopírované klíče použít.}\\

      Jaká hesla bylo nutné zadat?

      \item Zkuste se znovu přihlásit na druhý počítač.
        Jaké heslo bylo nyní nutné zadat?
        Zkuste zadat (opakovaně) špatné heslo a pozorujte, co se stalo.
        Při experimentech můžete také využít tzv. verbose režim ssh ({\tt ssh -v}).

    \end{enumerate}

  \item {\bf Omezení použití klíčů}

    Nyní bude naším cílem omezit použití klíče uživatele {\tt root}, který není chráněn heslem tak, aby pomocí něj bylo možné na vzdáleném serveru vykonat pouze konkrétní příkaz.

    \begin{enumerate}

      \item Přihlaste se jako uživatel {\tt root} na druhý počítač, kam jste nakopírovali své veřejné klíče a upravte soubor s autorizovanými veřejnými klíči tak, že na začátek řádku s klíčem uživatele {\tt root} (řádek poznáte tak, že končí řetězcem
        {\tt <login>@root}) napíšete \\
        \verb|command="ntpq -p" | \\
        (následovaný jednou mezerou a původním obsahem řádku).

      \item Odhlaste se ze vzdáleného počítače a znovu se na něj přihlaste z účtu {\tt root} jako {\tt root}.
        Aplikovalo se omezené využití klíče?

    \end{enumerate}


  \item {\bf Pohodlné opakované použití klíče zabezpečeného heslem.}

    \begin{enumerate}

      \item Ukončete terminál uživatele {\tt user} a vytvořte nový.
        Pomocí příkazu \verb|env| ověřte, že je nastavená proměnná \verb|SSH_AUTH_SOCK|.

      \item Přihlaste se na druhý počítač.
        Museli jste zadávat znovu heslo?

      \item Zachycenou komunikaci uložte do souboru {\tt cv2-ssh.pcap} který budete odevzdávat.

    \end{enumerate}

\end{enumerate}

\section{Zabezpečení transportní vrstvy -- TLS, Transport Layer Security}

Pro tento úkol doinstalujte balíček {\tt telnet} příkazem {\tt sudo yum install telnet}.

\begin{enumerate}

  \item {\bf  Nezabezpečený přenos dat}

    \begin{enumerate}

      \item Spusťte program Wireshark, zachytávejte komunikaci na portu 80 na rozhraní enp0s3.

      \item Pomocí programu {\tt telnet} se připojte k fakultnímu webovému serveru: \\
        \verb|telnet www.fit.vut.cz 80|.

      \item V programu Wireshark pozorujte navázání spojení TCP pomocí trojcestného handshaku.

      \item Zašlete serveru požadavek protokolem HTTP, např.:
        \verb|GET / HTTP/1.0|, dotaz ukončete prázdným řádkem.
        V terminálu pozorujte odpověď.

      \item Zobrazte si v programu Wireshark komunikaci pomocí HTTP.
        Je možné přečíst obsah komunikace?

      \item Zachycenou komunikaci uložte do souboru {\tt cv2-http.pcap} který budete odevzdávat.

    \end{enumerate}

  \item {\bf Přenos dat zabezpečený TLS}

    \begin{enumerate}

      \item Spusťte program Wireshark, zachytávejte komunikaci na portu 443 na rozhraní enp0s3.

      \item Pomocí programu {\tt openssl} se připojte k fakultnímu webovému serveru: \\
        \verb|openssl s_client -quiet -connect www.fit.vut.cz:443|.
        Všimněte si řádku \emph{Verify return code}, co říká o certifikátu?

      \item V programu Wireshark pozorujte navázání spojení TCP a TLS. Jaké
        informace lze vyčíst z~handshaku TLS? Je možné zjistit jméno serveru?

      \item Zašlete serveru požadavek protokolem HTTP, např.:
        \verb|GET / HTTP/1.0|, dotaz ukončete prázdným řádkem.
        V terminálu pozorujte odpověď.

      \item Zobrazte si v programu Wireshark komunikaci pomocí HTTP s využitím TLS.
        Je možné přečíst obsah komunikace?

      \item Zachycenou komunikaci uložte do souboru {\tt cv2-tls.pcap} který budete odevzdávat.

    \end{enumerate}

  \item {\bf Zabezpečený přenos dat TLS v prohlížeči}

    \begin{enumerate}

      \item Spusťte program Wireshark, zachytávejte komunikaci na portu 443.

      \item Spusťte prohlížeč Firefox a otevřete v něm stránku \verb|wis.fit.vutbr.cz|.

      \item Kliknutím na informace o certifikátu (vlevo od URL) zobrazte informace o použitém certifikátu.

      \item V menu \emph{Preferences » Privacy \& Security} v sekci \emph{Certificates} zobrazte důvěryhodné certifikáty.
        Prohlédněte si seznam a odhadněte jejich počet.

      \item V programu Wireshark pozorujte navázání spojení TCP a TLS.
        Jaké informace lze vyčíst z~handshaku TLS?
        Je možné zjistit jméno serveru?

      \item Zachycenou komunikaci uložte do souboru {\tt cv2-https.pcap} který budete odevzdávat.

    \end{enumerate}

  \item {\bf Certificate Transparency}

    \begin{enumerate}

      \item Příkazem \verb|host -t CAA fit.vut.cz| si zobrazte certifikační autority oprávněné vydávat certifikáty pro doménu \url{fit.vut.cz}.

      \item V prohlížeči se připojte na stránku \url{https://certstream.calidog.io/}, klikněte na tlačítko \emph{Open Fire Hose} a pozorujte nově vydávané certifikáty.
        K čemu si myslíte, že
        můžou být tato data dobrá?
        Myslíte si, že je možné data zneužít?

      \item V prohlížeči se připojte na stránku \url{https://crt.sh}.

      \item Nalezněte certifikáty vydané pro servery v rámci domény \emph{fit.vut.cz}, použijte znak \verb|%| jako zástupný znak.

      \item Najděte na stránce ikonu směřují k souboru Atom odkazující na soubor obsahující vystavené certifikáty.
        {\bf Zamyslete se k čemu byste tento soubor využili}.

    \end{enumerate}

\end{enumerate}

\section*{Odevzdávané soubory}
Zkontrolujte, zda máte všechny soubory které se budou odevzdávat:
\begin{itemize}
  \item \texttt{protokol.md}
  \item \texttt{cv2-ssh.pcap}
  \item \texttt{cv2-http.pcap}
  \item \texttt{cv2-tls.pcap}
  \item \texttt{cv2-https.pcap}
\end{itemize}

\section{Ukončení cvičení}
Do WIS-u odevzdejte vyplněný \texttt{protokol.md} a všechny zachycené \texttt{pcap} soubory.
Vypňete virtuální stroje a obnovte snapshot ze začátku.

\end{document}
%% END OF FILE
